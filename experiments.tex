\section{Preliminary Experiments}
\label{sec:experiment}

%% by comparing applying a model checker to original C language programs
%% with to abstracted behavior.

We conducted preliminary experiments to check feasibility and to
investigate potential issues in the current framework.  For the
following C programs, we extract the behavioral types manually in the
form of another C programs.
\begin{itemize}

\item \texttt{poker.c}: a program changed according to the program
  \texttt{person.c} from website
  (http://c.learncodethehardway.org/book/ex16.html), which models a
  poker game. It random  We rewrite this program as \texttt{poker\_rw.c} without
  changing its meaning.

\item \texttt{database.c}: from website
  (http://c.learncodethehardway.org/book/ex17.html), modeling a
  database. It opens a database and performs some operations on the
  database (retrieve, delete and update data, or create a new
  database) and closes the database. We rewrite this program as
  \texttt{database\_rw.c} without changing its meaning.
\item \texttt{gen\_init\_cpio.c} is a file from Linux kernel
  \texttt{/usr/gen\_init\_cpio.c}. It produces a binary file which
  performs on a file containing newline separated entries that
  describe the files to be included in the initramfs archive. Also, we
  rewrite this program to \texttt{gen\_init\_cpio\_rw.c}.
  \item \texttt{decompress\_unlzo.c} is a file from Linux kernel
    \texttt{/lib/decompress\_unlzo.c}. This is a LZO decompressor for
    the Linux kernel. We rewrite it to the file
    \texttt{decompress\_unlzo\_rw.c}
\end{itemize}

Table~\ref{tb:mcc} and Table~\ref{tb:mca} give the result.  All of the
experiments are done on a machine with an Intel(R) Core(TM) i7-3770
CPU @ 3.40GHz, 8MB cache and 3.76GB memory, running on Debian (kernel
version 2.6.32-5-amd64) and CPAchecker (version 1.3.4).

\begin{table}
  \scriptsize
\begin{tabular}{|c|c|c|c|c|c|c|}
\hline
& \multicolumn{6}{|c|}{\texttt{original programs}}  \\
\hline
 & $\sharp$\texttt{loc} & $\sharp$\texttt{fun} & \texttt{cpu time (total,sec)} & \texttt{memory (MB)} & \texttt{fixed num}& \texttt{verified result} \\
\hline
\texttt{poker.c} & 86 & 4 & 2.700 & 2797 & 4  & \texttt{TRUE}  \\
\hline
\texttt{poker\_rw.c} & 89 & 4 & 2.740 & 2800 & 4  & \texttt{TRUE}  \\
\hline
\texttt{database.c} & 153 & 10 & 12.010 & 2907 & 2  & \texttt{TRUE}  \\
\hline
\texttt{database\_rw.c} & 151 & 10 & 7.080 & 2907 & 2  & \texttt{TRUE}  \\
\hline
\texttt{gen\_init\_cpio.c} & 346 & 19 & 9.580 & 2809 & 2  & \texttt{TRUE}  \\
\hline
\texttt{gen\_init\_cpio\_rw.c} & 343 &19  & 4.850  & 2744  & 2  & \texttt{TRUE}  \\
\hline
\texttt{decompress\_unlzo.c} & 162 & 2  & 3.000  & 2806  & 2  & \texttt{TRUE}  \\
\hline
\texttt{decompress\_unlzo\_rw.c} & 92 & 2  & 2.650  & 2800  & 2  & \texttt{TRUE}  \\

\hline
\end{tabular}
\caption{Model checking on original C language programs}
\label{tb:mcc}
\end{table}

\begin{table}
  \scriptsize
\begin{tabular}{|c|c|c|c|c|c|c|}
\hline
&\multicolumn{6}{|c|}{abstracted behavior} \\
\hline
 &$\sharp$\texttt{loc} & $\sharp$\texttt{fun} & \texttt{cpu time (total,sec)} & \texttt{memory (MB)} & \texttt{fixed num} & \texttt{verified result} \\
\hline
\texttt{poker.c} & 16 & 4 & 1.980 & 2803 & 4  & \texttt{FALSE}  \\
\hline
\texttt{poker\_rw.c} & 18 & 4 & 2.020 & 2798 & 4  & \texttt{TRUE}  \\
\hline
\texttt{database.c} &  16 & 4 & 2.060 & 2800 & 2 & \texttt{FALSE} \\
\hline
\texttt{database\_rw.c} &  18 & 4 & 1.990 & 2737 & 2 & \texttt{TRUE} \\
\hline
\texttt{gen\_init\_cpio.c} & 16 & 4 & 2.020 & 2802 & 2  & \texttt{FALSE}  \\
\hline
\texttt{gen\_init\_cpio\_rw.c} & 18 & 4 & 2.000  & 2742  & 2  & \texttt{TRUE}  \\
\hline
\texttt{decompress\_unlzo.c} & 16 & 4 & 1.970  & 2738  & 2  & \texttt{FALSE}  \\
\hline
\texttt{decompress\_unlzo\_.c} & 18 & 4  & 2.000  & 2796  & 2  & \texttt{TRUE}  \\

\hline
\end{tabular}
\caption{Model checking on an abstracted behavior}
\label{tb:mca}
\end{table}

In these two tables, the meaning of each column is described as
follows. The column \texttt{original programs} and \texttt{abstracted
  behavior} mean applying the model checker to a original C language
program and to abstracted behavior respectively. These two columns
consist of several columns: $\sharp$\texttt{loc} means the number of
program locations; $\sharp$\texttt{fun} means the number of functions
in a program; \texttt{cpu time (total)} means the total execution time
of CPU in seconds; \texttt{memory(MB)} means the number of virtual
memory cells consumed by model checker in \texttt{MByte};
\texttt{fixed num} means the number of available memory cells for
allocation; the \texttt{TRUE} in the \texttt{verified result} column
means the number of the memory cells which a program consumes does not
exceed the \texttt{fixed num}, otherwise it is \texttt{FALSE} which
denotes overflow.

The results present in the these two tables show that the
resources required for model checking become smaller if applying model
checkers to the abstracted behavior, since the abstracted behavior
only consists of allocation and deallocation.

One thing we should notice in these two tables is that for the same
programs, the \texttt{verified result} should be the same when model
checking on original programs and abstracted behavior; but it is not,
for example, \texttt{database.c} has the \texttt{TRUE} in the
Table~\ref{tb:mcc} but the \texttt{FALSE} in the
Table~\ref{tb:mca}. The reason is that our approach, abstracted
behavior, may not correctly deal with some conditional statements
about allocation and deallocation. For example, the behavior of
\texttt{database.c} is $\mu\alpha.\Malloc;\Malloc;(((\Free + 0);\Free)
+ 0);\alpha)$, because of the choice type, it may perform like
$\mu\alpha.\Malloc;\Malloc;0;0;\alpha$, which means consuming unbound
number of memory cells. The rewritten \texttt{database\_rw.c}, which
does not change the semantics of original program, has the behavior
$\mu\alpha.(\Malloc;(\Malloc + \Free) + 0);\Free;\alpha)$ which
returns \texttt{TRUE} when doing model checking on it.


%% \begin{table}
%% \tiny
%% \begin{tabular}{|c|c|c|c|c|c|c|c|c|c|c|}
%% \hline
%% & \multicolumn{5}{|c|}{original programs} & \multicolumn{5}{|c|}{abstracted behavior} \\
%% \hline
%%  & $\sharp$loc & $\sharp$fun & cpu time (total) & memory (MB) & fixed num & $\sharp$loc & $\sharp$fun & cpu time (total) & memory (MB) & fixed num \\
%% \hline
%% linklist.c & 154 & 13 & 9.770 & 2943 & 6(true) & 20 & 4 & 3.190 & 2918 & 6(false) \\
%% \hline
%% linklst2.c & 140 & 13 & 25.620 & 2955 & 21(true) & 20 & 4 & 10.72 & 2945 & 21(false) \\
%% \hline
%% linkstack.c  & 87 & 10 & 10.830 & 2941 & 11(true) & 20 & 4 & 4.990 & 2916 & 11(false) \\
%% \hline
%% linkqueue.c & 119 & 11 & 13.110 & 2939 & 4(true) & 25 & 4 & 2.660 & 2919 & 4(false) \\
%% \hline
%% binarysorttree.c & 80 & 5 & 30.210 & 2950 & 10(true) & 19 & 4 & 5.130 & 2935 & 10(false) \\
%% \hline
%% database.c & 179 & 12 & 4.930 & 2922 & 3(true) & 21 & 4 & 2.760 & 2920 & 2(false) \\
%% \hline
%% ihex2fw.c & 202 & 7 & 23.490 & 2882 & 5(false) & 15 & 4 & 2.160 & 2797 & 5(false) \\
%% \hline
%% gen\_init\_cpio.c & 346 & 19 & 9.580 & 2809 & 1(true) & 15 & 4 & 2.160 & 2799 & 1(false) \\
%% \hline
%% \end{tabular}
%% %%\caption{My first table}
%% \end{table}

%% \begin{figure}
%%  \centering
%%  \includegraphics[width=14cm]{statistic.png}
%% \caption{Comparison}
%% \label{fig:statistic}
%% \end{figure}
