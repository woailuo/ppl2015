\section{Preliminary Experiments}
\label{sec:experiment}
In this section, we did the experiments by comparing applying a model
checker to original C language programs with to abstracted
behavior. Table~\ref{tb:mcc} and Table~\ref{tb:mca} show the results
of the experiments. All of the experiments are done on a machine with
an Intel(R) Core(TM) i7-3770 CPU @ 3.40GHz, 8MB cache and 3.76GB memory, running
on Debian (kernel version 2.6.32-5-amd64) and CPAchecker (version 1.3.4).

In these two tables, the meaning of each column is described as
follows. The column \texttt{original programs} and \texttt{abstracted
  behavior}, which mean applying the model checker to a original C
language program and to abstracted behavior respectively. These two
columns consist of several columns: $\sharp$\texttt{loc} means the
number of program locations; $\sharp$\texttt{fun} means the number of
functions in a program; \texttt{cpu time (total)} means the total
execution time of CPU in seconds; \texttt{memory(MB)} means the number
of virtual memory cells consumed by model checker in \texttt{MByte};
\texttt{fixed num} means the number of available memory cells which we
fix; the \texttt{TRUE} in the \texttt{verified result} column means
the number of the memory cells which a program consumes does not
exceed the \texttt{fixed num}, otherwise it is \texttt{FALSE} which
denoates memory-leak.

The programs used for the experiments are described as follows:[TODO]
\begin{itemize}
%% \item \texttt{linklist.c} and \texttt{linklist2.c} initialize a list,
%%   and perform some operations on the list (insert , delete, locate a
%%   node and traverses the list), and destroy the list. The difference
%%   between \texttt{linklist.c} and \texttt{linklist2.c} is that the
%%   former creates list by the head-insertion, whereas the latter creates a list
%%   by the tail-insertion.
%% \item \texttt{linkstack.c} creates a stack, pushes some elements in
%%   the stack, traverses the stack and clear the stack.
%% \item \texttt{linkqueue.c} initializes a queue, inserts, deletes some
%%   elements in the queue, clears the queue and destroys the queue.
%% \item \texttt{binarysorttree.c} inserts some nodes in a binary tree
%%   and deletes all the allocated nodes.
\item \texttt{database.c} models a database. It opens a database and
  performs some operations on the database (retrieve, delete and
  update data, or create a new database) and closes the database. We
  rewrite this program as \texttt{database\_rw.c}.
%% \item \texttt{ihex2fw.c} is a file from Linux kernel
%%   \texttt{/firmware/ihex2fw.c}, which converts ihex files into binary
%%   representation for use by Linux kernel.
%% \item \texttt{gen\_init\_cpio.c} is a file from Linux kernel
%%   \texttt{/usr/gen\_init\_cpio.c}, which produces a binary file which
%%   performs on a file containing newline separated entries that
%%   describe the files to be included in the initramfs archive.
\end{itemize}

The results present in the these two tables show that the
resources required for model checking become smaller if applying model
checkers to the abstracted behavior, since the abstracted behavior
only consists of allocation and deallocation.

One thing we should notice in these two tables is that for the same
programs, the \texttt{verified result} should be the same when model
checking on original programs and abstracted behavior; but it is not,
for example, \texttt{database.c} has the \texttt{TRUE} in the
Table~\ref{tb:mcc} but the \texttt{FALSE} in the
Table~\ref{tb:mca}. The reason is that our approach, abstracted
behavior, may not correctly deal with some conditional statements,
function calls which locate in another files consisting of allocation
or deallocation operations; for example, the \texttt{database.c}
consists of statements like if-then-else statements, which is
abstracted as $\mu\alpha.(\Malloc;\Malloc;((\Free + 0);\Free) +
0);\alpha)$. In fact, the \texttt{database.c} does $\Malloc$ 2 times,
and then $\Free$ 2 times, so it is safe; but the abstracted behavior
may perform like $\mu\alpha.(\Malloc;\Malloc;0;\alpha)$, which means
consuming unbound number of memory cells. Therefore, we rewrite this
program to \texttt{database\_rw.c} which does not change the semantics
of original program, and the behavior of \texttt{database\_rw.c} is
$\mu\alpha.(\Malloc;(\Malloc + \Free) + 0);\Free;\alpha)$ which
returns \texttt{TRUE} when doing model checking on it.

\begin{table}
  \scriptsize
\begin{tabular}{|c|c|c|c|c|c|c|}
\hline
& \multicolumn{6}{|c|}{\texttt{original programs}}  \\
\hline
 & $\sharp$\texttt{loc} & $\sharp$\texttt{fun} & \texttt{cpu time (total,sec)} & \texttt{memory (MB)} & \texttt{fixed num}& \texttt{verified result} \\
\hline
<<<<<<< HEAD
\texttt{person.c} & 51 & 4 & 2.660 & 2807 & 4 & \texttt{TRUE}   \\
\hline
\texttt{linklist.c} & 109 & 10 & 11.830 & 2817 & 6 & \texttt{TRUE}  \\
\hline
\texttt{ihex2fw.c} & 213 & 7 & 4.190 & 2801 & 1 &  \texttt{TRUE} \\
\hline
\texttt{linklst2.c} & 140 & 13 & 25.620 & 2955 & 21 & \texttt{TRUE} \\
\hline
\texttt{linkstack.c}  & 87 & 10 & 10.830 & 2941 & 11 & \texttt{TRUE} \\
\hline
\texttt{linkqueue.c} & 119 & 11 & 13.110 & 2939 & 4 & \texttt{TRUE} \\
\hline
\texttt{binarysorttree.c} & 80 & 5 & 30.210 & 2950 & 10 &\texttt{TRUE}  \\
\hline
\texttt{database.c} & 179 & 12 & 4.930 & 2922 & 3 & \texttt{TRUE} \\
\hline
\texttt{gen\_init\_cpio.c} & 346 & 19 & 9.580 & 2809 & 1 & \texttt{TRUE} \\
=======
\texttt{database.c} & 153 & 10 & 12.010 & 2907 & 2  & \texttt{TRUE}  \\
\hline
\texttt{database\_rw.c} & 151 & 10 & 7.080 & 2907 & 2  & \texttt{TRUE}  \\
>>>>>>> 2ef249661e9f97f4d20d862003ef4075c76063d5
\hline
\end{tabular}
\caption{Model checking on original C language programs}
\label{tb:mcc}
\end{table}

\begin{table}
  \scriptsize
\begin{tabular}{|c|c|c|c|c|c|c|}
\hline
&\multicolumn{6}{|c|}{abstracted behavior} \\
\hline
 &$\sharp$\texttt{loc} & $\sharp$\texttt{fun} & \texttt{cpu time (total,sec)} & \texttt{memory (MB)} & \texttt{fixed num} & \texttt{verified result} \\
\hline
<<<<<<< HEAD
\texttt{person.c} & 22 & 4 & 2.000 & 2738 & 4 & \texttt{TRUE}  \\
\hline
\texttt{linklist.c} &  20 & 4 & 1.990 & 2804 & 6 & \texttt{TRUE} \\
\hline
\texttt{ihex2fw.c}  & 16 & 4 & 2.040 & 2799 & 1 & \texttt{TRUE} \\
\hline
\texttt{linklst2.c} & 20 & 4 & 10.72 & 2945 & 21 & \texttt{FALSE} \\
\hline
\texttt{linkstack.c} & 20 & 4 & 4.990 & 2916 & 11 & \texttt{FALSE} \\
\hline
\texttt{linkqueue.c} & 25 & 4 & 2.660 & 2919 & 4 & \texttt{FALSE} \\
\hline
\texttt{binarysorttree.c} & 19 & 4 & 5.130 & 2935 & 10 & \texttt{FALSE} \\
\hline
\texttt{database.c}  & 21 & 4 & 2.760 & 2920 & 3 & \texttt{FALSE} \\
\hline
\texttt{gen\_init\_cpio.c} & 15 & 4 & 2.160 & 2799 & 1 & \texttt{FALSE} \\
=======
\texttt{database.c} &  16 & 4 & 2.060 & 2800 & 2 & \texttt{FALSE} \\
\hline
\texttt{database\_rw.c} &  18 & 4 & 1.990 & 2737 & 2 & \texttt{TRUE} \\
>>>>>>> 2ef249661e9f97f4d20d862003ef4075c76063d5
\hline
\end{tabular}
\caption{Model checking on abstracted behavior}
\label{tb:mca}
\end{table}

%% \begin{table}
%% \tiny
%% \begin{tabular}{|c|c|c|c|c|c|c|c|c|c|c|}
%% \hline
%% & \multicolumn{5}{|c|}{original programs} & \multicolumn{5}{|c|}{abstracted behavior} \\
%% \hline
%%  & $\sharp$loc & $\sharp$fun & cpu time (total) & memory (MB) & fixed num & $\sharp$loc & $\sharp$fun & cpu time (total) & memory (MB) & fixed num \\
%% \hline
%% linklist.c & 154 & 13 & 9.770 & 2943 & 6(true) & 20 & 4 & 3.190 & 2918 & 6(false) \\
%% \hline
%% linklst2.c & 140 & 13 & 25.620 & 2955 & 21(true) & 20 & 4 & 10.72 & 2945 & 21(false) \\
%% \hline
%% linkstack.c  & 87 & 10 & 10.830 & 2941 & 11(true) & 20 & 4 & 4.990 & 2916 & 11(false) \\
%% \hline
%% linkqueue.c & 119 & 11 & 13.110 & 2939 & 4(true) & 25 & 4 & 2.660 & 2919 & 4(false) \\
%% \hline
%% binarysorttree.c & 80 & 5 & 30.210 & 2950 & 10(true) & 19 & 4 & 5.130 & 2935 & 10(false) \\
%% \hline
%% database.c & 179 & 12 & 4.930 & 2922 & 3(true) & 21 & 4 & 2.760 & 2920 & 2(false) \\
%% \hline
%% ihex2fw.c & 202 & 7 & 23.490 & 2882 & 5(false) & 15 & 4 & 2.160 & 2797 & 5(false) \\
%% \hline
%% gen\_init\_cpio.c & 346 & 19 & 9.580 & 2809 & 1(true) & 15 & 4 & 2.160 & 2799 & 1(false) \\
%% \hline
%% \end{tabular}
%% %%\caption{My first table}
%% \end{table}

%% \begin{figure}
%%  \centering
%%  \includegraphics[width=14cm]{statistic.png}
%% \caption{Comparison}
%% \label{fig:statistic}
%% \end{figure}
