\section{Type reconstruction}
\label{sec:reconstruction}

%% This section describes how to construct syntax directed typing rules
%% according to the typing rules of above section, and it provides an
%% algorithm which inputs statements and returns a pair containing
%% constraints and behavior types.

This section describes a type reconstruction procedure for the type
system in Section~\ref{sec:typesystem}.  Since the procedure is
essentially the same as one in Kobayashi et
al.~\cite{DBLP:journals/lmcs/KobayashiSW06}, we do not give a concrete
definition here.  The procedure described here is not complete.  We
could reject a program that is well-typed in the type system.

The reconstruction procedure is a constraint-based one.  It generates,
given a program, constraints for the program to be well-typed by
constructing a derivation tree based on the rules in
Figure~\ref{fig:typingrules}.  A constraint is either a subtyping
constraint \(\alpha \ge P\), or \(\OK_\nu(\alpha)\), where \(\nu\) is
a symbol for an unknown natural number.  Since \rn{T-Prog} is the only
place where the condition \(\OK_n(P)\) is involved, the constraint set
includes exactly one constraint of the form \(\OK_\nu(\alpha)\).  The
concrete definition of the constraint generation is in the full
version~\cite{fullversion}.

By using the result obtained by Kobayashi et al.~\cite[Lemma
  3.8]{DBLP:journals/lmcs/KobayashiSW06}, a subtyping constraint
\(\alpha \ge P\) can be resolved by setting \(\alpha = \mu
\alpha. P\), which is the least solution of the constraint.  Hence,
the generated constraint set is reduced to a single constraint
\(\OK_{\nu}(P)\).

By definition, \(\OK_{\nu}(P)\) holds if there is a natural number
\(n\) such that, for all \(\sigma\) and \(P'\), \(P
\xlongrightarrow{\sigma} P'\) implies \(\sharp_{\Malloc}(\sigma) -
\sharp_{\Free}(\sigma) \le n\).  In order to check this condition
soundly, we fix the upper bound of \(\nu\) to be checked.  Then,
\(\OK_{\nu}(P)\) can be checked by model-checking a system with
finitely many states; hence, model checkers like CPAChecker~\cite{}
and SPIN~\cite{} are applicable.  More detailed description on how we
apply a model-checking algorithm to check \(\OK_\nu(P)\) is in the
forthcoming version.

%% \subsection{Constraint generation}

%% %% By syntax directed typing rules, the type reconstruction algorithm has
%% %% been designed as in Figure~\ref{fig:tyin}.

%% %% Function $PT_{v}(x) = (C,\emptyset)$ denotes that it receives a
%% %% pointer variable $x$ and outputs a pair consisting of
%% %% constraints set $C$ and an empty set. $PT_{\Theta}(s) = (C, P)$
%% %% is a mapping from statements to a pair -- constraints set $C$
%% %% and behavioral types $P$, where $\Theta$ is mapping from
%% %% function names to function types. $PT(\langle D,s \rangle) = (C,
%% %% P)$ denotes that it receives a program and produces a pair $(C,
%% %% P)$. $\alpha_{i}$ and $\beta$ are fresh type variables.



%% \subsection{Constraints reduction} $PT\langle D, s \rangle$
%% receives a program as argument and produces a pair which consists
%% of the subtype constraints on behavior types of the form $\alpha
%% \ge A$, and constraints of the form $OK_{n}(P)$. Thus, we obtain
%% the following constraints:\\ $$ \{ \alpha_{1} \ge A_{1}, \dots,
%% \alpha_{n} \ge A_{n}, OK_{n}(P)\} $$ Here, we can assume that
%% $\alpha_{1}, \dots, \alpha_{n}$ are pairwise-distinct, since
%% $\alpha \ge A_{1}$ and $\alpha \ge A_{2}$ can be replaced with
%% $\alpha \ge A_{1}+A_{2}$ by lemma 3.8 in
%% paper~\cite{DBLP:journals/lmcs/KobayashiSW06}. we can also assume
%% that $\{ \alpha_{1}, \dots, \alpha_{n} \}$ contains all the type
%% variables in the constraints, since otherwise we can always add the
%% tautology $\alpha \ge \alpha$. Each subtype constraints $\alpha \ge
%% A$ can be replaced by $\alpha \ge \mu \alpha. A$, by lemma
%% 3.8(4)~\cite{DBLP:journals/lmcs/KobayashiSW06} ( substituting
%% $\alpha$ for $B$ in this lemma). Therefore the above constraints
%% can be further reduced to $OK_{n}([\vec{A} \slash
%% \vec{\alpha}]P)$. Here, $A'_{1}, \dots, A'_{n}$ are the least
%% solutions for the subtype constraints.

%% \section{Preliminary Experiment}
