%\documentclass[japanese]{jssst_ppl} %% $BF|K\8l(B (default)
 \documentclass[english]{jssst_ppl} %% English
% \documentclass[japanese,draft]{jssst_ppl} %% You can use the draft option
\usepackage{amsmath,amsthm,amssymb,amsfonts,extarrows,geometry,amsopn}

% set some new commands %
\newcommand\tB{\;|\;}
\newcommand\LET{\mathbf{let}\;}
\newcommand\FREE{\mathbf{free(x)}\;}
\newcommand\IN{\mathbf{in}\;}
\newcommand\SKIP{\mathbf{skip}}
\newcommand\Rtab{\; \; \; \;}
\newcommand\NULL{\mathbf{null}\;}
\newcommand\IFNULL{\mathbf{ifnull}\;}
\newcommand\THEN{\mathbf{then}\;}
\newcommand\ELSE{\mathbf{else}\;}
\newcommand\Lcc{\left(}
\newcommand\Rcc{\right)}
\newcommand\Lfc{\left\{}
\newcommand\Rfc{\right\}}
\newcommand\Lb{\left[}
\newcommand\Rb{\right]}
\newcommand\coma{,\;}
\newcommand\MALLOC{\mathbf{malloc()}\;}
\newcommand\Malloc{\mathbf{malloc}}
\newcommand\Free{\mathbf{free}}
\newcommand\Cirx{(x)}
\newcommand\dtb{\;\;\ \;\;\ \;\;\ \;\;\  }
\newtheorem{theorem}{Theorem}[section]
\newtheorem{lemma}[theorem]{Lemma}
\newtheorem{proposition}[theorem]{Proposition}
\newtheorem{corollary}[theorem]{Corollary}
\newtheorem{myDef}{Definition}


\title{Safe Memory Deallocation for Non-Terminating Programs}
\author{Author Name}
\inst{%
$^1$ \\
\texttt{sample@example.ac.jp}
\medskip\par%
$^2$ \\
\texttt{example@sample.net}
}
\begin{document}
\maketitle
\begin{abstract}
We propose a type system to prevent programs infinitely consuming memory cells. The main idea is based on the previous type system that augments pointer types with fractional ownerships~\cite{DBLP:conf/aplas/SuenagaK09} and behavioral type system that abstracts the behavior of programs. Thanks to the previous type system, we can focus on the behavioral types to count the upper bound of the consumed memory cells soundly.
\end{abstract}

\section{Introduction}
Manual memory management primitives (e.g., {\tt malloc} and {\tt free} in C) often cause serious problems such as double frees, memory leaks, and illegal read/write to a deallocated memory cell. Verifying \emph{safe memory deallocation} -- a program not leading to such an unsafe state -- is an important problem.

Most of safe memory deallocation verification techniques proposed so far \cite{DBLP:conf/aplas/SuenagaK09,DBLP:conf/pldi/HeineL03,DBLP:conf/sigsoft/XieA05,DBLP:journals/scp/SwamyHMGJ06} focus on the memory leak for \emph{terminating} programs: If a program terminates, the program satisfies safe memory dealloction. For example, the type system by Suenaga and Kobayashi \cite{DBLP:conf/aplas/SuenagaK09} guarrantees that (1) a well-typed program does not perform read/write/free operations to any deallocated memory cell and that (2) after execution of a well-typed program, all the memory cells are deallocated.

Currently, we are investigating safe memory deallocation for non-terminating programs, because the safe memory deallocation is very important in real world programs such as Web servers and operating systems.

The main idea of our approach is to decompose this problem into two subproblems: (1) partial correctness and (2) \emph{behavioral correctness}. The former is verified based on the previous type system \cite{DBLP:conf/aplas/SuenagaK09}, whereas the latter based on the behavioral type system that is mainly used to abstact the behavior of a program. Behavioral types are heavily used in the contex of concurrent program verification \cite{DBLP:journals/lmcs/KobayashiSW06,DBLP:journals/tcs/IgarashiK04,DBLP:conf/esop/HondaVK98}. I will describe the concept "memory leak" and  our idea by several examples written in an ML-like language as shown in Figure 1 and Figure 2:

\begin{figure}[h]
1 \;\; $f(x)$= \dtb\dtb \dtb \;\;\;\; \;\;$g(x)$= \\
2 \;\; $\LET \; x = \MALLOC  \; \IN$ \;\dtb \;\;\;\;$\LET \; x = \MALLOC  \; \IN$\\
3 \;\; $\Free\Cirx$; \; $f(x)$\dtb\dtb\;\;\; \;\;$g(x)$; \; $\Free\Cirx$
\caption{Examples of memory leak.}
\label{example:memoryLeak}
\end{figure}

A program \emph{leaks} memory if the program consumes unbounded number of memory cells. For example, the left-hand side program in Example~\ref{example:memoryLeak} does not leak memory, whereas the right-hand side does; the former consumes at most one memory cell at once but the latter consumes unbounded number of memory cells. Notice that these two programs are all partially corrected by the previous type system, because they do not terminate. Let us consider another examples as follows:
\begin{figure}[h]
1 \;\; $h(x)$= \dtb\dtb\dtb\dtb $h'(x)$=\\
2 \;\; $\LET \; x = \MALLOC  \; \IN$\dtb $\LET \; x = \MALLOC \; \IN$ \\
3 \;\; $\LET \; y = \MALLOC  \; \IN$ \dtb $\LET \; y = \MALLOC \; \IN$\\
4 \;\; $\Free\Cirx$; \; $\Free(y) $; \; $h(x)$;  \dtb\dtb\dtb$\Free(x)$; $h'(x)$; $\Free(y)$
\caption{Example for demonstrating the main observation.}
\label{example:observation}
\end{figure}
The same to examples in Figure 1, the left-hand side program does not leak memory but the right-hand side does; these two are all partially corrected by previous type system. Another thing about programs in Figure 2 we should notice is that once partial correctness is guaranteed, we can guarantee memory-leak freedom by estimating upper bound of memory consumption \emph{ignoring the relation between variables and pointers to memory cells}. Specially speaking, partial correctness has proved that there is no double free in programs, which means if allocating a memory cell in the program, definitely there is a deallocation to that pointer. Say, in order to verify $h$ consumes at most two cells at once, we can ignore $x$ and $y$ in $h$; We can focus on the fact that $h$ executes $\Malloc$ twice, $\Free$ twice, and then calls $h$. This abstraction is sound because the correspondence between allocations and deallocations is guaranteed by the partial correctness verification. And from the abstraction of $h$, we know the number of $\Malloc$ and $\Free$ is balanced, so we can say that function $h$ is memory-leak freedom. About the function $h'$, its behavior is abstracted as $\Malloc$ twice,$\Free$ once, calls $h'$ itself ,and then $\Free$ once. So the number of $\Malloc$ exceeds the number of $\Free$ once, which makes the recursive function $h'$ consume infinitely memory cells.

Thanks to the previous type system, which guarantes partial correctness, We can focus on the abstraction of behaviour by \emph{behavioral type system}. In our paper, the behaviour of a program is abstracted by CCS-like processes. For example, the behaviour of $f$ is abstracted as $\mu \alpha. \Malloc;\Free;\alpha$; the behaviour of $g$ is abstracted as $\mu \alpha. \Malloc;\alpha;\Free$; the behaviour of $h$ is abstracted as $\mu \alpha. \Malloc;\Malloc;\Free;\Free;\alpha$; the behaviour of $h'$ is abstracted as $\mu \alpha. \Malloc;\Malloc;\Free;\alpha;\Free$.

The rest of this paper is structed as follows. Section 2 introduces a simple imperative language, as well as its syntax and operational semantics. Section 3 introduces the behavioral type system and its semantics.

\section{Language}
This section introduces a sublanguage of Suenaga and Kobayashi~\cite{DBLP:conf/aplas/SuenagaK09} with primitives for memory allocation/deallocation. And the values in our paper are only pointers. \\
The syntax of language is as follows.
\subsection{Syntax}
\begin{eqnarray*}
  s \ (statements)& &::=  \SKIP \tB s_{1};s_{2} \tB *x \leftarrow y \tB \Free \Cirx \\
  & & \tB \LET x = \MALLOC \IN s \tB \LET x = \NULL \IN s  \\
  & & \tB \LET x = y \; \IN s \tB   \LET x = *y \; \IN s \\
  & &\tB \IFNULL(x) \; \THEN s_{1}\; \ELSE s_{2} \tB f(\vec{x})\\
  d \ (definition)& &::= f(x_{1},\dots,x_{n}) = s
\end{eqnarray*}
A program is a pair $(D,s)$, where $D$ is the set of definition.\\
The command $\SKIP$ does nothing. The command $s_{1};s_{2}$ is executed as a sequence, first executing $s_{1}$ and then $s_{2}$. The command $*x \leftarrow y$ update the content of the memory cell which is pointed by pointer $x$ with value $y$. The command $\Free \Cirx$ deallocates the memory cell which is pointed by pointer $x$. Then command $\LET x = e \IN s$ first evaluates the expresstion $e$ and binds the return value of $e$ to $x$ and then executes statement $s$. The command $\LET x = \Malloc \IN s$ first allocates a memory cell to a pointer $x$ and then executes the statement $s$. The command $\LET x = \NULL \IN s$ first allocates a null pointer to $x$ and then executes $s$. The command $\LET x = y \IN s$ assign the pointer $y$ to $x$, so the pointer $x$ and $y$ are said aliases for the same memory cell, and then executes statement $s$. The command $\LET x = *y \IN s$ transfers a part of memory cells pointed by $y$ and then executes statement $s$. The command $\IFNULL(x) \; \THEN s_{1} \; \ELSE s_{2}$ denotes that  executing statement $s_{1}$ if pointer $x$ is a null pointer, if not, executing statement $s_{2}$. The command $f(\vec{x})$ is a function call in which $\vec{x}$ denotes mutually distinct variables like \{$x_{1}, \dots, x_{n}$\}. The notation $d$ denotes the definition of function $f(\vec{x})$ which has a body of statement $s$. And examples are described by this syntax you can see in Figure 1 and  Figure 2.

\subsection{Operational Semantics}
Because we want to estimate the number of available memory cells at every operation step, we extend the triple $\langle H\coma R\coma s \rangle$ that is represented as run-time state in previous type system to a quadruple $\langle H\coma R\coma s\coma n \rangle$ in our paper. The introduced notation $n$ denotes the number of available memory cells. When executing the operation $\Malloc$, the number of available memory cells will decrease 1, which is denoted as ($n - 1$); when executing the opreation $\Free$, the number of available memory cells will increase 1, which is denoted as ($n + 1$). The notation $H$, which models heap memory, is a mapping from finite subset of $\mathcal{H}$ to $\mathcal{H}$ $\cup$ \{$null$\}, where $\mathcal{H}$ represents the set of \emph{heap addresses}. $R$, which models registers, is a mapping from finite set of variables to $\mathcal{H}$ $\cup$ \{$null$\}.

Transition rules are listed in Figure 3. In these rules, $f\{x\to v\}$ is defined as a function $f'(y)$ such that $f'(y) = v$ if $x = y$, otherwise $f'(y) = f(y)$. $n$ represents the number of available memory cells, so it is a nature number.
%\begin{figure}[center]
% Skip Command
$$
    \frac{n \in \mathbb{N}}
            {\langle H\coma R\coma  \SKIP;s , n \rangle
              \longrightarrow_{D}
                \langle H\coma R\coma   s , n \rangle }
     \Rtab \mbox{(E-Skip)}
$$
% Assignment
$$
     \frac{R(x) \in dom(H), n \in \mathbb{N}}
           {\langle H\coma R\coma  *x \leftarrow y , n \rangle
             \longrightarrow_{D}
             \langle H \Lfc R(x) \rightarrow R(y) \Rfc \coma R \coma   \SKIP , n  \rangle }
     \Rtab \mbox{(E-Assign)}
$$
% Free Command
$$
     \frac{R(x) \in dom(H) , n \in \mathbb{N}}
          {\langle H\coma R\coma  \FREE , n \rangle
            \xlongrightarrow{\Free}_{D}
            \langle H\backslash \Lfc R(x) \Rfc \coma R \coma   \SKIP , n+1  \rangle }
     \Rtab \mbox{(E-Free)}
$$
% Let Null Command
$$
     \frac{x' \notin dom(R)}
           {\langle H\coma R\coma  \LET x = \NULL \IN s , n \rangle
             \longrightarrow_{D}
             \langle H\coma R\Lfc x' \rightarrow \NULL \Rfc \coma   \Lb x'/x \Rb s , n  \rangle }
     \Rtab \mbox{(E-LetNull)}
$$
% Let Eq Command
$$
     \frac{x' \notin dom(R)}
            {\langle H\coma R\coma \LET x = y \; \IN s , n \rangle
              \longrightarrow_{D}
              \langle H\coma R\Lfc x' \rightarrow R(y) \Rfc \coma   \Lb x'/x \Rb s , n  \rangle }
\Rtab \mbox{(E-LetEq)}
$$
% Reference Command
$$
     \frac{x' \notin dom(R)}
            {\langle H\coma R\coma  \LET x = *y \; \IN s , n \rangle
              \longrightarrow_{D}
              \langle H\coma R\Lfc x' \rightarrow H(R(y)) \Rfc \coma   \Lb x'/x \Rb s , n  \rangle }
     \Rtab \mbox{(E-LetDref)}
$$
% Malloc (allocate) Command
$$
     \frac{h \notin dom(H)}
            {\langle H\coma R\coma  \LET x = \Malloc() \; \IN s , n \rangle
              \xlongrightarrow{\Malloc}_{D}
              \langle H \Lfc h \rightarrow v\Rfc \coma R\Lfc x' \rightarrow h \Rfc \coma   \Lb x'/x \Rb s , n-1  \rangle }
\Rtab \mbox{(E-Malloc)}
$$
% IFNULL T
$$
    \frac{R(x) = \NULL}
           {\langle H \coma R \coma \IFNULL\Cirx   \THEN   s_{1} \ELSE\  s_{2} \coma  n \rangle
           \longrightarrow_{D}
           \langle H\coma R\coma s_{1} \coma n \rangle}
    \Rtab \mbox{(E-IfNullT)}
$$
% IFNULL F
$$
    \frac{R(x) \neq \NULL}
           {\langle H \coma R \coma \IFNULL\Cirx \THEN  s_{1} \ELSE  s_{2} \coma  n \rangle
           \longrightarrow_{D}
           \langle H\coma R\coma s_{2} \coma  n, \rangle}
    \Rtab \mbox{(E-IfNullF)}
$$
% Function Call
$$
     \frac{f(\vec{y}) = s \in D}
            { \langle H\coma R\coma  f(\vec{x}) , n \rangle
               \longrightarrow_{D}
               \langle H\coma R\coma  \Lb \vec{x}/\vec{y} \Rb s , n \rangle}
      \Rtab \mbox{(E-Call)}
$$
% Error : access the null memory cell
$$
      \frac{R(x) = null}
            {\langle H\coma R\coma  *x \leftarrow y , n \rangle
              \longrightarrow_{D}
              NullEx }
      \Rtab \mbox{(E-AssignNullError)}
$$
% ERROR : access the null memory cell
$$
      \frac{R(y) = null}
             {\langle H\coma R\coma  x = *y, n \rangle
               \longrightarrow_{D}
               NullEx }
             \Rtab \mbox{(E-DrefNullError)}
$$
% ERROR :
$$
     \frac{R(x) \notin dom(H) \cup \Lfc null \Rfc}
           {\langle H\coma R\coma   *x \leftarrow y,  n \rangle
             \longrightarrow_{D}
             Error }
    \Rtab \mbox{(E-AssignError)}
$$
% ERROR
$$
      \frac{R(y) \notin dom(H) \cup \Lfc null \Rfc}
           {\langle H\coma R\coma  \LET x  = *y \; \IN s, n \rangle
              \longrightarrow_{D}
                  Error }
      \Rtab \mbox{(E-DrefError)}
$$
%
$$
      \frac{R(x) \notin dom(H) \cup \Lfc null \Rfc}
            {\langle H\coma R\coma  \FREE , n \rangle
              \xlongrightarrow{\Free}_{D}
              Error }
     \Rtab \mbox{(E-FreeError)}
$$
%
$$
     \frac{R(x) =  null }
           {\langle H\coma R\coma  \FREE , n \rangle
             \xlongrightarrow{\Free}_{D} NullEx  }
      \Rtab \mbox{(E-FreeNullError)}
$$
% ERROR: no enough space
$$
      \langle H\coma R\coma \LET x = \Malloc() \ \IN s ,  0  \rangle
      \xlongrightarrow{\Malloc}_{D}
      Error
      \Rtab \mbox{(E-MallocError)}
$$
%\caption{Operational Semantics.}
%\label{example:os}
%\end{figure}

\section{Type System}
This section elaborate the behavioral type system to prevent leaking memory in non-terminating programs. We define behavioral types, CCS-like processes that abstract the behavior of programs, as follows.
\subsection{Syntax of Types}
     \begin{eqnarray*}
       P (behaviral \ types)::=&& {\bf 0} \tB P_{1};P_{2} \tB P_{1}+P_{2} \tB \Malloc\\
       &&\tB \Free \tB \alpha \tB \mu\alpha.P \\
       \tau (value \ types)::=&&    \mathbf{Ref}  \\ %
       \sigma (function \ types)::=&& (\tau_{1},\dots, \tau_{n}) P
     \end{eqnarray*}

The type $\bf 0$ abstracts the behavior of $\SKIP$ and means "does nothing". $P_{1};P_{2}$ is for sequential exetution. $P_{1} + P_{2}$ is abstracted as contional. $\Malloc$ is the behavior of a program that allocates a memory cell exactly once. $\Free$ is for deallocating memory cell exactly once. $\mu \alpha. P$ is a recursive type. For example, the behavior of  the body of function $h$ in Figure 2 is abstracted as $\mu \alpha. \Malloc;\Malloc;\Free;\Free;\alpha$. $\alpha$ is a type variable and bounded to the recursive constructor $\mu \alpha$.

The only value in our paper is reference, and its type is $\mathbf{Ref}$.

The function type is described as $(\tau_{1}, \dots, \tau_{n})P$, which means a function receives some pointers as arguments and then executes its body abstracted as behavioral type $P$.

% Semantics of Behavioral Types %
\subsection{Semantics of Behavioral Types}
The semantics of behavioral type are given by labeled transition system, and listed as follows:
    $$
         0;P \rightarrow P
    $$
    $$
          \Malloc \xlongrightarrow{\Malloc} 0
    $$
    $$
           \Free \xlongrightarrow{\Free} 0
    $$
    $$
          \mu \alpha.P \rightarrow  [\mu \alpha . P/\alpha]  P
    $$
   $$
          P_{1} + P_{2} \longrightarrow P_{1}
   $$
   $$
          P_{1} + P_{2} \longrightarrow P_{2}
   $$
   $$
           \frac{P_{1} \xlongrightarrow{\alpha} P_{1}' }
                 {P_{1};P_{2} \xlongrightarrow{\alpha} P_{1}';P_{2}}
   $$
The notation $\rightarrow$ denotes that a behavioral type can be reduced by the internal action. Notation $\xlongrightarrow{\alpha}$ means that a behavioral type can be reduced by executing $\alpha$ actions, and the $\alpha$ here is $\{\Malloc, \Free\}$.

% Type Judgments
\subsection{Typing Rules}
The type judgement of our type system is given by the form $\Theta ; \Gamma \vdash s : P$, where $\Theta$ is a mapping from function varibles to function types, $\Gamma$ is a type environment that denotes a mapping from variables to value types.
It reads ``the behavior of $s$ is abstracted as $P$ under $\Theta$ and $\Gamma$ environments''. We design the type system so that this type judgement implies the property: when $s$ executes $\Malloc$(resp.$\Free$), then $P$ is equivalent to $\Malloc;P'$(resp.$\Free;P'$) for a type $P'$ such that $\Theta; \Gamma \vdash s': P'$, where $s'$ is the continuation of $s$. This property guarantees the behavioral type soundly abstracts the upper bound of the consumed memory cells.

Typing rules are presented in Figure. In the rule for assignment, the behavior of  $*x \leftarrow y$ is $\bf 0$. The rule for $\Free$ represents that the behavior of $\Free \Cirx$ is $\Free$. The rule T-Malloc represents that $\LET x = \MALLOC \; \IN s$ has the behavior $\Malloc;P$, where $P$ is the behavior of statement $s$.

    $P_{1}$ is a subtype of $P_{2}$, which is written as $P_{1} \le P_{2}$, means that: \\
   (1) if $P_{1} \xlongrightarrow{\alpha}  P_{1}'$ then $\exists P_{2}' $ s.t. $P_{2} \overset{\text{$\alpha$}}{\Longrightarrow} P_{2}'$ and $ P_{1}' \le P_{2}' $ \\
   (2) if $P_{1} \rightarrow P_{1}'$ then $\exists P_{2}'$ s.t. $P_{2} \xlongrightarrow{\tau}^{*} P_{2}'$ and  $P_{1}' \le P_{2}'$\\
    where $\overset{\text{$\alpha$}}{\Longrightarrow}$ means that: $\rightarrow^{*} \xlongrightarrow{\alpha} \rightarrow^{*}$ and $\tau$ means internal action.
$$
       \frac{\Gamma \vdash s : P \Rtab OK_{n}(P)}{\vdash \langle H,R,s,n
       \rangle .}
$$
  where $dom(H) \cup dom(R) = dom(\Gamma).$

\begin{myDef}
 $OK_{n}(P) \iff \forall P',\; P \xlongrightarrow{\alpha}^{*}P'$ then $\sharp_{malloc}(\alpha)-\sharp_{free}(\alpha)\le n$.
\end{myDef}

% Skip type
$$
         \Theta ; \Gamma \vdash \SKIP : \mathbf{0}
      \Rtab \mbox{(T-Skip)}
$$
% Sequence type
$$
      \frac{\Theta ; \Gamma \vdash s_{1} : P_{1} \Rtab \Theta ; \Gamma \vdash s_{2} : P_{2}}
          {\Theta ; \Gamma \vdash s_{1} ; s_{2} : P_{1};P_{2} }
     \Rtab \mbox{(T-Seq)}
$$
% Assignment type
$$
     \frac{\Theta ; \Gamma \vdash y :  \mathbf{Ref} \Rtab \Theta ; \Gamma \vdash x : \mathbf{Ref} }
          {\Theta ; \Gamma \vdash *x \leftarrow y : \mathbf{0} }
     \Rtab \mbox{(T-Assign)}
$$
% Free(deallocate) type
$$
     \frac{\Theta ; \Gamma \vdash x : \mathbf{Ref} }
           {\Theta ; \Gamma \vdash \Free(x) : \Free}
     \Rtab \mbox{(T-Free)}
$$
% Malloc type
$$
     \frac{\Theta ; \Gamma,x : \mathbf{Ref} \vdash s : P}
           {\Theta ; \Gamma \vdash \LET x = \MALLOC \; \IN s  : \Malloc;P}
           \Rtab \mbox{(T-Malloc)}
$$
% Let eq type
$$
     \frac{\Theta ; \Gamma \vdash y : \mathbf{Ref}  \Rtab \Theta ; \Gamma , x : \mathbf{Ref} \vdash s : P}
           {\Theta ; \Gamma \vdash \LET x = y \; \IN s : P}
     \Rtab \mbox{(T-LetEq)}
$$
% Dereference type
$$
     \frac{\Theta ; \Gamma \vdash y : \mathbf{Ref}  \Rtab \Theta ; \Gamma , x : \mathbf{Ref} \vdash s : P}
           {\Theta ; \Gamma \vdash \LET x = *y \; \IN s : P}
     \Rtab \mbox{(T-LetDref)}
$$
% Let NULL type
$$
     \frac{\Theta ; \Gamma, x : \mathbf{Ref} \vdash s : P}
           {\Theta ; \Gamma \vdash \LET x = \mathbf{null} \; \IN s : P}
     \Rtab \mbox{(T-LetNull)}
$$
% Subtyping
$$
     \frac{\Theta ; \Gamma \vdash s : P_{1} \Rtab P_{1} \le P_{2}}
            {\Theta ; \Gamma \vdash s : P_{2}}
     \Rtab \mbox{(T-Sub)}
$$
 % ifnull s then s type
$$
     \frac{\Theta ; \Gamma \vdash x : \mathbf{Ref}   \ \ \ \  \Theta ; \Gamma \vdash s_{1} : P \ \ \ \ \Theta ; \Gamma \vdash s_{1} : P}
           {\Theta ; \Gamma \vdash \IFNULL(x) \; \THEN s_{1}\; \ELSE s_{2} : P}
     \Rtab \mbox{(T-IfNull)}
$$
% Function call type
$$ \frac{ \Theta(f) = P}
{\Theta; \Gamma, \vec{x} : \vec{\tau} \vdash f(\vec{x}) : P}
\Rtab \mbox{(T-Call)} $$
% Program
$$\frac{\vdash D : \Theta \;\;\;\; \Theta; \emptyset\vdash s : P \Rtab OK_{n}(P)}
{\vdash (D, s)}
\Rtab \mbox{(T-Program)} $$

\section{Type Soundness}
\begin{theorem}\label{thm1}
If $\vdash (D, s)$ then $(D, s)$ does not lead to $memory\;leak$.\\
Memory leak freedom: $\exists n \in \mathbb{N}$ s.t.
$\langle \emptyset, \emptyset, s, n \rangle \nrightarrow^{*}Error$
\end{theorem}
\noindent

% Lemma Preservation %
\begin{lemma}[Preservation $\mathbf{I}$]%\label{preser}
If $OK_{n}(P)$, $\Theta; \Gamma \vdash s : P$ and $\langle H\coma R\coma s\coma n \rangle
\xlongrightarrow{\alpha}\langle H'\coma R'\coma s'\coma n'
\rangle$, then $\exists P'$ s.t. \\
(1) $ \Theta; \Gamma \vdash s' : P' $ \\
(2) $ P \overset{\text{$\alpha$}}{\Longrightarrow} P'$\\
(3) $ OK_{n'}(P') $
\end{lemma}
\begin{lemma}[Preservation $\mathbf{II}$]%\label{preser}
If $OK_{n}(P)$, $\Theta ; \Gamma \vdash s : P$ and $\langle H\coma R\coma s\coma n \rangle
\rightarrow \langle H'\coma R'\coma s'\coma n'
\rangle$, then $\exists P'$ s.t. \\
(1) $\Theta; \Gamma \vdash s' : P'$\\
(2) $ P \xlongrightarrow{\tau}^{*} P'  $\\
(3) $OK_{n'}(P')$
\end{lemma}

\begin{lemma}%\label{error}
$when \  partial \  correctness \  is \  guaranteed \  \vdash \langle H,R,s \rangle and \  if \ \vdash \langle H,R,s,n \rangle, \  then \  \vdash \langle
H',R',s',n' \rangle \nrightarrow Error$
\end{lemma}

\section{Syntax Directed Typing Rules}

$C$ is contraint for subtype. \\

$$
     \frac{ C = \emptyset}
           {\Theta; \Gamma; C \vdash \SKIP : \mathbf{0}
             }
      \Rtab \mbox{(ST-Skip)}
$$

$$
      \frac{\Theta;\Gamma ; C_{1} \vdash s_{1} : P_{1} \Rtab \Theta; \Gamma ; C_{2} \vdash s_{2} : P_{2} \Rtab C = C_{1}\cup C_{2} \cup \{ P_{1};P_{2} \le P\}}
      {\Theta;\Gamma; C \vdash s_{1};s_{2} : P}
      \Rtab \mbox{(ST-Seq)}
$$

$$
      \frac{\Theta;\Gamma;C_{1} \vdash y \Rtab \Theta;\Gamma; C_{2} \vdash x : \mathbf{Ref} \Rtab C = C_{1}\cup C_{2}}
      {\Theta;\Gamma; C \vdash *x \leftarrow y : \mathbf{0}}
      \Rtab \mbox{(ST-Assign)}
$$

$$
      \frac{C = \emptyset}
      {\Gamma ; C \vdash \Free() : \Free;\mathbf{0}}
     \Rtab \mbox{(ST-Free)}
$$

$$
     \frac{\Theta;\Gamma, x ; C_{1} \vdash s : P_{1} \Rtab C = C_{1} \cup\{P_{1}\le P\}}
     {\Theta;\Gamma; C \vdash \LET x = \Malloc() \; \IN s : \Malloc ; P}
     \Rtab \mbox{(ST-Malloc)}
$$

$$
     \frac{\Theta;\Gamma; C_{1} \vdash y \Rtab \Theta;\Gamma, x ; C_{2} \vdash s : P_{1} \Rtab C = C_{1}\cup C_{2} \cup \{P_{1} \le P \}}
     {\Theta;\Gamma ; C \vdash \LET x = y \;  \IN s : P}
     \Rtab \mbox{(ST-LetEq)}
$$

$$
     \frac{\Theta;\Gamma ; C_{1} \vdash y: \mathbf{Ref} \Rtab \Theta;\Gamma, x ; C_{2} \vdash s : P_{1} \Rtab C = C_{1}\cup C_{2}\cup\{P_{1} \le P\}}
     {\Theta;\Gamma ; C \vdash \LET x = *y \; \IN s : P}
     \Rtab \mbox{(ST-LetDref)}
$$


$$
     \frac{\Theta;\Gamma; C_{1} \vdash x \Rtab \Theta;\Gamma; C_{2} \vdash s_{1} : P_{1} \Rtab \Theta;\Gamma; C_{3} \vdash s_{2} : P_{2}  \Rtab  C = C_{1} \cup C_{2} \cup C_{3} \{P_{1}\le P, P_{2}\le P \}}
     {\Theta;\Gamma; C \vdash \IFNULL\Cirx \THEN s_{1} \ELSE s_{2} : P }
     \Rtab \mbox{(ST-IfNull)}
$$

$$
     \frac{\Theta(f) = P_{1} \Rtab C = P_{1} \le P}
     {\Gamma,\vec{x}:\vec{\tau} \vdash f(\vec{x}) : P }
     \Rtab \mbox{(ST-Call)}
$$

$$
     \frac{\Theta \vdash D : \Theta \Rtab \Theta ; \emptyset ; C_{1} \vdash s : P \Rtab C = C_{1}\cup\{OK_{n}(P)\}}
     {C \vdash (D , s) }
     \Rtab \mbox{(ST-Prog)}
$$

\section{Type Inference}
$PT_{v}(x) = (\emptyset,\emptyset)$, where $x$ maybe a value or reference.\\
$PT_{\Theta}$ is a mapping from statements to a pair of constraints and types with function $\Theta$ from function names to function types, like $\Theta(f) = P$.
\begin{flalign*}
   PT_{\Theta}(f) &  =  &\\
  \LET & \alpha = \Theta(f) & \\
  \IN  & (C = \{\alpha \le \beta \}, \beta) &
\end{flalign*}

\begin{flalign*}
   PT_{\Theta}(\SKIP) &  =  (\emptyset, 0)&
\end{flalign*}

\begin{flalign*}
   PT_{\Theta}(s_{1};s_{2}) &  =  &\\
   \LET & (C_{1}, P_{1}) = PT_{\Theta}(s_{1}) & \\
        & (C_{2}, P_{2}) = PT_{\Theta}(s_{2}) & \\
   \IN   &(C_{1} \cup C_{2}\cup \{P_{1}; P_{2} \le \beta \}, \beta) &
\end{flalign*}

\begin{flalign*}
   PT_{\Theta}(*x \leftarrow y) &  =  &\\
   \LET & (C_{1}, \emptyset) = PT_{v}(*x) & \\
        & (C_{2}, \emptyset) = PT_{v}(y) & \\
  \IN &(C_{1} \cup C_{2},  0) &
\end{flalign*}

\begin{flalign*}
   PT_{\Theta}(\Free(x)) &  = (\emptyset, \Free; 0)  &
\end{flalign*}

\begin{flalign*}
   PT_{\Theta}(\LET x = \Malloc() \  \IN s) &  =  &\\
   \LET  (C_{1}, P_{1})& = PT_{v}(s) & \\
   \IN (C_{1} \cup \{P_{1}& \le \beta \} ,  \Malloc; \beta) &
\end{flalign*}

\begin{flalign*}
   PT_{\Theta}(\LET x = y \  \IN s ) &  =  &\\
   \LET & (C_{1}, \emptyset) = PT_{v}(y) & \\
  & (C_{2}, P_{1}) = PT_{\Theta}(s) & \\
  \IN &(C_{1} \cup C_{2}\cup \{P_{1} \le \beta \},  \beta) &
\end{flalign*}

\begin{flalign*}
   PT_{\Theta}(\LET x = *y \  \IN s ) &  =  &\\
   \LET & (C_{1}, \emptyset) = PT_{v}(y) & \\
        & (C_{2}, P_{1}) = PT_{\Theta}(s) & \\
  \IN &(C_{1} \cup C_{2}\cup \{P_{1} \le \beta \},  \beta) &
\end{flalign*}

\begin{flalign*}
   PT_{\Theta}(\IFNULL(x) \  \THEN  s_{1} \  \ELSE \ s_{2} ) &  =  &\\
   \LET & (C_{1}, P_{1}) = PT_{\Theta}(s_{1}) & \\
        & (C_{2}, P_{2}) = PT_{\Theta}(s_{2}) & \\
        & (C_{3}, \emptyset) = PT_{v}(x) & \\
  \IN &(C_{1} \cup C_{2}\cup C_{3}\cup \{P_{1} \le \beta, P_{2} \le \beta \},  \beta) &
\end{flalign*}

\begin{flalign*}
   PT(<D, s>) &  =  &\\
   \LET & \Theta = \{ f_{1}:\alpha_{1}, \dots, f_{n}:\alpha_{n}  \} &\\
        & where \ \{ f_{1},\dots, f_{n} \} = dom(D) \ and \ \alpha_{1}, \dots, \alpha_{n} \  are \ fresh  & \\
  \IN   & \LET  (C_{i}, P_{i}) = PT_{\Theta}(D(f_{i})) \  for \  each \ i & \\
  \IN   & \LET  C_{i}^{'} = \{ \alpha_{i} \le P_{i} \} \ for \  each \ i & \\
  \IN   & \LET  (C, P) = PT_{\Theta}(s)  & \\
  \IN  &(C_{i} \cup C_{i}^{'} ) \cup C \cup  \{OK(P)\},  P) &
\end{flalign*}

\bibliographystyle{IEEEtran}
\bibliography{tan}

\end{document}
