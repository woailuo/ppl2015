%\documentclass[japanese]{jssst_ppl} %% $BF|K\8l(B (default)
 \documentclass[english]{jssst_ppl} %% English
% \documentclass[japanese,draft]{jssst_ppl} %% You can use the draft option

% set some new commands %
\newcommand\tB{\;|\;}
\newcommand\LET{\mathbf{let}\;}
\newcommand\FREE{\mathbf{free(x)}\;}
\newcommand\IN{\mathbf{in}\;}
\newcommand\SKIP{\mathbf{skip}}
\newcommand\Rtab{\; \; \; \;}
\newcommand\NULL{\mathbf{null}\;}
\newcommand\IFNULL{\mathbf{ifnull}\;}
\newcommand\THEN{\mathbf{then}\;}
\newcommand\ELSE{\mathbf{else}\;}
\newcommand\Lcc{\left(}
\newcommand\Rcc{\right)}
\newcommand\Lfc{\left\{}
\newcommand\Rfc{\right\}}
\newcommand\Lb{\left[}
\newcommand\Rb{\right]}
\newcommand\coma{,\;}
\newcommand\MALLOC{\mathbf{malloc()}\;}
\newcommand\Malloc{\mathbf{malloc}}
\newcommand\Free{\mathbf{free}}
\newcommand\Cirx{(x)}
\newcommand\dtb{\;\;\ \;\;\ \;\;\ \;\;\  }
\newtheorem{theorem}{Theorem}[section]
\newtheorem{lemma}[theorem]{Lemma}
\newtheorem{proposition}[theorem]{Proposition}
\newtheorem{corollary}[theorem]{Corollary}
\newtheorem{myDef}{Definition}


\title{Safe Memory Deallocation for Non-terminating Programs}
\author{Author Name}
\inst{%
$^1$ \\
\texttt{sample@example.ac.jp}
\medskip\par%
$^2$ \\
\texttt{example@sample.net}
}
\begin{document}
\maketitle
\begin{abstract}
We propose a type system to prevent programs infinitely consuming memory cells. The main idea is based on the previous type system that augments pointer types with fractional ownerships and behavioral type system that abstracts the behavior of programs. We design a behavioral type system for verification of behavioral correctness.
\end{abstract}

\section{Introduction}
Manual memory management primitives (e.g., {\tt malloc} and {\tt free} in C) often cause serious problems such as double frees, memory leaks, and illegal read/write to a deallocated memory cell. Verifying \emph{safe memory deallocation} -- a program not leading to such an unsafe state -- is an important problem.

Most of safe memory deallocation verification techniques proposed so far \cite{DBLP:conf/aplas/SuenagaK09,DBLP:conf/pldi/HeineL03,DBLP:conf/sigsoft/XieA05,DBLP:journals/scp/SwamyHMGJ06} focus on the memory leak for \emph{terminating} programs: If a program terminates, the program satisfies safe memory dealloction. For example, the type system by Suenaga and Kobayashi \cite{DBLP:conf/aplas/SuenagaK09} guarrantees that (1) a well-typed program does not perform read/write/free operations to any deallocated memory cell and that (2) after execution of a well-typed program, all the memory cells are deallocated.

Currently, we are investigating safe memory deallocation for non-terminating programs, because the safe memory deallocation is very important in real world programs such as Web servers and operating systems.

The main idea of our approach is to decompose this problem into two subproblems: (1) partial correctness and (2) \emph{behavioral correctness}. The former is verified based on the previous type system \cite{DBLP:conf/aplas/SuenagaK09}, whereas the latter based on the behavioral type system that is mainly used to abstact the behavior of a program. I will describe our idea by several examples written in an ML-like language as follows:

\begin{figure}[h]
1 \;\; $f(x)$= \dtb\dtb \dtb \;\;\;\; \;\;$g(x)$= \\
2 \;\; $\LET \; x = \MALLOC  \; \IN$ \;\dtb \;\;\;\;$\LET \; x = \MALLOC  \; \IN$\\
3 \;\; $\Free\Cirx$; \; $f(x)$\dtb\dtb\;\;\; \;\;$g(x)$; \; $\Free\Cirx$
\caption{Examples of memory leak.}
\label{example:memoryLeak}
\end{figure}
\begin{figure}[h]
1 \;\; $h(x)$=\\
2 \;\; $\LET \; x = \MALLOC  \; \IN$\\
3 \;\; $\LET \; y = \MALLOC  \; \IN$\\
4 \;\; $\Free\Cirx$; \; $\Free(y) $; \; $h(x)$
\caption{Example for demonstrating the main observation.}
\label{example:observation}
\end{figure}

A program \emph{leaks} memory if the program consumes unbounded number of memory cells.  For example, the left-hand side program in Example~\ref{example:memoryLeak} does not leak memory, whereas the right-hand side does; the former consumes at most one memory cell at once
but the latter consumes unbounded number of memory cells.  Notice that both are partially correct because they do not terminate.

Our main observation is as follows: once partial correctness is guaranteed, we can guarantee memory-leak freedom
by estimating upper bound of memory consumption \emph{ignoring the relation between variables and pointers to memory cells}.
Example~\ref{example:observation} describes the observation.  The function \(h\) is partially correct.  Then, in order to
verify that it consumes at most two cells at once, we can ignore \(x\) and \(y\) in \(h\); We can focus on the fact that
\(h\) executes \(\Malloc\) twice, \(\Free\) twice, and then calls \(h\).  This abstraction is sound because
the correspondence between allocations and deallocations is guaranteed by the partial correctness verification.

In order to verify partial correctness, we use the type system by Suenaga and Kobayashi~\cite{DBLP:conf/aplas/SuenagaK09}.
Then, we use a \emph{behavioral type system} for the abstraction described above; For example, the behavior of \(h\)
is abstracted by a CCS-like process \(\mu \alpha. \Malloc.\Malloc.\Free.\Free.\alpha\), which is then passed to a model checker
for estimating memory consumption.

The rest of this paper is structed as follows.

\section{Syntax and Semantics of Language}

\subsection{Syntax}
\begin{eqnarray*}
  s \ (statements)& &::=  \SKIP \tB s_{1};s_{2} \tB *x \leftarrow y \tB \Free \Cirx \\
  & & \tB \LET x = \MALLOC \IN s \tB \LET x = \NULL \IN s  \\
  & & \tB \LET x = y \; \IN s \tB   \LET x = *y \; \IN s \\
  & &\tB \IFNULL(x) \; \THEN s_{1}\; \ELSE s_{2} \tB f(\vec{x})
\end{eqnarray*}
\subsection{Semantics}
  $\mathbf{H}$ models the heap address. \\
  $\mathbf{R}$ models the registers.

% Skip Command
$$
    \frac{n \in \mathbb{N}}
            {\langle H\coma R\coma  \SKIP;s , n \rangle
              \longrightarrow_{D}
                \langle H\coma R\coma   s , n \rangle }
     \Rtab \mbox{(E-Skip)}
$$

% Assignment
$$
     \frac{R(x) \in dom(H), n \in \mathbb{N}}
           {\langle H\coma R\coma  *x \leftarrow y , n \rangle
             \longrightarrow_{D}
             \langle H \Lfc R(x) \rightarrow R(y) \Rfc \coma R \coma   \SKIP , n  \rangle }
     \Rtab \mbox{(E-Assign)}
$$

% Free Command
$$
     \frac{R(x) \in dom(H) , n \in \mathbb{N}}
          {\langle H\coma R\coma  \FREE , n \rangle
            \xlongrightarrow{\Free}_{D}
            \langle H\backslash \Lfc R(x) \Rfc \coma R \coma   \SKIP , n+1  \rangle }
     \Rtab \mbox{(E-Free)}
$$

% Let Null Command
$$
     \frac{x' \notin dom(R)}
           {\langle H\coma R\coma  \LET x = \NULL \IN s , n \rangle
             \longrightarrow_{D}
             \langle H\coma R\Lfc x' \rightarrow \NULL \Rfc \coma   \Lb x'/x \Rb s , n  \rangle }
     \Rtab \mbox{(E-LetNull)}
$$

% Let Eq Command
$$
     \frac{x' \notin dom(R)}
            {\langle H\coma R\coma \LET x = y \; \IN s , n \rangle
              \longrightarrow_{D}
              \langle H\coma R\Lfc x' \rightarrow R(y) \Rfc \coma   \Lb x'/x \Rb s , n  \rangle }
\Rtab \mbox{(E-LetEq)}
$$

% Reference Command
$$
     \frac{x' \notin dom(R)}
            {\langle H\coma R\coma  \LET x = *y \; \IN s , n \rangle
              \longrightarrow_{D}
              \langle H\coma R\Lfc x' \rightarrow H(R(y)) \Rfc \coma   \Lb x'/x \Rb s , n  \rangle }
     \Rtab \mbox{(E-LetDref)}
$$

% Malloc (allocate) Command
$$
     \frac{h \notin dom(H)}
            {\langle H\coma R\coma  \LET x = \Malloc() \; \IN s , n \rangle
              \xlongrightarrow{\Malloc}_{D}
              \langle H \Lfc h \rightarrow v\Rfc \coma R\Lfc x' \rightarrow h \Rfc \coma   \Lb x'/x \Rb s , n-1  \rangle }
\Rtab \mbox{(E-Malloc)}
$$

% IFNULL T
$$
    \frac{R(x) = \NULL}
           {\langle H \coma R \coma \IFNULL\Cirx   \THEN   s_{1} \ELSE\  s_{2} \coma  n \rangle
           \longrightarrow_{D}
           \langle H\coma R\coma s_{1} \coma n \rangle}
    \Rtab \mbox{(E-IfNullT)}
$$

% IFNULL F
$$
    \frac{R(x) \neq \NULL}
           {\langle H \coma R \coma \IFNULL\Cirx \THEN  s_{1} \ELSE  s_{2} \coma  n \rangle
           \longrightarrow_{D}
           \langle H\coma R\coma s_{2} \coma  n, \rangle}
    \Rtab \mbox{(E-IfNullF)}
$$

% Function Call
$$
     \frac{f(\vec{y}) = s \in D}
            { \langle H\coma R\coma  f(\vec{x}) , n \rangle
               \longrightarrow_{D}
               \langle H\coma R\coma  \Lb \vec{x}/\vec{y} \Rb s , n \rangle}
      \Rtab \mbox{(E-Call)}
$$

% Error : access the null memory cell
$$
      \frac{R(x) = null}
            {\langle H\coma R\coma  *x \leftarrow y , n \rangle
              \longrightarrow_{D}
              NullEx }
      \Rtab \mbox{(E-AssignNullError)}
$$

% ERROR : access the null memory cell
$$
      \frac{R(y) = null}
             {\langle H\coma R\coma  x = *y, n \rangle
               \longrightarrow_{D}
               NullEx }
             \Rtab \mbox{(E-DrefNullError)}
$$

% ERROR :
$$
     \frac{R(x) \notin dom(H) \cup \Lfc null \Rfc}
           {\langle H\coma R\coma   *x \leftarrow y,  n \rangle
             \longrightarrow_{D}
             Error }
    \Rtab \mbox{(E-AssignError)}
$$

% ERROR
$$
      \frac{R(y) \notin dom(H) \cup \Lfc null \Rfc}
           {\langle H\coma R\coma  \LET x  = *y \; \IN s, n \rangle
              \longrightarrow_{D}
                  Error }
      \Rtab \mbox{(E-DrefError)}
$$

%
$$
      \frac{R(x) \notin dom(H) \cup \Lfc null \Rfc}
            {\langle H\coma R\coma  \FREE , n \rangle
              \xlongrightarrow{\Free}_{D}
              Error }
     \Rtab \mbox{(E-FreeError)}
$$

%
$$
     \frac{R(x) =  null }
           {\langle H\coma R\coma  \FREE , n \rangle
             \xlongrightarrow{\Free}_{D} NullEx  }
      \Rtab \mbox{(E-FreeNullError)}
$$

% ERROR: no enough space
$$
      \langle H\coma R\coma \LET x = \Malloc() \ \IN s ,  0  \rangle
      \xlongrightarrow{\Malloc}_{D}
      Error
      \Rtab \mbox{(E-MallocError)}
$$

\section{Type System}
\subsection{Syntax of Type}
     \begin{eqnarray*}
       P (behaviral \ types)::=&& {\bf 0} \tB P_{1};P_{2} \tB P_{1}+P_{2} \tB \Malloc\\
       &&\tB \Free \tB \alpha \tB \mu\alpha.P \\
       \tau (value \ types)::=&&    \mathbf{Ref}  \\ %
       \sigma (function \ types)::=&& (\tau_{1},\dots, \tau_{n}) P
     \end{eqnarray*}

% Semantics of Behavioral Types %
\subsection{Semantics of Behavioral Types}
    $$
         0;P \rightarrow P
    $$
    $$
          \mu \alpha.P \rightarrow  [\mu \alpha . P/\alpha]  P
    $$
    $$
          \Malloc \xlongrightarrow{\Malloc} 0
    $$
    $$
           \Free \xlongrightarrow{\Free} 0
    $$
    $$
           \frac{P_{1} \xlongrightarrow{\alpha} P_{1}' }
                 {P_{1};P_{2} \xlongrightarrow{\alpha} P_{1}';P_{2}}
   $$
   $$
          P_{1} + P_{2} \longrightarrow P_{1}
   $$
   $$
          P_{1} + P_{2} \longrightarrow P_{2}
   $$
   $
        \text{where} \  \alpha ::= \Malloc\ | \ \Free
   $

% Type Judgments
\subsection{Type Judgments}
    $$
        \Theta ; \Gamma \vdash s : P
    $$
   %$$\Theta ; \Gamma \vdash x : \tau $$
    $\Theta$ : a finite mapping from function variables to function types.\\
    $\Gamma$ : a finite mapping from variables to value types.

% Typing Rules %
\subsection{Typing Rules}

% Skip type
$$
         \Theta ; \Gamma \vdash \SKIP : \mathbf{0}
      \Rtab \mbox{(T-Skip)}
$$

% Sequence type
$$
      \frac{\Theta ; \Gamma \vdash s_{1} : P_{1} \Rtab \Theta ; \Gamma \vdash s_{2} : P_{2}}
          {\Theta ; \Gamma \vdash s_{1} ; s_{2} : P_{1};P_{2} }
     \Rtab \mbox{(T-Seq)}
$$

% Assignment type
$$
     \frac{\Theta ; \Gamma \vdash y :  \mathbf{Ref} \Rtab \Theta ; \Gamma \vdash x : \mathbf{Ref} }
          {\Theta ; \Gamma \vdash *x \leftarrow y : \mathbf{0} }
     \Rtab \mbox{(T-Assign)}
$$

% Free(deallocate) type
$$
     \frac{\Theta ; \Gamma \vdash x : \mathbf{Ref} }
           {\Theta ; \Gamma \vdash \Free(x) : \Free;0}
     \Rtab \mbox{(T-Free)}
$$

% Malloc type
$$
     \frac{\Theta ; \Gamma,x \vdash s : P}
           {\Theta ; \Gamma \vdash \LET x = \MALLOC \; \IN s  : \Malloc;P}
           \Rtab \mbox{(T-Malloc)}
$$

% Let eq type
$$
     \frac{\Theta ; \Gamma \vdash y : \mathbf{Ref}  \Rtab \Theta ; \Gamma , x  \vdash s : P}
           {\Theta ; \Gamma \vdash \LET x = y \; \IN s : P}
     \Rtab \mbox{(T-LetEq)}
$$

% Dereference type
$$
     \frac{\Theta ; \Gamma \vdash y : \mathbf{Ref}  \Rtab \Theta ; \Gamma , x  \vdash s : P}
           {\Theta ; \Gamma \vdash \LET x = *y \; \IN s : P}
     \Rtab \mbox{(T-LetDref)}
$$

% Let NULL type
$$
     \frac{\Theta ; \Gamma, x \vdash s : P}
           {\Theta ; \Gamma \vdash \LET x = \mathbf{null} \; \IN s : P}
     \Rtab \mbox{(T-LetNull)}
$$

% ifnull s then s type
$$
     \frac{\Theta ; \Gamma \vdash x : \mathbf{Ref}   \ \ \ \  \Theta ; \Gamma \vdash s_{1} : P \ \ \ \ \Theta ; \Gamma \vdash s_{1} : P}
           {\Theta ; \Gamma \vdash \IFNULL(x) \; \THEN s_{1}\; \ELSE s_{2} : P}
     \Rtab \mbox{(T-IfNull)}
$$

% Function call type
$$ \frac{ \Theta(f) = P}
{\Theta; \Gamma, \vec{x} : \vec{\tau} \vdash f(\vec{x}) : P}
\Rtab \mbox{(T-Call)} $$

% Program
$$\frac{\vdash D : \Theta \;\;\;\; \Theta; \emptyset\vdash s : P \Rtab OK_{n}(P)}
{\vdash (D, s)}
\Rtab \mbox{(T-Program)} $$

% Subtyping
$$
     \frac{\Theta ; \Gamma \vdash s : P_{1} \Rtab P_{1} \le P_{2}}
            {\Theta ; \Gamma \vdash s : P_{2}}
     \Rtab \mbox{(T-Sub)}
$$

\section{Type Soundness}
\begin{theorem}\label{thm1}
If $\vdash (D, s)$ then $(D, s)$ does not lead to $memory\;leak$.\\
Memory leak freedom: $\exists n \in \mathbb{N}$ s.t.
$\langle \emptyset, \emptyset, s, n \rangle \nrightarrow^{*}Error$
\end{theorem}
\noindent

% Lemma Preservation %
\begin{lemma}[Preservation $\mathbf{I}$]%\label{preser}
If $OK_{n}(P)$, $\Theta; \Gamma \vdash s : P$ and $\langle H\coma R\coma s\coma n \rangle
\xlongrightarrow{\alpha}\langle H'\coma R'\coma s'\coma n'
\rangle$, then $\exists P'$ s.t. \\
(1) $ \Theta; \Gamma \vdash s' : P' $ \\
(2) $ P \overset{\text{$\alpha$}}{\Longrightarrow} P'$\\
(3) $ OK_{n'}(P') $
\end{lemma}
\begin{lemma}[Preservation $\mathbf{II}$]%\label{preser}
If $OK_{n}(P)$, $\Theta ; \Gamma \vdash s : P$ and $\langle H\coma R\coma s\coma n \rangle
\rightarrow \langle H'\coma R'\coma s'\coma n'
\rangle$, then $\exists P'$ s.t. \\
(1) $\Theta; \Gamma \vdash s' : P'$\\
(2) $ P \xlongrightarrow{\tau}^{*} P'  $\\
(3) $OK_{n'}(P')$
\end{lemma}

\begin{lemma}%\label{error}
$when \  partial \  correctness \  is \  guaranteed \  \vdash \langle H,R,s \rangle and \  if \ \vdash \langle H,R,s,n \rangle, \  then \  \vdash \langle
H',R',s',n' \rangle \nrightarrow Error$
\end{lemma}

\section{Syntax Directed Typing Rules}

$C$ is contraint for subtype. \\

$$
     \frac{ C = \emptyset}
           {\Theta; \Gamma; C \vdash \SKIP : \mathbf{0}
             }
      \Rtab \mbox{(ST-Skip)}
$$

$$
      \frac{\Theta;\Gamma ; C_{1} \vdash s_{1} : P_{1} \Rtab \Theta; \Gamma ; C_{2} \vdash s_{2} : P_{2} \Rtab C = C_{1}\cup C_{2} \cup \{ P_{1};P_{2} \le P\}}
      {\Theta;\Gamma; C \vdash s_{1};s_{2} : P}
      \Rtab \mbox{(ST-Seq)}
$$

$$
      \frac{\Theta;\Gamma;C_{1} \vdash y \Rtab \Theta;\Gamma; C_{2} \vdash x : \mathbf{Ref} \Rtab C = C_{1}\cup C_{2}}
      {\Theta;\Gamma; C \vdash *x \leftarrow y : \mathbf{0}}
      \Rtab \mbox{(ST-Assign)}
$$

$$
      \frac{C = \emptyset}
      {\Gamma ; C \vdash \Free() : \Free;\mathbf{0}}
     \Rtab \mbox{(ST-Free)}
$$

$$
     \frac{\Theta;\Gamma, x ; C_{1} \vdash s : P_{1} \Rtab C = C_{1} \cup\{P_{1}\le P\}}
     {\Theta;\Gamma; C \vdash \LET x = \Malloc() \; \IN s : \Malloc ; P}
     \Rtab \mbox{(ST-Malloc)}
$$

$$
     \frac{\Theta;\Gamma; C_{1} \vdash y \Rtab \Theta;\Gamma, x ; C_{2} \vdash s : P_{1} \Rtab C = C_{1}\cup C_{2} \cup \{P_{1} \le P \}}
     {\Theta;\Gamma ; C \vdash \LET x = y \;  \IN s : P}
     \Rtab \mbox{(ST-LetEq)}
$$

$$
     \frac{\Theta;\Gamma ; C_{1} \vdash y: \mathbf{Ref} \Rtab \Theta;\Gamma, x ; C_{2} \vdash s : P_{1} \Rtab C = C_{1}\cup C_{2}\cup\{P_{1} \le P\}}
     {\Theta;\Gamma ; C \vdash \LET x = *y \; \IN s : P}
     \Rtab \mbox{(ST-LetDref)}
$$


$$
     \frac{\Theta;\Gamma; C_{1} \vdash x \Rtab \Theta;\Gamma; C_{2} \vdash s_{1} : P_{1} \Rtab \Theta;\Gamma; C_{3} \vdash s_{2} : P_{2}  \Rtab  C = C_{1} \cup C_{2} \cup C_{3} \{P_{1}\le P, P_{2}\le P \}}
     {\Theta;\Gamma; C \vdash \IFNULL\Cirx \THEN s_{1} \ELSE s_{2} : P }
     \Rtab \mbox{(ST-IfNull)}
$$

$$
     \frac{\Theta(f) = P_{1} \Rtab C = P_{1} \le P}
     {\Gamma,\vec{x}:\vec{\tau} \vdash f(\vec{x}) : P }
     \Rtab \mbox{(ST-Call)}
$$

$$
     \frac{\Theta \vdash D : \Theta \Rtab \Theta ; \emptyset ; C_{1} \vdash s : P \Rtab C = C_{1}\cup\{OK_{n}(P)\}}
     {C \vdash (D , s) }
     \Rtab \mbox{(ST-Prog)}
$$

\section{Type Inference}
$PT_{v}(x) = (\emptyset,\emptyset)$, where $x$ maybe a value or reference.\\
$PT_{\Theta}$ is a mapping from statements to a pair of constraints and types with function $\Theta$ from function names to function types, like $\Theta(f) = P$.
\begin{flalign*}
   PT_{\Theta}(f) &  =  &\\
  \LET & \alpha = \Theta(f) & \\
  \IN  & (C = \{\alpha \le \beta \}, \beta) &
\end{flalign*}

\begin{flalign*}
   PT_{\Theta}(\SKIP) &  =  (\emptyset, 0)&
\end{flalign*}

\begin{flalign*}
   PT_{\Theta}(s_{1};s_{2}) &  =  &\\
   \LET & (C_{1}, P_{1}) = PT_{\Theta}(s_{1}) & \\
        & (C_{2}, P_{2}) = PT_{\Theta}(s_{2}) & \\
   \IN   &(C_{1} \cup C_{2}\cup \{P_{1}; P_{2} \le \beta \}, \beta) &
\end{flalign*}

\begin{flalign*}
   PT_{\Theta}(*x \leftarrow y) &  =  &\\
   \LET & (C_{1}, \emptyset) = PT_{v}(*x) & \\
        & (C_{2}, \emptyset) = PT_{v}(y) & \\
  \IN &(C_{1} \cup C_{2},  0) &
\end{flalign*}

\begin{flalign*}
   PT_{\Theta}(\Free(x)) &  = (\emptyset, \Free; 0)  &
\end{flalign*}

\begin{flalign*}
   PT_{\Theta}(\LET x = \Malloc() \  \IN s) &  =  &\\
   \LET  (C_{1}, P_{1})& = PT_{v}(s) & \\
   \IN (C_{1} \cup \{P_{1}& \le \beta \} ,  \Malloc; \beta) &
\end{flalign*}

\begin{flalign*}
   PT_{\Theta}(\LET x = y \  \IN s ) &  =  &\\
   \LET & (C_{1}, \emptyset) = PT_{v}(y) & \\
  & (C_{2}, P_{1}) = PT_{\Theta}(s) & \\
  \IN &(C_{1} \cup C_{2}\cup \{P_{1} \le \beta \},  \beta) &
\end{flalign*}

\begin{flalign*}
   PT_{\Theta}(\LET x = *y \  \IN s ) &  =  &\\
   \LET & (C_{1}, \emptyset) = PT_{v}(y) & \\
        & (C_{2}, P_{1}) = PT_{\Theta}(s) & \\
  \IN &(C_{1} \cup C_{2}\cup \{P_{1} \le \beta \},  \beta) &
\end{flalign*}

\begin{flalign*}
   PT_{\Theta}(\IFNULL(x) \  \THEN  s_{1} \  \ELSE \ s_{2} ) &  =  &\\
   \LET & (C_{1}, P_{1}) = PT_{\Theta}(s_{1}) & \\
        & (C_{2}, P_{2}) = PT_{\Theta}(s_{2}) & \\
        & (C_{3}, \emptyset) = PT_{v}(x) & \\
  \IN &(C_{1} \cup C_{2}\cup C_{3}\cup \{P_{1} \le \beta, P_{2} \le \beta \},  \beta) &
\end{flalign*}

\begin{flalign*}
   PT(<D, s>) &  =  &\\
   \LET & \Theta = \{ f_{1}:\alpha_{1}, \dots, f_{n}:\alpha_{n}  \} &\\
        & where \ \{ f_{1},\dots, f_{n} \} = dom(D) \ and \ \alpha_{1}, \dots, \alpha_{n} \  are \ fresh  & \\
  \IN   & \LET  (C_{i}, P_{i}) = PT_{\Theta}(D(f_{i})) \  for \  each \ i & \\
  \IN   & \LET  C_{i}^{'} = \{ \alpha_{i} \le P_{i} \} \ for \  each \ i & \\
  \IN   & \LET  (C, P) = PT_{\Theta}(s)  & \\
  \IN  &(C_{i} \cup C_{i}^{'} ) \cup C \cup  \{OK(P)\},  P) &
\end{flalign*}

\sectiion{Experiments}

\section{Conclusion}
this is the end

%\bibliographystyle{amsplain}
\thebibliography{tan}

\end{document}
