\section{Conclusion}

We have described a type system to verify memory-leak freedom for
(possibly) nonterminating programs with manual memory-management
primitives where every memory cell is fixed size.  Our type system
abstracts the memory allocation/deallocation behavior of a program
with a CCS-like process with actions corresponding to memory
allocation and deallocation.  We have described a type reconstruction
algorithm for the type system.

Our current type system excludes many features of the real-world
programs for simplification.  We are currently investigating the C
programs in the real world to investigate what extension we need to
make to the type system.  One feature we have already noticed is
variable-sized memory blocks.  The current behavioral types ignores
the size of the allocated block, counting only the number of
\(\Malloc\) and \(\Free\), which makes the abstraction unsound for
actual programs.

Flow sensitivity is another issue we are pursuing.  \todo{Complete
  this paragraph.}

%% type-based approach to safe memory deallocation
%% for non-terminating programs. The approach is based on the idea of
%% decomposing safe memory memory deallocation into partial correctness,
%% which is verified by previous type system, and behavioral
%% correctness. We designed a behavioral type system in our paper for
%% verification of behavioral correctness. Currently, we are looking for
%% a model checker to estimate an upper bound of consumption given a
%% behavioral type and planning to implement a verifier and conduct
%% experiment to see whether our approach is feasible.
