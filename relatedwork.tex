
\section{Related work}\label{sec:relatedwork}
Many methods for static memory-leak freedom verification have been
proposed~\cite{DBLP:conf/aplas/SuenagaK09,DBLP:conf/pldi/HeineL03,DBLP:conf/sigsoft/XieA05,DBLP:journals/scp/SwamyHMGJ06,DBLP:conf/sas/OrlovichR06,DBLP:conf/issta/SuiYX12}. These
methods guarantee partial memory-leak freedom and lack of illegal
accesses, whereas our type system guarantees total memory-leak
freedom. By using both their methods and our type system, we can
guarantee that a program correctly uses memory-allocation and
memory-deallocation primitives even if the program does not terminate.

Behavioral types are extensively studied in the context of concurrent
program
verification~\cite{DBLP:conf/esop/HondaVK98,DBLP:journals/tcs/IgarashiK04,DBLP:conf/esop/VieiraCS08,DBLP:journals/lmcs/KobayashiSW06}.
These type systems guarantee that the communication pattern of
concurrent programs are as intended.  Our type system is largely
inspired by one proposed by Kobayashi et
al.~\cite{DBLP:journals/lmcs/KobayashiSW06}, which guarantees that a
concurrent program accesses resources according to specification.

%% Model checking is a widely used technique for automatically verifying
%% correctness properties of finite-state
%% system~\cite{clarke1999model,ben2008principles,beyer2011cpachecker}. We
%% assume that some model checkers solves the constraints \(\OK_\nu(P)\)
%% in our paper. We expect that applying model checker to inferred
%% behavioral types is better than to the original programs, because a
%% behavioral type focuses on the actions related to allocations and
%% deallocations, abstracted away from other features. We plan to do some
%% experiments for this prospect.

One possible approach to total memory-leak freedom verification would
be checking that a program consumes only bounded number of memory
cells by using a model
checker~\cite{clarke1999model,ben2008principles,beyer2011cpachecker}.
We expect, from the result of the experiments, that the resource
required for model checking would become smaller if we apply a model
checker to the inferred behavior, which focuses on memory allocation
and deallocation.
