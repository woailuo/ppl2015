
\section{Language \(\mathcal{L}\)}\label{sec:language}

The following BNF defines our language \(\mathcal{L}\).

\begin{eqnarray*}
  x,y,z,\dots \mbox{ (variables)} &\in& \VAR\\
  s \mbox{ (statements)} & ::= &  \SKIP \mid s_{1};s_{2} \mid *x \leftarrow y \mid \Free(x) \\
  & \mid & \LET x = \MALLOC \IN s \mid \LET x = \NULL\ \IN s  \\
  & \mid & \LET x = y \; \IN s \mid   \LET x = *y \; \IN s \\
  & \mid & \IFNULL(x) \; \THEN s_{1}\; \ELSE s_{2} \mid f(\vec{x})\\
  d \mbox{ (proc. defs.)} & ::= & \set{f \mapsto (x_1,\dots,x_n)s}\\
  D \mbox{(definitions) } &::=& \langle d_1 \cup \dots \cup d_n \rangle\\
  P \mbox{ (programs)} &::=& \langle D, s \rangle\\
\end{eqnarray*}

The language is equipped with procedure calls, dynamic memory
allocation and deallocation, and memory accesses with pointers.
\(\VAR\) is a countably infinite set of \emph{variables}.  The
statement \(\SKIP\) does nothing.  The statement \(s_1;s_2\) executes
\(s_1\) and \(s_2\) sequentially.  The statement \(*x \leftarrow y\)
writes \(y\) to the memory cell that \(x\) points to.  The statement
\(\LET x = e\ \IN s\) evaluates the expression \(e\), binds \(x\) to
the result, and executes \(s\).  The expression \(\Malloc()\)
allocates a new memory cell and evaluates to the pointer to the cell.
The expression \(\NULL\) evaluates to the null pointer.  The
expression \(y\) evaluates to its value.  The expression \(*y\)
evaluates to the value in the memory cell that \(y\) points to.  The
statement \(\IFNULL(x)\ \THEN\ s_1\ \ELSE\ s_2\) executes \(s_1\) if
\(x\) is \(\NULL\) and executes \(s_2\) otherwise.  The statement
\(f(\vec{x})\) calls procedure \(f\) with arguments \(\vec{x}\).

A procedure definition ranged over by \(d\) is a map from a procedure
name to an abstraction of the form \((\vec{x})s\).  We assume that
there is no arity mismatch between function definitions and function
calls.  We use a metavariable \(D\) for a set of function definitions
\(d_1 \cup \dots \cup d_n\).  A program is a pair of function
definitions \(D\) and a main statement \(s\).


%% A program is a pair $(D,s)$, where $D$ is the set of definition.\\ The
%% command $\SKIP$ does nothing. The command $s_{1};s_{2}$ is executed as
%% a sequence, first executing $s_{1}$ and then $s_{2}$. The command $*x
%% \leftarrow y$ updates the content of the memory cell which is pointed
%% by pointer $x$ with value $y$. The command $\Free \Cirx$ deallocates
%% the memory cell which is pointed by a pointer $x$. Then command $\LET
%% x = e \; \IN s$ first evaluates the expression $e$ and binds the
%% return value of $e$ to $x$ and then executes statement $s$. The
%% command $\LET x = \Malloc \; \IN s$ first allocates a memory cell to a
%% pointer $x$ and then executes the statement $s$. The command $\LET x =
%% \NULL \IN s$ first allocates a null pointer to $x$ and then executes
%% $s$. The command $\LET x = y \; \IN s$ assign the pointer $y$ to $x$,
%% so the pointer $x$ and $y$ are said aliases for the same memory cell,
%% and then executes statement $s$. The command $\LET x = *y \; \IN s$
%% transfers a part of memory cells pointed by $y$ and then executes
%% statement $s$. The command $\IFNULL(x) \; \THEN s_{1} \; \ELSE s_{2}$
%% denotes that executing statement $s_{1}$ if pointer $x$ is a null
%% pointer, otherwise executing statement $s_{2}$. The command
%% $f(\vec{x})$ is a function call in which $\vec{x}$ denotes mutually
%% distinct variables like \{$x_{1}, \dots, x_{n}$\}. The notation $d$
%% denotes the definition of function $f(\vec{x})$ which has a body of
%% statement $s$. And examples are described by this syntax you can see
%% in Figure 1 and Figure 2.

\subsection{Operational semantics}
\label{sec:languageSemantics}

This section introduces the operational semantics of \(\mathcal{L}\).
\(\mathcal{H}\) is a countably infinite set of \emph{locations} ranged
over by \(l\).

We give the operational semantics of the language \(\mathcal{L}\) as a
labeled transition on \emph{configurations} \(\langle H, R, s, n
\rangle\).  A configuration consists of the following four components:
\begin{itemize}
\item \(H\), a \emph{heap}, is a finite mapping from \(\mathcal{H}\)
  to \(\mathcal{H} \cup \set{\NULL}\);
\item \(R\), an \emph{environment}, is a finite mapping from \(\VAR\)
  to \(\mathcal{H} \cup \set{\NULL}\);
\item \(s\) is the statement that is being executed; and
\item \(n\) is a natural number that represents the number of
  available memory cells.
\end{itemize}
\(n\) in a configuration is later used to formalize memory leaks
caused by nonterminating program.

The operational semantics is given by a labeled transition relation
\(\langle H, R, s, n \rangle \xlongrightarrow{\rho}_D \langle H', R',
s', n' \rangle\) where \(\rho\), an \emph{action}, is \(\Malloc\),
\(\Free\), or \(\tau\).  The action \(\Malloc\) expresses an
allocation of a memory cell; \(\Free\) expresses a deallocation;
\(\tau\) expresses the other actions.  We often omit \(\tau\) in
\(\xlongrightarrow{\tau}_D\).  We use a metavariable \(\sigma\) for a
finite sequence of actions \(\rho_1\dots\rho_n\).  We write
\(\xlongrightarrow{\rho_1\dots\rho_n}_D\) for
\(\xlongrightarrow{\rho_1}_D\xlongrightarrow{\rho_2}_D\dots\xlongrightarrow{\rho_n}_D\).
We write \(\xLongrightarrow{\rho}_D\) for
\(\xlongrightarrow{}_D^*\xlongrightarrow{\rho}_D\xlongrightarrow{}_D^*\).
We write \(\xLongrightarrow{\rho_1\dots\rho_n}_D\) for
\(\xLongrightarrow{\rho_1}_D\dots\xLongrightarrow{\rho_n}_D\).

%% Because we want to estimate the number of available memory cells at
%% every operation step, we extend the triple $\langle H\coma R\coma s
%% \rangle$ that is represented as run-time state in previous type system
%% to a quadruple $\langle H\coma R\coma s\coma n \rangle$ in our
%% paper. The introduced notation $n$ denotes the number of available
%% memory cells, a nature number. When executing the operation $\Malloc$,
%% the number of available memory cells will decrease 1, which is denoted
%% as ($n - 1$); when executing the operation $\Free$, the number of
%% available memory cells will increase 1, which is denoted as ($n +
%% 1$). The notation $H$, which models heap memory, is a mapping from
%% finite subset of $\mathcal{H}$ to $\mathcal{H}$ $\cup$ \{$null$\},
%% where $\mathcal{H}$ represents the set of \emph{heap addresses}. $R$,
%% which models registers, is a mapping from finite set of variables to
%% $\mathcal{H}$ $\cup$ \{$null$\}.

% <<<<<<< HEAD
%% \todo{Rewrite this paragraph.}  Figure~\ref{fig:transitionRules}
%% defines the relation \(\xlongrightarrow{\rho}_D\). In the \rn{Sem-Free}
%% rule, before deallocating a memory cell, we require \(R(x) \in
%% \DOM(H)\); we remove the pointer from the heap \(H\) and the number
%% \(n\) of available memory cells should increase one after
%% deallocation.  In the \rn{Sem-Malloc} rule, the \(\Malloc()\) command
%% allocates a memory cell and returns a pointer to a variable; contents
%% \(v\) of the allocated memory cell can be any value in \(\mathcal{H}
%% \cup \set{\NULL}\); the number \(n\) should decrease one after
%% allocation.  There are three kinds of run-time errors: \(\bf NullEx\)
%% for accessing null pointers, \(\bf MemEx\) for illegal accesses to a
%% deallocated memory cell, and \(\bf Overflow\) for performing
%% allocation after running out of memory cells. In our paper, we only
%% prevent the \(\bf Overflow\).
%% %% =======
Figure~\ref{fig:transitionRules} defines the relation
\(\xlongrightarrow{\rho}_D\). In the rule for \(\Free\), we require
that a memory cell has been allocated before deallocating it, and
after deallocating the memory cell from the heap \(H\), the number
\(n\) of available memory cells should increase one.  In the rule for
\(\Malloc\), the number of available memory cells should be positive;
the \(\Malloc\) command allocate e cell for allocates a new memory
cell \(l\) of which contents \(v\) can be any value in \(\mathcal{H}
\cup \set{\NULL}\) and reduces the number of available memory cells by
one. The \(\bf NullEx\) for accessing null pointers or illegal accessing to a
deallocated memory cell, and \(\OVERFLOW\) for performing allocation
after running out of memory cells.



\begin{figure}[p]
\begin{minipage}{\textwidth}

%% =======
%% %% Transition rules are listed in Figure 3. In these rules, $f\{x\to v\}$ is defined as a function $f'$ such that $f'(y) = v$ if $x = y$, otherwise $f'(y) = f(y)$ and $y \in dom(f)$. There are three rules about $\mathbf{NullEx}$ which denotes accessing a null pointer, three rules about $\mathbf{Error}$ for accessing a deallocated memory cell, and one rule about $\mathbf{Error}$ which denotes allocating a memory cell when there is no memory space.
%% %\begin{figure}[h]
%% % Skip Command
%% Runtime state is represented by $\langle H, R, s, n \rangle$, where $H$ is a mapping from finite subset of $\mathcal{H}$ to $\mathcal{H}$ $\cup$ \{$null$\}, in which $\mathcal{H}$ represents the set of \emph{heap addresses}, intuitively, the $H$ models a heap memory; $R$, which models registers, is a mapping from finite set of variables to $\mathcal{H}$ $\cup$ \{$null$\}.

%% Operational semantics are defined by the relations: $\rightarrow_{D}, \xlongrightarrow{\Malloc}_{D}, and \xlongrightarrow{\Free}_{D} $, in which $\xlongrightarrow{\Malloc}_{D}, and \xlongrightarrow{\Free}_{D} $ denote performing a allocation and deallocation operation respectively, and $\rightarrow_{D}$ expresses that performing an internal action like $\SKIP$ or assignment.
%% >>>>>>> 9a92815042ab957eb9b2d406af7a45d4abb729b3

\begin{minipage}{0.5\textwidth}
\infax[Sem-Skip]
{\langle H, R, \SKIP;s, n \rangle
\longrightarrow_{D}
\langle H, R, s, n \rangle}
\end{minipage}
\begin{minipage}{0.5\textwidth}
\infrule[Sem-Seq]
{\langle H, R, s_1, n \rangle \xlongrightarrow{\rho}_{D} \langle H', R', s_1', n' \rangle}
{\langle H, R, s_1;s_2, n \rangle \xlongrightarrow{\rho}_{D} \langle H', R', s_1';s_2, n' \rangle}
\end{minipage}

\infax[Sem-Assign]
{ \langle H \set{R(x) \mapsto v}, R, *x \leftarrow y , n \rangle \xlongrightarrow{\tau}_{D}
  \langle H \Lfc R(x) \mapsto R(y) \Rfc , R, \SKIP , n \rangle }

\infax[Sem-Free] 
{\langle H\set{R(x) \mapsto v}\coma R\coma \Free(x) , n \rangle \xlongrightarrow{\Free}_{D}
  \langle H\backslash R(x) \coma R \coma \SKIP , n+1 \rangle}

\infrule[Sem-LetNull]
{x' \notin \DOM(R)}
{\langle H\coma R\coma  \LET x = \NULL \ \IN s , n \rangle
  \longrightarrow_{D}
  \langle H\coma R\Lfc x' \mapsto \NULL \Rfc \coma   \Lb x'/x \Rb s , n  \rangle }

\infrule[Sem-LetEq]
{x' \notin \DOM(R)}
{\langle H\coma R\coma \LET x = y \; \IN s , n \rangle
  \longrightarrow_{D}
  \langle H\coma R\Lfc x' \mapsto R(y) \Rfc \coma   \Lb x'/x \Rb s , n  \rangle }

\infrule[Sem-LetDeref]
{x' \notin \DOM(R) \andalso R(y) \in \DOM(H)}
{\langle H\coma R\coma  \LET x = *y \; \IN s , n \rangle
  \longrightarrow_{D}
  \langle H\coma R\Lfc x' \mapsto H(R(y)) \Rfc \coma   \Lb x'/x \Rb s , n  \rangle }

\infrule[Sem-Malloc]
{l \notin \DOM(H) \andalso n > 0}
{\langle H\coma R\coma  \LET x = \Malloc() \; \IN s , n \rangle
  \xlongrightarrow{\Malloc}_{D}
  \langle H \Lfc l \rightarrow v\Rfc \coma R\Lfc x' \rightarrow l \Rfc \coma   \Lb x'/x \Rb s , n-1  \rangle }

\infrule[Sem-IfNullT]
{R(x) = \NULL}
{\langle H \coma R \coma \IFNULL\Cirx\ \THEN   s_{1}\ \ELSE\  s_{2} \coma  n \rangle
  \longrightarrow_{D}
  \langle H\coma R\coma s_{1} \coma n \rangle}

\infrule[Sem-IfNullF]
{R(x) \neq \NULL}
{\langle H \coma R \coma \IFNULL\Cirx\ \THEN  s_{1}\ \ELSE  s_{2} \coma  n \rangle
  \longrightarrow_{D}
  \langle H\coma R\coma s_{2} \coma  n \rangle}

\infrule[Sem-Call]
{D(f) = (\vec{y})s}
{ \langle H\coma R\coma  f(\vec{x}) , n \rangle
  \longrightarrow_{D}
  \langle H\coma R\coma  \Lb \vec{x}/\vec{y} \Rb s , n \rangle}

\begin{minipage}{0.5\textwidth}
\infrule[Sem-AssignExn]
{R(x) = \NULL \mbox{ or } R(x) \notin \DOM(H)}
{\langle H\coma R\coma  *x \leftarrow y , n \rangle
  \longrightarrow_{D} \MEMEX }
\end{minipage}
\begin{minipage}{0.5\textwidth}
\infrule[Sem-DerefExn]
{R(y) = \NULL \mbox{ or } R(y) \notin \DOM(H)}
{\langle H\coma R\coma  \LET x = *y \; \IN s, n \rangle
    \longrightarrow_{D} \MEMEX}
\end{minipage}

\infrule[Sem-FreeExn]
{R(x) = \NULL \mbox{ or } R(x) \notin \DOM(H)}
{\langle H\coma R\coma \Free(x) , n \rangle \xlongrightarrow{\Free}_{D} \MEMEX}

\infax[Sem-OutOfMem]
{ \langle H\coma R\coma \LET x = \Malloc() \ \IN s ,  0  \rangle    \xlongrightarrow{\Malloc}_{D}
     \OVERFLOW}

\end{minipage}

\caption{Operational semantics of \(\mathcal{L}\).}
\label{fig:transitionRules}
\end{figure}

%% Transition rules are listed in Figure 3. In these rules, $f\{x\to v\}$
%% is defined as a function $f'$ such that $f'(y) = v$ if $x = y$,
%% otherwise $f'(y) = f(y)$ and $y \in dom(f)$. There are three rules
%% about $\mathbf{NullEx}$ which denotes accessing a null pointer, three
%% rules about $\mathbf{Error}$ for accessing a deallocated memory cell,
%% and one rule about $\mathbf{Error}$ which denotes allocating a memory
%% cell when there is no memory space.
%% %\begin{figure}[h]
%% % Skip Command
%% Runtime state is represented by $\langle H, R, s, n \rangle$, where
%% Operational semantics is defined by the relations\\
%% $\rightarrow_{D}, \xlongrightarrow{\Malloc}_{D}, and \xlongrightarrow{\Free}_{D} $ defined by the rules in ....\\
%% Here, $\rightarrow_{D}$ expresses; ....\\

%% $$
%%     \frac{n \in \mathbb{N}}
%%             {\langle H\coma R\coma  \SKIP;s , n \rangle
%%               \longrightarrow_{D}
%%                 \langle H\coma R\coma   s , n \rangle }
%%      \Rtab \mbox{(E-Skip)}
%% $$
%% % Assignment
%% $$
%%      \frac{R(x) \in dom(H), n \in \mathbb{N}}
%%            {\langle H\coma R\coma  *x \leftarrow y , n \rangle
%%              \longrightarrow_{D}
%%              \langle H \Lfc R(x) \rightarrow R(y) \Rfc \coma R \coma   \SKIP , n  \rangle }
%%      \Rtab \mbox{(E-Assign)}
%% $$
%% % Free Command
%% $$
%%      \frac{R(x) \in dom(H) , n \in \mathbb{N}}
%%           {\langle H\coma R\coma  \FREE , n \rangle
%%             \xlongrightarrow{\Free}_{D}
%%             \langle H\backslash \Lfc R(x) \Rfc \coma R \coma   \SKIP , n+1  \rangle }
%%      \Rtab \mbox{(E-Free)}
%% $$
%% % Let Null Command
%% $$
%%      \frac{x' \notin dom(R)}
%%            {\langle H\coma R\coma  \LET x = \NULL \IN s , n \rangle
%%              \longrightarrow_{D}
%%              \langle H\coma R\Lfc x' \rightarrow \NULL \Rfc \coma   \Lb x'/x \Rb s , n  \rangle }
%%      \Rtab \mbox{(E-LetNull)}
%% $$
%% % Let Eq Command
%% $$
%%      \frac{x' \notin dom(R)}
%%             {\langle H\coma R\coma \LET x = y \; \IN s , n \rangle
%%               \longrightarrow_{D}
%%               \langle H\coma R\Lfc x' \rightarrow R(y) \Rfc \coma   \Lb x'/x \Rb s , n  \rangle }
%% \Rtab \mbox{(E-LetEq)}
%% $$
%% % Reference Command
%% $$
%%      \frac{x' \notin dom(R)}
%%             {\langle H\coma R\coma  \LET x = *y \; \IN s , n \rangle
%%               \longrightarrow_{D}
%%               \langle H\coma R\Lfc x' \rightarrow H(R(y)) \Rfc \coma   \Lb x'/x \Rb s , n  \rangle }
%%      \Rtab \mbox{(E-LetDref)}
%% $$
%% % Malloc (allocate) Command
%% $$
%%      \frac{h \notin dom(H)}
%%             {\langle H\coma R\coma  \LET x = \Malloc() \; \IN s , n \rangle
%%               \xlongrightarrow{\Malloc}_{D}
%%               \langle H \Lfc h \rightarrow v\Rfc \coma R\Lfc x' \rightarrow h \Rfc \coma   \Lb x'/x \Rb s , n-1  \rangle }
%% \Rtab \mbox{(E-Malloc)}
%% $$
%% % IFNULL T
%% $$
%%     \frac{R(x) = \NULL}
%%            {\langle H \coma R \coma \IFNULL\Cirx   \THEN   s_{1} \ELSE\  s_{2} \coma  n \rangle
%%            \longrightarrow_{D}
%%            \langle H\coma R\coma s_{1} \coma n \rangle}
%%     \Rtab \mbox{(E-IfNullT)}
%% $$
%% % IFNULL F
%% $$
%%     \frac{R(x) \neq \NULL}
%%            {\langle H \coma R \coma \IFNULL\Cirx \THEN  s_{1} \ELSE  s_{2} \coma  n \rangle
%%            \longrightarrow_{D}
%%            \langle H\coma R\coma s_{2} \coma  n, \rangle}
%%     \Rtab \mbox{(E-IfNullF)}
%% $$
%% % Function Call
%% $$
%%      \frac{f(\vec{y}) = s \in D}
%%             { \langle H\coma R\coma  f(\vec{x}) , n \rangle
%%                \longrightarrow_{D}
%%                \langle H\coma R\coma  \Lb \vec{x}/\vec{y} \Rb s , n \rangle}
%%       \Rtab \mbox{(E-Call)}
%% $$
%% % Error : access the null memory cell
%% $$
%%       \frac{R(x) = null}
%%             {\langle H\coma R\coma  *x \leftarrow y , n \rangle
%%               \longrightarrow_{D}
%%              \bf NullEx }
%%       \Rtab \mbox{(E-AssignNullError)}
%% $$
%% % ERROR : access the null memory cell
%% $$
%%       \frac{R(y) = null}
%%              {\langle H\coma R\coma  x = *y, n \rangle
%%                \longrightarrow_{D}
%%               \bf NullEx }
%%              \Rtab \mbox{(E-DrefNullError)}
%% $$
%% $$
%%      \frac{R(x) =  null }
%%            {\langle H\coma R\coma  \FREE , n \rangle
%%              \xlongrightarrow{\Free}_{D} \bf NullEx  }
%%       \Rtab \mbox{(E-FreeNullError)}
%% $$
%% % ERROR :
%% $$
%%      \frac{R(x) \notin dom(H) \cup \Lfc null \Rfc}
%%            {\langle H\coma R\coma   *x \leftarrow y,  n \rangle
%%              \longrightarrow_{D}
%%            \bf  Error }
%%     \Rtab \mbox{(E-AssignError)}
%% $$
%% % ERROR
%% $$
%%       \frac{R(y) \notin dom(H) \cup \Lfc null \Rfc}
%%            {\langle H\coma R\coma  \LET x  = *y \; \IN s, n \rangle
%%               \longrightarrow_{D}
%%                 \bf  Error }
%%       \Rtab \mbox{(E-DrefError)}
%% $$
%% %
%% $$
%%       \frac{R(x) \notin dom(H) \cup \Lfc null \Rfc}
%%             {\langle H\coma R\coma  \FREE , n \rangle
%%               \xlongrightarrow{\Free}_{D}
%%               \bf Error }
%%      \Rtab \mbox{(E-FreeError)}
%% $$
%%  % ERROR: no enough space
%% $$
%%       \langle H\coma R\coma \LET x = \Malloc() \ \IN s ,  0  \rangle
%%       \xlongrightarrow{\Malloc}_{D}
%%       \mathbf{Error}
%%       \Rtab \mbox{(E-MallocError)}
%% $$
%% $$
%%      \mathbf{Figure \; 3.} \;\;  \mbox{ Operational Semantics}
%% $$
%
%\caption{Operational Semantics.}
%\label{example:os}
     %\end{figure}

As mentioned in Section~\ref{sec:introduction}, we define a memory
leak to be consuming unbounded number of memory cells.  Formally, this
is defined as follows.

\begin{myDef}[Memory leaks]
\label{df:ml}
%% memory leaks:\\
%% if $\vdash \langle D, s \rangle : n$, then $\langle \emptyset, \emptyset, s, n \rangle \xlongrightarrow{\rho_{1}, \dots, \rho_{n}}_{D} \langle H, R, s', 0 \rangle \xlongrightarrow{\Malloc}_{D} Overflow $\\
%% memory-leak \ freedom: $\exists n \in \mathbb{N}$ s.t. $\langle \emptyset, \emptyset, s, n \rangle \nrightarrow^{*}Overflow$
A configuration \(\langle H, R, s, n \rangle\) \emph{goes overflow} if
there is \(\sigma\) such that \(\langle H, R, s, n \rangle
\xLongrightarrow{\sigma} \OVERFLOW\).  A program \(\langle D, s
\rangle\) \emph{requires more than \(n\) cells} if \(\langle
\emptyset, \emptyset, s, n \rangle\) goes overflow.  A program
\(\langle D, s \rangle\) is \emph{totally memory-leak free} if there
is a natural number \(n\) such that it does not require more than
\(n\) cells.
\end{myDef}
