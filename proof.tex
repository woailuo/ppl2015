\section{Proof of Lemmas}
\label{sec:proof}

\begin{lemma}
\label{lem:okPreserved}
If \(\OK_n(P)\) and \(P \xlongrightarrow{\rho} P'\), then
\begin{itemize}
\item \(\OK_{n-1}(P')\) if \(\rho = \Malloc\),
\item \(\OK_{n+1}(P')\) if \(\rho = \Free\), and
\item \(\OK_n(P')\) if \(\rho = \tau\).
\end{itemize}
\end{lemma}
\begin{proof}

Case analysis on \(P \xlongrightarrow{\rho} P'\).

\begin{itemize}
\item Case $P = \TSKIP;P'$, \Rtab \(\TSKIP;P' \xlongrightarrow{\tau} P'\)

  We need to prove \(\OK_n(P')\).   We suppose that
  \(\OK_n(P')\) does not hold. Then, we have \(\TSKIP;P'
  \xlongrightarrow{\tau} P' \xlongrightarrow{\exists \rho}
  Q\), \(s.t.\) \(\sharp_{m}(\rho) - \sharp_{f}(\rho) > n\).

  From the definition of \(\OK_n(P)\), we have \(\sharp_m(\tau \cdot
  \rho) - \sharp_f(\tau \cdot \rho) \le n \). Then, we get
  \(\sharp_m(\tau) + \sharp_m(\rho) - \sharp_f(\tau) - \sharp_f(\rho)
  \le n\). From the definition of \(\sharp_\rho(\sigma)\), we have
  \(\sharp_m(\rho) - \sharp_f(\rho) \le n\). Therefore, we get the
  contradiction.

\item Case $P = \Malloc$, \Rtab \(\Malloc \xlongrightarrow{\Malloc} \TSKIP\)

%We need to prove \(\OK_{n-1}()
  
We need to prove \(\sharp_{m}(P') -
\sharp_{f}(P') \le n'\), where \(n'\) is \(n - 1\) and \(P'\) is \(\bf
0\). That is, we need to prove \( 1 \le n \).

 We have\(\ OK_{n}(\Malloc)\). From the definition of \(\OK_n(P)\),
 we have \(\sharp_{m}(\Malloc) - \sharp_{f}(\Malloc) \le
 n\). From the definition of \(\sharp_{\rho}(\sigma)\), we have
 \( 1 \le n\).

\item Case $P = \Free$

From the rule \rn{Sem-Free}, we need to prove \(\sharp_{m}(P') -
\sharp_{f}(P') \le n'\) where \(n'\) is \(n + 1\) and \(P'\) is \(\bf
0\). That is, we need to prove \( 0 \le n + 1\).

We have \(\OK_{n}(\Free)\). From the definition of \(\OK_n(P)\), we
have \(\sharp_{m}(\Free) - \sharp_{f}(\Free) \le n\). From the
definition of \(\sharp_{\rho}(\sigma)\), we have \( \ 0 \le n+1\).

\item Case $P = P_{1} + P_{2}$, \Rtab \(P_1 + P_2 \xlongrightarrow{\tau} P_1\)

  We prove \(\OK_n(P_1)\) by contradiction.  Assuming that
  \(OK_{n}(P_1)\) does not hold.  Then, we have \(P_{1} + P_{2}
  \xlongrightarrow{\tau} P_{1} \xlongrightarrow{\exists \rho} Q\),
  \(s.t.\) \(\sharp_{m}(\rho) - \sharp_{f}(\rho) > n\).

  From the definition of \(\OK_n(P)\),
  we have \(\sharp_m(\tau \cdot \rho) -\sharp_f(\tau \cdot \rho) \le
  n\). Then, we have \(\sharp_m(\tau) + \sharp_m(\rho) -
  \sharp_f(\tau) -\sharp_f(\rho) \le n\).  By the definition of
  \(\sharp_\rho(\sigma)\),  we get \(\sharp_m(\rho) -\sharp_f(\rho)
  \le n\). Therefore, we get the contradiction.

\item Case $P = P_{1} + P_{2}$, \Rtab \(P_1 + P_2 \xlongrightarrow{\tau} P_2\)

 The same to above.

\item Case $P = P_{1};P_{2}$

  We need to prove \(\OK_n(P_1)\).  To prove it by contradiction.
  Assuming that \(OK_{n'}(P_{1}';P_{2})\) does not hold. Then, we have
  \(P_{1};P_{2} \xlongrightarrow{\rho} P_{1}';P_{2}
  \xlongrightarrow{\exists \rho_1} Q\), \(s.t.\) \(\sharp_{m}(\rho_1)
  - \sharp_{f}(\rho_1) > n'\).

  From the definition of \(\OK_n(P)\), we have \( \sharp_{m}(\rho
  \cdot \rho_1) - \sharp_{f}(\rho \cdot \rho_1) \le n\).  We also have
  \(\sharp_{m}(\rho) + \sharp_{m}(\rho_1) - \sharp_{f}(\rho)
  -\sharp_{f}(\rho_1) \le n\).  From the assumption
  \(\sharp_{m}(\rho_1) - \sharp_{f}(\rho_1) > n'\), we have \infax{ n'
    + \sharp_m(\rho) - \sharp_f(\rho) < \sharp_{m}(\rho) +
    \sharp_{m}(\rho_1) - \sharp_{f}(\rho) -\sharp_{f}(\rho_1) \le n}

We have
$$
   n'=\left\{
   \begin{aligned}
     n + 1, && \rho = \Free \\
     n - 1,  && \rho = \Malloc  \\
     n ,      && otherwise
   \end{aligned}
   \right.
$$

   Hence, we have: if \(\rho = \Free\), then \(n + 1 - 1 < n\) from
   the definition of \(\sharp_{\rho}(\sigma)\); if \(\rho = \Malloc\),
   then \( n - 1 + 1 < n \); if \( \rho = other\), then \( n < n
   \). All of the three cases have \(n < n\). Therefore, we get the
   contradiction.

 \item Case  \(P = \mu\alpha.P'\),\Rtab  \( \mu\alpha.P' \xlongrightarrow{\tau} [\mu\alpha.P'\slash\alpha]P'\)

   We need to prove \(\OK_n([\mu\alpha.P'\slash\alpha]P')\).

   Considering the position of type variable \(\alpha\) in \(P'\), we need prove the following subcases:
$$
   P'=\left\{
   \begin{aligned}
     &P_1;\alpha,& \\
     &P_1;\alpha;P_2, and&  \\
     &\alpha;P_1&
   \end{aligned}
   \right.
$$

   \begin{itemize}
   \item Subcase \(P = \mu\alpha.P_1;\alpha\),\Rtab \(
     \mu\alpha.P_1;\alpha \xlongrightarrow{\tau} P_1;(\mu\alpha.P_1;\alpha) \)
     
     We need to prove \(\OK_n(P_1;(\mu\alpha.P_1;\alpha) \).

     By contradiction, we suppose that \( \mu\alpha.P_1;\alpha
     \xlongrightarrow{\tau} P_1;(\mu\alpha.P_1;\alpha)
     \xlongrightarrow{\exists \rho} Q s.t.  \sharp_m(\rho) -
     \sharp_f(\rho) > n\).

     We have \(\OK_n(P)\)).  By the definition of \(\OK_n(P)\), we
     have \(\sharp_m(\tau \cdot \rho) -\sharp_f(\tau \cdot \rho) \le
     n\). Hence, we have \(\sharp_m(\tau) + \sharp_m(\rho) -
     \sharp_f(\tau) -\sharp_f(\rho) \le n\).  From the definition of
     \(\sharp_\rho(\sigma)\), we have \(\sharp_m(\rho) -\sharp_f(\rho)
  \le n\). Therefore, we get the contradiction.

\item Subcase  \(P = \mu\alpha.P_1;\alpha;P_2\),\Rtab \(\mu\alpha.P_1;\alpha;P_2 \xlongrightarrow{\tau} P_1;(\mu\alpha.P_1;\alpha;P_2);P_2 \).

  We need to prove  \(\OK_n(P_1;(\mu\alpha.P_1;\alpha;P_2);P_2) \).

  By contradiction, the proof is similar to the above subcase.

     \item Subcase \(P = \mu\alpha.\alpha;P_1\),\Rtab
       \(\mu\alpha.\alpha;P_1 \xlongrightarrow{\tau}
       (\mu\alpha.\alpha;P_1);P_1 \).

       We need to prove \(\OK_n((\mu\alpha.\alpha;P_1);P_1)\).

       We can observe that this subcase is of the form like
       \(P\xlongrightarrow{\tau}P'\xlongrightarrow{\tau}P''\xlongrightarrow{\tau}
       \cdots\). Clearly, \(\OK_n(P')\) holds.

     \end{itemize}

\end{itemize}
\end{proof}

\begin{pfof}{Lemma~\ref{lem:preservation}}
By induction on the derivation of evaluation rules.\\

\begin{itemize}
\item Case: $\langle H, R, \FREE, n \rangle \xlongrightarrow{\Free}
  \langle H', R', \SKIP, n + 1 \rangle $.

We have \(\OK_n(P)\) and \(\Theta; \Gamma \vdash \Free(x) \COL P\).
From inversion of the typing rules, we have \(\Theta; \Gamma \vdash
\Free(x) \COL \Free\) and \(\Free \le P\) for some \(P'\).  Hence,
from the definition of subtyping, we have \(\TSKIP \le P''\) and \(P
\xLongrightarrow{\Free} P''\) for some \(P''\).

We need to find \(P_1\) such that \(P \xLongrightarrow{\Free} P_1\),
\(\Theta; \Gamma \vdash \SKIP \COL P_1\), and \(\OK_{n+1}(P_1)\).
Take \(P''\) as \(P_1\).  Then, \(P \xLongrightarrow{\Free} P''\) as
we stated above.  We also have \(\Theta; \Gamma \vdash \SKIP \COL
P''\) from \rn{T-Skip}, \(\TSKIP \le P''\), and \rn{T-Sub}.
\(\OK_{n+1}(P'')\) follows from Lemma~\ref{lem:okPreserved}.


\item Case: $\langle H, R, \LET x = \MALLOC \IN s, n \rangle
  \xlongrightarrow{\Malloc} \langle H', R', [x'/x]s, n - 1 \rangle
  $.

  From the assumption, we have \(\Theta; \Gamma \vdash \LET x =
  \MALLOC \IN s \COL P\) and \(\OK_{n}(P)\). By the inversion of
  typing rules, we have \(\Malloc;P_1 \le P\) and \(\Theta; \Gamma
  \vdash s : P_{1}\) for some \(P_1\). We have the following
  derivation: \infrule{ \Malloc \xlongrightarrow{\Malloc} 0}
  {\Malloc;P_1 \xlongrightarrow{\Malloc} 0;P_{1}} and \(0;P_1
  \rightarrow P_{1}\), then we have \(\Malloc;P_{1}
  \xLongrightarrow{\Malloc} P_{1}\). Hence, By the definition of
  subtyping, we have \(P \xLongrightarrow{\Malloc}P''\) and \(P_{1}
  \le P''\) for some \(P''\)

  We need to find \(P'\) such that \(\Theta; \Gamma' \vdash
  [x'/x]s\COL P'\),\(P\xLongrightarrow{\Malloc} P'\). Take \(P''\) as
  \(P'\). Then \(P \xLongrightarrow{\Malloc} P''\) as we state above. We
  also have \(\Theta;\Gamma \vdash [x'/x]s \COL P''\) from \rn{T-Sub},
  \(\Theta;\Gamma \vdash [x'/x]s \COL P_1\) and \(P_1 \le
  P''\). \(\OK_{n-1}(P'')\) follows from Lemma~\ref{lem:okPreserved}.

\item Case: $\langle H, R, \SKIP;s, n \rangle \rightarrow \langle
  H, R, s, n \rangle $.

  we have \(\Theta;\Gamma \vdash \SKIP;s\COL  P\) and
  \(\OK_{n}(P)\). From the inversion of the typing rules, we have
  \(\Theta; \Gamma \vdash s\COL P_{1}\) and \(0;P_{1} \le
  P\). Hence, from the definition of subtyping, we have \(P
  \xLongrightarrow{\tau} P''\) and \(P_{1} \le P''\) for some \(P''\).

  We need to find \(P'\) such that \(\Theta; \Gamma \vdash s : P'\)
  and \(P \xLongrightarrow{\tau} P'\). Take \(P''\) as \(P'\). Then \(P
  \xLongrightarrow{\tau} P''\) as we stated above. We also have
  \(\Theta;\Gamma \vdash s\COL P''\) from \rn{T-Sub}, \(\Gamma \vdash
  s\COL P_{1}\) and \(P_{1} \le P''\). \(\OK_n(P'')\) follows from
  Lemma~\ref{lem:okPreserved}

\item Case: $\langle H, R, *x \leftarrow y , n \rangle \rightarrow
  \langle H', R', \SKIP, n \rangle $.

  We have \(\Theta; \Gamma \vdash *x \leftarrow y : P\) and
  \(\OK_{n}(P)\). From the inversion of typing rules, we have \(0 \le
  P\).

  We need to find $P'$ such that \(\Theta; \Gamma \vdash \SKIP: P'\),
  \(P \xLongrightarrow{\tau} P'\) and \(\OK_n(P')\). Take $P$ as $P'$. Then,
  \(P \xLongrightarrow{\tau} P'\) and \(\OK_n(P')\) hold. We also have \(\Theta;
  \Gamma \vdash \SKIP: P'\) from \rn{T-Skip}, \(0 \le P\) and
  \rn{T-Sub}.

\item Case: $\langle H, R, \LET x = y\ \IN s , n \rangle
  \rightarrow \langle H', R', [x'/x]s, n \rangle $.

  We have \(\Theta; \Gamma \vdash \LET x = y\ \IN \ s \COL P\) and
  \(OK_{n}(P)\). From the inversion of typing rules, we have \(\Theta;
  \Gamma \vdash s\COL P_{1}\) and \(P_{1} \le P\).

  We need to find $P'$ such that \(\Theta; \Gamma \vdash [x'/x]s : P'\) ,
  \(P \xLongrightarrow{\tau} P'\) and \(\OK_n(P'\)). Take \(P\) as
  \(P'\). Then \( P \xLongrightarrow{\tau} P'\) and \(\OK_n(P')\) hold.  We
  also have \(\Theta; \Gamma \vdash [x'/x]s\COL P\) from \rn{T-Sub},
  \(\Theta; \Gamma \vdash [x'/x]s\COL P_{1}\) and \( P_{1} \le
  P\).

\item Case: $\langle H, R, \LET x = \NULL \ \IN \ s, n \rangle
  \rightarrow \langle H', R', [x'/x]s, n \rangle $

  We have \(\Theta; \Gamma \vdash \LET x = \NULL \ \IN \ s\COL P\)
  and \(OK_{n}(P)\). From the inversion of typing rules, we have
  \(\Theta; \Gamma \vdash s\COL P_{1}\) and \( P_{1} \le P\).

  We need to find $P'$ such that \(\Theta; \Gamma' \vdash [x'/x]s\COL P'\),
  \(P \xLongrightarrow{\tau} P'\) and \(\OK_n(P')\).  Take \(P\) as
  \(P'\).  Then, \(P \xLongrightarrow{\tau} P'\) and \(\OK_n(P')\) hold.  We
  also have \(\Theta; \Gamma \vdash [x'/x]s\COL P\) from \rn{T-Sub},
  \(\Theta; \Gamma \vdash [x'/x]s\COL P_{1}\)\( P_{1} \le
  P\). 

\item Case: $\langle H, R, \LET x = *y \ \IN \ s, n \rangle
  \rightarrow \langle H', R', [x'/x]s, n \rangle $

  We have \(\Theta; \Gamma \vdash \LET x = *y \ \IN \ s\COL  P\) and
  \(OK_{n}(P)\). From the inversion of typing rules, we have \(\Theta;
  \Gamma \vdash s\COL P_{1}\) and \(P_{1} \le P\).

  We need to find \(P'\) such that \(\Theta; \Gamma \vdash [x'/x]s\COL
  P'\), \(P \xLongrightarrow{\tau} P'\) and \(\OK_n(P')\). Take \(P\) as
  \(P'\). Then, \(P \xLongrightarrow{\tau} P'\) and \(\OK_n(P')\) hold.  We
  also have \(\Theta; \Gamma' \vdash [x'/x]s\COL P\) from \rn{T-Sub},
  \(\Theta; \Gamma' \vdash [x'/x]s\COL P_{1}\) and \(P_{1} \le P\).
        
\item Case(\rn{Tr-IfNullT}): \(\langle H, R, \IFNULL \Cirx \ \THEN s_{1} \ \ELSE \ s_{2},
  n \rangle \rightarrow \langle H, R, s_{1}, n \rangle\)

  We have \(\Theta; \Gamma \vdash \IFNULL \Cirx \ \THEN \ s_{1}
  \ \ELSE \ s_{2}\COL P\) and \(\OK_{n}(P)\). From the inversion of
  typing rules, we have \(\Theta; \Gamma \vdash s_{1} : P_{1}\) and \(P_{1}
  \le P\).

  We need to find $P'$ such that \(\Theta; \Gamma \vdash s_1\COL P'\)
  , \(P \xLongrightarrow{\tau} P'\) and \(\OK_n(P'\)).  Take P as P'.  Then,
  \(\OK_n(P')\) and \(P \xLongrightarrow{\tau} P'\) hold. We also have \(\Theta;
  \Gamma' \vdash s_{1}\COL P\) from \(\Theta; \Gamma \vdash s_{1}\COL
  P_1\), \(P_{1} \le P\) and \rn{T-Sub}.

\item Case(\rn{Tr-IfNullF}): \(\langle H, R, \IFNULL \Cirx \ \THEN s_{1} \ \ELSE \ s_{2},
  n \rangle \rightarrow \langle H, R, s_{2}, n \rangle\).

 The proof is similar to case \rn{Tr-IfNullT}. 

\item Case: $\langle H, R, f(\vec{x}) , n \rangle \rightarrow  \langle H, R, [\vec{x}/\vec{y}]s, n  \rangle $

  Assuming that \(D(f) = s\) and \(\Theta(f) = P_1\), we have \(s\COL
  P_1\). The \([\vec{x}/\vec{y}]s\) also has a type \(P_1\).

  From the assumption, we have \(\Theta;\Gamma \vdash f(x)\COL P\) and
  \(\OK_n(P)\). From the inversion, we have \(\Theta;\Gamma \vdash
  f(x)\COL P_1\) and \(P_1 \le P\).

  We need to find \(P'\) such that \(\Theta;\Gamma \vdash [\vec{x}/\vec{y}]s\COL
  P'\), \(P \xLongrightarrow{\tau} P'\) and \(\OK_n(P')\). Take \(P\) as
  \(P'\). Then \(P \xLongrightarrow{\tau} P'\) and \(\OK_n(P')\) hold. We
  also have \(\Theta;\Gamma \vdash [\vec{x}/\vec{y}]s\COL P \) from
  \(\Theta;\Gamma \vdash [\vec{x}/\vec{y}]s\COL P_1\), \(P_1 \le P\) and
  \rn{T-Sub}.

\end{itemize}

\end{pfof}
