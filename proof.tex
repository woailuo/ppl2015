\section{Proof of Lemmas}
\label{sec:proof}

\begin{lemma}
\label{lem:okPreserved}
If \(\OK_n(P)\) and \(P \xlongrightarrow{\rho} P'\), then
\begin{itemize}
\item \(\OK_{n-1}(P')\) if \(\rho = \Malloc\),
\item \(\OK_{n+1}(P')\) if \(\rho = \Free\), and
\item \(\OK_n(P')\) if \(\rho = \tau\).
\end{itemize}
\end{lemma}
\begin{proof}

Case analysis on \(P \xlongrightarrow{\rho} P'\).

\begin{itemize}
\item Case $P = \TSKIP;P'$ and \(\TSKIP;P' \xlongrightarrow{\tau} P'\)

  We need to prove \(\OK_n(P')\).  Assume that \(\OK_n(P')\) does not
  hold. Then, we have \(\TSKIP;P' \xlongrightarrow{\tau} P'
  \xlongrightarrow{\sigma} Q\) such that \(\sharp_{m}(\sigma) -
  \sharp_{f}(\sigma) > n\) for some \(\sigma\).  From the definition
  of \(\OK_n(P)\), we have \(\sharp_m(\tau\sigma) - \sharp_f(\tau
  \sigma) \le n \).  From the definition of \(\sharp_\rho(\sigma)\),
  we get \(\sharp_m(\tau) + \sharp_m(\rho) - \sharp_f(\tau) -
  \sharp_f(\rho) \le n\) that is equivalent to \(\sharp_m(\isgma) -
  \sharp_f(\sigma) \le n\), which is contradiction.

\item Case $P = \Malloc$ and \(\Malloc
  \xlongrightarrow{\Malloc} \TSKIP\)

  We need to prove \(\OK_{n-1}(\TSKIP)\).  From the definition of
  \(\OK_n(P)\), we have that \(\OK_{n-1}(\TSKIP)\) holds if, for any
  \(P'\), if \(\TSKIP \xlongrightarrow{\sigma} P'\) then
  \(\sharp_m(\sigma) - \sharp_f(\sigma) \le n - 1\). There is no such \(P'\) s.t. \(\TSKIP
  \xlongrightarrow{\sigma} P'\), therefore, \(\OK_{n-1}(\TSKIP)\) holds.

  \item Case $P = \Free$ and \(\Free \xlongrightarrow{\Free} \TSKIP\).

    We need to prove \(\OK_{n+1}(\TSKIP)\).  From the definition of
  \(\OK_n(P)\), we have that \(\OK_{n+1}(\TSKIP)\) holds if, for any
  \(P'\), if \(\TSKIP \xlongrightarrow{\sigma} P'\) then
  \(\sharp_m(\sigma) - \sharp_f(\sigma) \le n + 1\). There is no such \(P'\) s.t. \(\TSKIP
  \xlongrightarrow{\sigma} P'\), therefore, \(\OK_{n+1}(\TSKIP)\) holds.

\item Case $P = P_{1} + P_{2}$ and \(P_1 + P_2 \xlongrightarrow{\tau} P_1\)

  We prove \(\OK_n(P_1)\) by contradiction.  Assume that
  \(OK_{n}(P_1)\) does not hold.  Then, there exists \(\sigma\) such
  that \(P_{1} + P_{2} \xlongrightarrow{\tau} P_{1}
  \xlongrightarrow{\sigma} Q\) and \(\sharp_{m}(\sigma) -
  \sharp_{f}(\sigma) > n\).  From the definition of \(\OK_n(P)\), we
  have \(\sharp_m(\tau\sigma) -\sharp_f(\tau\sigma) \le n\). Then, we
  have \(\sharp_m(\tau) + \sharp_m(\rho) - \sharp_f(\tau)
  -\sharp_f(\rho) \le n\).  By the definition of
  \(\sharp_\rho(\sigma)\), we get \(\sharp_m(\sigma) -\sharp_f(\sigma)
  \le n\). Therefore, we get the contradiction.

\item Case $P = P_{1} + P_{2}$ and \(P_1 + P_2 \xlongrightarrow{\tau} P_2\)

  Similar to the case of \(P_1 + P_2 \xlongrightarrow{\tau} P_1\).

\item Case $P = P_{1};P_{2}$ and \(P_1;P_2 \xlongrightarrow{\rho} P_1';P_2\).

We need to prove \(\OK_{n'}(P_1';P_2)\) where \(n'\) is determined by
\[
   n'=\left\{
   \begin{array}{ll}
     n + 1 & \rho = \Free\\
     n - 1 & \rho = \Malloc\\
     n & \mbox{Otherwise.}
   \end{array}
   \right.
\]
Assume that \(OK_{n'}(P_{1}';P_{2})\) does not hold. Then, there
exists \(\sigma\) such that \(P_{1};P_{2} \xlongrightarrow{\rho}
P_{1}';P_{2} \xlongrightarrow{\sigma} Q\) and \(\sharp_{m}(\sigma) -
\sharp_{f}(\sigma) > n'\).  From the definition of \(\OK_n(P)\), we
have \( \sharp_{m}(\rho\sigma) - \sharp_{f}(\rho\sigma) \le n\); hence
we have \(\sharp_{m}(\rho) + \sharp_{m}(\sigma) - \sharp_{f}(\rho)
-\sharp_{f}(\sigma) \le n\).  From the assumption \(\sharp_{m}(\sigma)
- \sharp_{f}(\sigma) > n'\), we have 
\[
n' + \sharp_m(\rho) - \sharp_f(\rho) < \sharp_{m}(\rho) +
\sharp_{m}(\rho_1) - \sharp_{f}(\rho) -\sharp_{f}(\rho_1) \le n.
\]
%% If \(\rho = \Free\), then \(n + 1 - 1 < n\); if \(\rho = \Malloc\),
%% then \( n - 1 + 1 < n \); if \( \rho = other\), then \( n < n
%% \). Therefore, we have \(n < n\), and then we get the contradiction.
The leftmost term \(n' + \sharp_m(\rho) - \sharp_f(\rho)\) is equal to
\(n\) for any \(\rho\).  Hence, we get contradiction.

\item Case \(P = \mu\alpha.P'\) and \(\mu\alpha.P'
  \xlongrightarrow{\tau} [\mu\alpha.P'\slash\alpha]P'\)

We need to prove \(\OK_n([\mu\alpha.P' / \alpha]P')\).  Suppose not.
Then, there exists \(\sigma\) and \(Q\) such that \([\mu\alpha.P' /
  \alpha]P' \xlongrightarrow{\sigma} Q\) and \(\sharp_m(\sigma) -
\sharp_f(\sigma) > n\).  However, this contradicts to \(P
\xlongrightarrow{\tau\sigma} Q\) and \(\OK_n(P)\) as follows.
\[
n \ge \sharp_m(\tau\sigma) - \sharp_f(\tau\sigma) = \sharp_m(\sigma) - \sharp_f(\sigma) > n.
\]

%%    We need to prove \(\OK_n([\mu\alpha.P'\slash\alpha]P')\).

%%    Considering the position of type variable \(\alpha\) in \(P'\), we need prove the following subcases:
%% $$
%%    P'=\left\{
%%    \begin{aligned}
%%      &P_1;\alpha,& \\
%%      &P_1;\alpha;P_2, and&  \\
%%      &\alpha;P_1&
%%    \end{aligned}
%%    \right.
%% $$

%%    \begin{itemize}
%%    \item Subcase \(P = \mu\alpha.P_1;\alpha\),\Rtab \(
%%      \mu\alpha.P_1;\alpha \xlongrightarrow{\tau} P_1;(\mu\alpha.P_1;\alpha) \)
     
%%      We prove \(\OK_n(P_1;(\mu\alpha.P_1;\alpha) \) by
%%      contradiction. Supposing that \(\OK_n(P_1;(\mu\alpha.P_1;\alpha)
%%      \) does not hold. Then, we have \( \mu\alpha.P_1;\alpha
%%      \xlongrightarrow{\tau} P_1;(\mu\alpha.P_1;\alpha)
%%      \xlongrightarrow{\exists \rho}Q\) s.t. \(\sharp_m(\rho) -
%%      \sharp_f(\rho) > n\).

%%      We have \(\OK_n(P)\).  By the definition of \(\OK_n(P)\), we
%%      have \(\sharp_m(\tau \cdot \rho) -\sharp_f(\tau \cdot \rho) \le
%%      n\). Hence, we have \(\sharp_m(\tau) + \sharp_m(\rho) -
%%      \sharp_f(\tau) -\sharp_f(\rho) \le n\).  From the definition of
%%      \(\sharp_\rho(\sigma)\), we have \(\sharp_m(\rho) -\sharp_f(\rho)
%%   \le n\). Therefore, we get the contradiction.

%% \item Subcase \(P = \mu\alpha.P_1;\alpha;P_2\),\Rtab
%%   \(\mu\alpha.P_1;\alpha;P_2 \xlongrightarrow{\tau}
%%   P_1;(\mu\alpha.P_1;\alpha;P_2);P_2 \).

%%   We need to prove  \(\OK_n(P_1;(\mu\alpha.P_1;\alpha;P_2);P_2) \).

%%   By contradiction, the proof is similar to the above subcase.

%%      \item Subcase \(P = \mu\alpha.\alpha;P_1\),\Rtab
%%        \(\mu\alpha.\alpha;P_1 \xlongrightarrow{\tau}
%%        (\mu\alpha.\alpha;P_1);P_1 \).

%%        We need to prove \(\OK_n((\mu\alpha.\alpha;P_1);P_1)\).

%%        We can observe that this subcase is of the form like
%%        \(P\xlongrightarrow{\tau}P'\xlongrightarrow{\tau}P''\xlongrightarrow{\tau}
%%        \cdots\). Clearly, \(\OK_n(P')\) holds.

%%      \end{itemize}

\end{itemize}
\end{proof}

\begin{pfof}{Lemma~\ref{lem:preservation}}
By induction on the derivation of evaluation rules.

\paragraph{Notation}
 The type of \([x'\slash x]s\) is same as \(s\), because our interest
 is the behavior of a program, a variable type environment does not
 carry information on the types of variables. For example, if
 \(\Theta;\Gamma \vdash s\COL P\), then \(\Theta;\Gamma \vdash
         [x'\slash x]s\COL P\).

\begin{itemize}
\item Case: $\langle H, R, \FREE, n \rangle \xlongrightarrow{\Free}
  \langle H', R', \SKIP, n + 1 \rangle $.

We have \(\OK_n(P)\) and \(\Theta; \Gamma \vdash \Free(x) \COL P\).
From inversion of the typing rules, we have \(\Theta; \Gamma \vdash
\Free(x) \COL \Free\) and \(\Free \le P\).  Hence, from the definition
of subtyping, we have \(\TSKIP \le P''\) and \(P
\xLongrightarrow{\Free} P''\) for some \(P''\).

We need to find \(P'\) such that \(P \xLongrightarrow{\Free} P'\),
\(\Theta; \Gamma \vdash \SKIP \COL P'\), and \(\OK_{n+1}(P')\).
Take \(P''\) as \(P'\).  Then, \(P \xLongrightarrow{\Free} P''\) as
we stated above.  We also have \(\Theta; \Gamma \vdash \SKIP \COL
P''\) from \rn{T-Skip}, \(\TSKIP \le P''\), and \rn{T-Sub}.
\(\OK_{n+1}(P'')\) follows from Lemma~\ref{lem:okPreserved}.


\item Case: $\langle H, R, \LET x = \MALLOC \IN s, n \rangle
  \xlongrightarrow{\Malloc} \langle H', R', [x'/x]s, n - 1 \rangle
  $.

  From the assumption, we have \(\Theta; \Gamma \vdash \LET x =
  \MALLOC \IN s \COL P\) and \(\OK_{n}(P)\). By the inversion of
  typing rules, we have \(\Malloc;P_1 \le P\) and \(\Theta; \Gamma
  \vdash s : P_{1}\) for some \(P_1\). We have the following
  derivation: \infrule{ \Malloc \xlongrightarrow{\Malloc} 0}
  {\Malloc;P_1 \xlongrightarrow{\Malloc} 0;P_{1}} and \(0;P_1
  \rightarrow P_{1}\), then we have \(\Malloc;P_{1}
  \xLongrightarrow{\Malloc} P_{1}\). Hence, By the definition of
  subtyping, we have \(P \xLongrightarrow{\Malloc}P''\) and \(P_{1}
  \le P''\) for some \(P''\).

  We need to find \(P'\) such that \(\Theta; \Gamma' \vdash
  [x'/x]s\COL P'\),\(P\xLongrightarrow{\Malloc} P'\). Take \(P''\) as
  \(P'\). Then \(P \xLongrightarrow{\Malloc} P''\) as we state above. We
  also have \(\Theta;\Gamma \vdash [x'/x]s \COL P''\) from \rn{T-Sub},
  \(\Theta;\Gamma \vdash [x'/x]s \COL P_1\) and \(P_1 \le
  P''\). \(\OK_{n-1}(P'')\) follows from Lemma~\ref{lem:okPreserved}.

\item Case: $\langle H, R, \SKIP;s, n \rangle \rightarrow \langle
  H, R, s, n \rangle $.

  we have \(\Theta;\Gamma \vdash \SKIP;s\COL  P\) and
  \(\OK_{n}(P)\). From the inversion of the typing rules, we have
  \(\Theta; \Gamma \vdash s\COL P_{1}\) and \(0;P_{1} \le
  P\). Hence, from the definition of subtyping, we have \(P
  \xLongrightarrow{\tau} P''\) and \(P_{1} \le P''\) for some \(P''\).

  We need to find \(P'\) such that \(\Theta; \Gamma \vdash s : P'\)
  and \(P \xLongrightarrow{\tau} P'\). Take \(P''\) as \(P'\). Then \(P
  \xLongrightarrow{\tau} P''\) as we stated above. We also have
  \(\Theta;\Gamma \vdash s\COL P''\) from \rn{T-Sub}, \(\Gamma \vdash
  s\COL P_{1}\) and \(P_{1} \le P''\). \(\OK_n(P'')\) follows from
  Lemma~\ref{lem:okPreserved}

\item Case: $\langle H, R, *x \leftarrow y , n \rangle \rightarrow
  \langle H', R', \SKIP, n \rangle $.

  We have \(\Theta; \Gamma \vdash *x \leftarrow y : P\) and
  \(\OK_{n}(P)\). From the inversion of typing rules, we have \(0 \le
  P\).

  We need to find $P'$ such that \(\Theta; \Gamma \vdash \SKIP: P'\),
  \(P \xLongrightarrow{\tau} P'\) and \(\OK_n(P')\). Take $P$ as $P'$. Then,
  \(P \xLongrightarrow{\tau} P'\) and \(\OK_n(P')\) hold. We also have \(\Theta;
  \Gamma \vdash \SKIP: P'\) from \rn{T-Skip}, \(0 \le P\) and
  \rn{T-Sub}.

\item Case: $\langle H, R, \LET x = y\ \IN s , n \rangle
  \rightarrow \langle H', R', [x'/x]s, n \rangle $.

  We have \(\Theta; \Gamma \vdash \LET x = y \IN s \COL P\) and
  \(OK_{n}(P)\). From the inversion of typing rules, we have \(\Theta;
  \Gamma \vdash s\COL P_{1}\) and \(P_{1} \le P\).

  We need to find $P'$ such that \(\Theta; \Gamma \vdash [x'/x]s : P'\) ,
  \(P \xLongrightarrow{\tau} P'\) and \(\OK_n(P'\)). Take \(P\) as
  \(P'\). Then \( P \xLongrightarrow{\tau} P'\) and \(\OK_n(P')\) hold.  We
  also have \(\Theta; \Gamma \vdash [x'/x]s\COL P\) from \rn{T-Sub},
  \(\Theta; \Gamma \vdash [x'/x]s\COL P_{1}\) and \( P_{1} \le
  P\).

\item Case: $\langle H, R, \LET x = \NULL \ \IN \ s, n \rangle
  \rightarrow \langle H', R', [x'/x]s, n \rangle $

  We have \(\Theta; \Gamma \vdash \LET x = \NULL \ \IN \ s\COL P\)
  and \(OK_{n}(P)\). From the inversion of typing rules, we have
  \(\Theta; \Gamma \vdash s\COL P_{1}\) and \( P_{1} \le P\).

  We need to find $P'$ such that \(\Theta; \Gamma' \vdash [x'/x]s\COL P'\),
  \(P \xLongrightarrow{\tau} P'\) and \(\OK_n(P')\).  Take \(P\) as
  \(P'\).  Then, \(P \xLongrightarrow{\tau} P'\) and \(\OK_n(P')\) hold.  We
  also have \(\Theta; \Gamma \vdash [x'/x]s\COL P\) from \rn{T-Sub},
  \(\Theta; \Gamma \vdash [x'/x]s\COL P_{1}\)\( P_{1} \le
  P\).

\item Case: $\langle H, R, \LET x = *y \ \IN \ s, n \rangle
  \rightarrow \langle H', R', [x'/x]s, n \rangle $

  We have \(\Theta; \Gamma \vdash \LET x = *y \ \IN \ s\COL  P\) and
  \(OK_{n}(P)\). From the inversion of typing rules, we have \(\Theta;
  \Gamma \vdash s\COL P_{1}\) and \(P_{1} \le P\).

  We need to find \(P'\) such that \(\Theta; \Gamma \vdash [x'/x]s\COL
  P'\), \(P \xLongrightarrow{\tau} P'\) and \(\OK_n(P')\). Take \(P\) as
  \(P'\). Then, \(P \xLongrightarrow{\tau} P'\) and \(\OK_n(P')\) hold.  We
  also have \(\Theta; \Gamma' \vdash [x'/x]s\COL P\) from \rn{T-Sub},
  \(\Theta; \Gamma' \vdash [x'/x]s\COL P_{1}\) and \(P_{1} \le P\).
        
\item Case(\rn{Tr-IfNullT}): \(\langle H, R, \IFNULL \Cirx \ \THEN s_{1} \ \ELSE \ s_{2},
  n \rangle \rightarrow \langle H, R, s_{1}, n \rangle\)

  We have \(\Theta; \Gamma \vdash \IFNULL \Cirx \ \THEN \ s_{1}
  \ \ELSE \ s_{2}\COL P\) and \(\OK_{n}(P)\). From the inversion of
  typing rules, we have \(\Theta; \Gamma \vdash s_{1} : P_{1}\) and \(P_{1}
  \le P\).

  We need to find $P'$ such that \(\Theta; \Gamma \vdash s_1\COL P'\)
  , \(P \xLongrightarrow{\tau} P'\) and \(\OK_n(P'\)).  Take P as P'.  Then,
  \(\OK_n(P')\) and \(P \xLongrightarrow{\tau} P'\) hold. We also have \(\Theta;
  \Gamma' \vdash s_{1}\COL P\) from \(\Theta; \Gamma \vdash s_{1}\COL
  P_1\), \(P_{1} \le P\) and \rn{T-Sub}.

\item Case(\rn{Tr-IfNullF}): \(\langle H, R, \IFNULL \Cirx \ \THEN s_{1} \ \ELSE \ s_{2},
  n \rangle \rightarrow \langle H, R, s_{2}, n \rangle\).

 The proof is similar to case \rn{Tr-IfNullT}. 

\item Case: $\langle H, R, f(\vec{x}) , n \rangle \rightarrow  \langle H, R, [\vec{x}/\vec{y}]s, n  \rangle $

  Assuming that \(D(f) = s\) and \(\Theta(f) = P_1\), we have \(s\COL
  P_1\). The \([\vec{x}/\vec{y}]s\) also has a type \(P_1\).

  From the assumption, we have \(\Theta;\Gamma \vdash f(x)\COL P\) and
  \(\OK_n(P)\). From the inversion, we have \(\Theta;\Gamma \vdash
  f(x)\COL P_1\) and \(P_1 \le P\).

  We need to find \(P'\) such that \(\Theta;\Gamma \vdash [\vec{x}/\vec{y}]s\COL
  P'\), \(P \xLongrightarrow{\tau} P'\) and \(\OK_n(P')\). Take \(P\) as
  \(P'\). Then \(P \xLongrightarrow{\tau} P'\) and \(\OK_n(P')\) hold. We
  also have \(\Theta;\Gamma \vdash [\vec{x}/\vec{y}]s\COL P \) from
  \(\Theta;\Gamma \vdash [\vec{x}/\vec{y}]s\COL P_1\), \(P_1 \le P\) and
  \rn{T-Sub}.

\end{itemize}

\end{pfof}


\begin{pfof}{Lemma~\ref{lem:immediateSafety}}
Prove by contradiction.

Assuming that \(\langle H, R, s, n \rangle
\not\xlongrightarrow{\Malloc} \OVERFLOW \) does not hold. Then, \(n\)
is \(0\) and the action is \(\Malloc\), from transition rule
\rn{Sem-OutOfMem}.

We have \(\Theta;\Gamma \vdash s \COL P\) and \(\OK_n(P)\). From the
definition of \(\OK_n(P)\), we have \(P \xlongrightarrow{\rho} P'\)
for some \(P'\) and \(\sharp_m(\rho) - \sharp_f(\rho) \le n \). Take
\(\Malloc\) as \(\rho\), we get \( 1 \le n \). Therefore, we get the
contradiction.

\end{pfof}


\section{Proof of Theorem}

\begin{pfof}{Theorem~\ref{thm1}}


We have \( \Theta;\emptyset \vdash s\COL P, \vdash D\COL \Theta\) and
\(\OK_n(P)\) from the well-typed program \(\vdash \langle D,
s\rangle\).

It suffices to prove \( \langle \emptyset, \emptyset, s, n\rangle
\not\xlongrightarrow{\sigma} \OVERFLOW\) for any action sequence
\(\sigma\).

By contradiction, we assume that \( \langle \emptyset, \emptyset,
s, n\rangle \xlongrightarrow{\sigma} \OVERFLOW \) for any \(\sigma =
\rho_0\rho_1\cdots\rho_m\).  That is, \( \langle \emptyset, \emptyset, s, n\rangle
\xlongrightarrow{\rho_0} \langle H_0, R_0, s_0, n_0\rangle
\xlongrightarrow{\rho_1} \cdots \xlongrightarrow{\rho_m} \langle H_m,
R_m, s_m, n_m \rangle \xlongrightarrow{\Malloc} \OVERFLOW \).

By Lemma~\ref{lem:preservation}, we have \(\OK_{n_m}(P_m)\),
\(\Theta;\Gamma\vdash s_m \COL P_m\) and \(P \xLongrightarrow{\sigma}
P_m\). Then, by the lemma~\ref{lem:immediateSafety}, we get the
contradiction.

\end{pfof}
