%\documentclass[japanese]{jssst_ppl} %% 
 \documentclass[english]{jssst_ppl} %% English
% \documentclass[japanese,draft]{jssst_ppl} %% You can use the draft option
\usepackage{amsmath,amsthm,amssymb,amsfonts,extarrows,geometry,amsopn,enumerate}

% set some new commands %
\newcommand\tB{\;|\;}
\newcommand\LET{\mathbf{let}\;}
\newcommand\FREE{\mathbf{free(x)}\;}
\newcommand\IN{\mathbf{in}\;}
\newcommand\SKIP{\mathbf{skip}}
\newcommand\Rtab{\; \; \; \;}
\newcommand\NULL{\mathbf{null}\;}
\newcommand\IFNULL{\mathbf{ifnull}\;}
\newcommand\THEN{\mathbf{then}\;}
\newcommand\ELSE{\mathbf{else}\;}
\newcommand\Lcc{\left(}
\newcommand\Rcc{\right)}
\newcommand\Lfc{\left\{}
\newcommand\Rfc{\right\}}
\newcommand\Lb{\left[}
\newcommand\Rb{\right]}
\newcommand\coma{,\;}
\newcommand\MALLOC{\mathbf{malloc()}\;}
\newcommand\Malloc{\mathbf{malloc}}
\newcommand\Free{\mathbf{free}}
\newcommand\Cirx{(x)}
\newcommand\dtb{\;\;\ \;\;\ \;\;\ \;\;\  }
\newtheorem{theorem}{Theorem}[section]
\newtheorem{lemma}[theorem]{Lemma}
\newtheorem{proposition}[theorem]{Proposition}
\newtheorem{corollary}[theorem]{Corollary}
\newtheorem{myDef}{Definition}

\newcommand\todo[1]{{\bf KS: {#1}}}

\title{A Behavioral Type System for Memory-Leak Freedom}
\author{Author Name}
\inst{%
$^1$ \\
\texttt{sample@example.ac.jp}
\medskip\par%
$^2$ \\
\texttt{example@sample.net}
}
\begin{document}
\maketitle
\begin{abstract}
We propose an approach to guarantee safe memory deallocation for nonterminating programs. The main idea is to guarantee the property using two type systems: one existed type system, proposed by Suenaga and Kobayashi,  which can ensure programs no double frees and no illegal read/write operations to a deallocated memory cell, but partial memory-leak freedom (i.e., not leaking memory if a program terminates); and the other type sytem proposed in our paper which abstracts the behavior of memory allocation and deallocation by CCS-like processes. The latter one, a behavioral type system, mainly guarantees memory-leak freedom for nonterminating programs by estimating the upper bound of consumed memory cells ignoring the relationship between variables and pointers to memory cells, which means that the difference between the number of allocations and deallocations will never exceed the number of avalibale memory cells. A nonterminating program can be guaranteed safe memory deallocation if it is verified by these two type systems.
\end{abstract}

\section{Introduction}
\subsection{Motivation and Problems}
Manual memory management primitives (e.g., \texttt{malloc} and \texttt{free} in C) often cause serious problems such as double frees, memory leaks, and illegal read/write operations to a deallocated memory cell. Verifying \emph{safe memory deallocation} -- a program not leading to such an unsafe state -- is an important problem.

Most of safe memory deallocation verification techniques proposed so far~\cite{DBLP:conf/aplas/SuenagaK09,DBLP:conf/pldi/HeineL03,DBLP:conf/sigsoft/XieA05,DBLP:journals/scp/SwamyHMGJ06} refer to \emph{partial memory-leak freedom}: if a program terminates, already allocated memory cells are all deallocated. For example, the type system by Suenaga and Kobayashi~\cite{DBLP:conf/aplas/SuenagaK09}, which is called \textbf{SK} type system in our paper, guarantees that (1) a well-typed program does not perform read/write/free operations to any deallocated memory cell and that (2) after execution of a well-typed program, all the memory cells are deallocated. That is, it guarantees \emph{partial correctness}: no double frees, no illegal accesses to a deallocated memory cell, but partial memory-leak freedom. We should notice that the first property about no double frees and no use after deallocation can be ensured for nonterminating programs, while the second is guaranteed if a well-typed program terminates.

The function $g$ shown in Figure~\ref{example:pam} describe this situation. It executes like allocating a memory cell, calling it self again and then deallocating the allocated memory cell. By the \textbf{SK} type system, the first property is guaranteed because of no double frees and no illegal accesses to a deallocated memory cell, and the second property, partial memory-leak freedom, is guaranteed because it does not terminate. However, it infinitely allocates a memory cell, which causes a dangerous situation -- memory leaks.
\begin{figure}[h]
1 \dtb $g(x)$= \dtb\dtb\dtb\Rtab\;\;$f(x)$=  \\
2 \dtb $\LET \; x = \MALLOC  \; \IN$\dtb\dtb$\LET \; x = \MALLOC  \; \IN$\\
3 \dtb $g(x)$ ; \; $\Free\Cirx$ \dtb\dtb\Rtab\;\;$\Free\Cirx$; \; $f(x)$
\caption{Explanation for partial correctness and memory leak}
\label{example:pam}
\end{figure}

A program \emph{leaks} memory if the program consumes unbounded number of memory cells. For example, function $g$ in Figure~\ref{example:pam} leaks memory, whereas function $f$ does not; the former consumes unbounded number of memory cells but the latter consumes at most one memory cell at once. Notice that these two programs are all partially corrected by the \textbf{SK} type system because they do not terminate.

In the real-world programs, nonterminating programs such as Web servers and operating systems are very important. To guarantee memory-leak freedom for non-terminating programs has great practical significance.
\subsection{Main Observation}
We notice that once partial correctness is guaranteed, we can guarantee memory-leak freedom for nonterminating by estimating the upper bound of memory consumption ignoring the relationship between variables and pointers to memory cells. For demonstrating this observation, we use an example in Figure~\ref{example:observation}. The function $h$ is partially corrected. The behavior of $h$ is that it consumes two memory cells at once. That is, it is safe if the difference between the allocations and deallocations does not exceed the number of available memory cells, and the number of available memory cells is at least 2. In order to verify this behavior, we ignore the variables $x$ and $y$ in $h$ to focus on the fact that $h$ executes $\Malloc$ twice, $\Free$ twice, and then calls $h$. This abstraction is sound because the correspondence between allocations and deal locations is guaranteed by the partial correctness verification. Now We can focus on abstraction of hevavior of allocation and deallocaion.
\begin{figure}[h]
1\dtb \dtb \dtb $h(x)$=\\
2\dtb \dtb \dtb $\LET \; x = \MALLOC  \; \IN$\\
3\dtb \dtb \dtb $\LET \; y = \MALLOC  \; \IN$\\
4\dtb \dtb \dtb $\Free\Cirx$; $\Free(y) $;\;$h(x)$
\caption{Example for demonstrating the main observation.}
\label{example:observation}
\end{figure}
\subsection{Approach}
To guarantee safe memory deallocation for nonterminating programs, our key idea is to decompose this problem into two subproblems: (1) partial correctness and (2) \emph{behavioral correctness}. The partial correctness has been described above: no double frees and no use after deallocation, but partial memory-leak freedom. The behavioral correctness means a program does not leak memory and uses the method described in above subsection math\{1.2} to guarantee memory leak freedom for nonterminating programs. It is verified by behavioral type system which is mainly used to abstract the behavior of a program. Behavioral types are heavily used in the context of concurrent program verification\cite{DBLP:journals/lmcs/KobayashiSW06,DBLP:journals/tcs/IgarashiK04,DBLP:conf/esop/HondaVK98}.

In our paper, the behavior of a program is abstracted as CCS-like processes~\cite{DBLP:journals/iandc/MilnerPW92a}. For example, the behavior of $f$ is abstracted as $\mu \alpha. \Malloc;\Free;\alpha$; the behavior of $g$ is abstracted as $\mu \alpha. \Malloc;\alpha;\Free$; the behavior of $h$ is abstracted as $\mu \alpha. \Malloc;\Malloc;$ \\ $\Free;\Free;\alpha$. 

The rest of this paper is structured as follows. Section 2 introduces a simple imperative language, as well as its syntax and operational semantics. Section 3 introduces the behavioral type system, which describes how to guarantee memory-leak freedom for non-terminating programs. Section 4 proposes an inference algorithm, and talks about syntax directed typing rules. Section 5 describes current status and future work.

\section{Language}
This section introduces a sublanguage of Suenaga and Kobayashi~\cite{DBLP:conf/aplas/SuenagaK09} with primitives for memory allocation/deallocation. And the values in our paper are only pointers. \\
The syntax of language is as follows.
\subsection{Syntax}
\begin{eqnarray*}
  s \ (\mathit{statements})& &::=  \SKIP \tB s_{1};s_{2} \tB *x \leftarrow y \tB \Free \Cirx \\
  & & \tB \LET x = \MALLOC \IN s \tB \LET x = \NULL \IN s  \\
  & & \tB \LET x = y \; \IN s \tB   \LET x = *y \; \IN s \\
  & &\tB \IFNULL(x) \; \THEN s_{1}\; \ELSE s_{2} \tB f(\vec{x})\\
  d \ (\mathit{definition})& &::= f(x_{1},\dots,x_{n}) = s
\end{eqnarray*}
A program is a pair $(D,s)$, where $D$ is the set of definition.\\
The command $\SKIP$ does nothing. The command $s_{1};s_{2}$ is executed as a sequence, first executing $s_{1}$ and then $s_{2}$. The command $*x \leftarrow y$ updates the content of the memory cell which is pointed by pointer $x$ with value $y$. The command $\Free \Cirx$ deallocates the memory cell which is pointed by a pointer $x$. Then command $\LET x = e \; \IN s$ first evaluates the expression $e$ and binds the return value of $e$ to $x$ and then executes statement $s$. The command $\LET x = \Malloc \; \IN s$ first allocates a memory cell to a pointer $x$ and then executes the statement $s$. The command $\LET x = \NULL  \IN s$ first allocates a null pointer to $x$ and then executes $s$. The command $\LET x = y \; \IN s$ assign the pointer $y$ to $x$, so the pointer $x$ and $y$ are said aliases for the same memory cell, and then executes statement $s$. The command $\LET x = *y \; \IN s$ transfers a part of memory cells pointed by $y$ and then executes statement $s$. The command $\IFNULL(x) \; \THEN s_{1} \; \ELSE s_{2}$ denotes that  executing statement $s_{1}$ if pointer $x$ is a null pointer, otherwise executing statement $s_{2}$. The command $f(\vec{x})$ is a function call in which $\vec{x}$ denotes mutually distinct variables like \{$x_{1}, \dots, x_{n}$\}. The notation $d$ denotes the definition of function $f(\vec{x})$ which has a body of statement $s$. And examples are described by this syntax you can see in Figure 1 and  Figure 2.

\subsection{Operational Semantics}
Because we want to estimate the number of available memory cells at every operation step, we extend the triple $\langle H\coma R\coma s \rangle$ that is represented as run-time state in previous type system to a quadruple $\langle H\coma R\coma s\coma n \rangle$ in our paper. The introduced notation $n$ denotes the number of available memory cells, a nature number. When executing the operation $\Malloc$, the number of available memory cells will decrease 1, which is denoted as ($n - 1$); when executing the operation $\Free$, the number of available memory cells will increase 1, which is denoted as ($n + 1$). The notation $H$, which models heap memory, is a mapping from finite subset of $\mathcal{H}$ to $\mathcal{H}$ $\cup$ \{$null$\}, where $\mathcal{H}$ represents the set of \emph{heap addresses}. $R$, which models registers, is a mapping from finite set of variables to $\mathcal{H}$ $\cup$ \{$null$\}.

Transition rules are listed in Figure 3. In these rules, $f\{x\to v\}$ is defined as a function $f'$ such that $f'(y) = v$ if $x = y$, otherwise $f'(y) = f(y)$ and $y \in dom(f)$. There are three rules about $\mathbf{NullEx}$ which denotes accessing a null pointer, three rules about $\mathbf{Error}$ for accessing a deallocated memory cell, and one rule about $\mathbf{Error}$ which denotes allocating a memory cell when there is no memory space.
%\begin{figure}[h]
% Skip Command
$$
    \frac{n \in \mathbb{N}}
            {\langle H\coma R\coma  \SKIP;s , n \rangle
              \longrightarrow_{D}
                \langle H\coma R\coma   s , n \rangle }
     \Rtab \mbox{(E-Skip)}
$$
% Assignment
$$
     \frac{R(x) \in dom(H), n \in \mathbb{N}}
           {\langle H\coma R\coma  *x \leftarrow y , n \rangle
             \longrightarrow_{D}
             \langle H \Lfc R(x) \rightarrow R(y) \Rfc \coma R \coma   \SKIP , n  \rangle }
     \Rtab \mbox{(E-Assign)}
$$
% Free Command
$$
     \frac{R(x) \in dom(H) , n \in \mathbb{N}}
          {\langle H\coma R\coma  \FREE , n \rangle
            \xlongrightarrow{\Free}_{D}
            \langle H\backslash \Lfc R(x) \Rfc \coma R \coma   \SKIP , n+1  \rangle }
     \Rtab \mbox{(E-Free)}
$$
% Let Null Command
$$
     \frac{x' \notin dom(R)}
           {\langle H\coma R\coma  \LET x = \NULL \IN s , n \rangle
             \longrightarrow_{D}
             \langle H\coma R\Lfc x' \rightarrow \NULL \Rfc \coma   \Lb x'/x \Rb s , n  \rangle }
     \Rtab \mbox{(E-LetNull)}
$$
% Let Eq Command
$$
     \frac{x' \notin dom(R)}
            {\langle H\coma R\coma \LET x = y \; \IN s , n \rangle
              \longrightarrow_{D}
              \langle H\coma R\Lfc x' \rightarrow R(y) \Rfc \coma   \Lb x'/x \Rb s , n  \rangle }
\Rtab \mbox{(E-LetEq)}
$$
% Reference Command
$$
     \frac{x' \notin dom(R)}
            {\langle H\coma R\coma  \LET x = *y \; \IN s , n \rangle
              \longrightarrow_{D}
              \langle H\coma R\Lfc x' \rightarrow H(R(y)) \Rfc \coma   \Lb x'/x \Rb s , n  \rangle }
     \Rtab \mbox{(E-LetDref)}
$$
% Malloc (allocate) Command
$$
     \frac{h \notin dom(H)}
            {\langle H\coma R\coma  \LET x = \Malloc() \; \IN s , n \rangle
              \xlongrightarrow{\Malloc}_{D}
              \langle H \Lfc h \rightarrow v\Rfc \coma R\Lfc x' \rightarrow h \Rfc \coma   \Lb x'/x \Rb s , n-1  \rangle }
\Rtab \mbox{(E-Malloc)}
$$
% IFNULL T
$$
    \frac{R(x) = \NULL}
           {\langle H \coma R \coma \IFNULL\Cirx   \THEN   s_{1} \ELSE\  s_{2} \coma  n \rangle
           \longrightarrow_{D}
           \langle H\coma R\coma s_{1} \coma n \rangle}
    \Rtab \mbox{(E-IfNullT)}
$$
% IFNULL F
$$
    \frac{R(x) \neq \NULL}
           {\langle H \coma R \coma \IFNULL\Cirx \THEN  s_{1} \ELSE  s_{2} \coma  n \rangle
           \longrightarrow_{D}
           \langle H\coma R\coma s_{2} \coma  n, \rangle}
    \Rtab \mbox{(E-IfNullF)}
$$
% Function Call
$$
     \frac{f(\vec{y}) = s \in D}
            { \langle H\coma R\coma  f(\vec{x}) , n \rangle
               \longrightarrow_{D}
               \langle H\coma R\coma  \Lb \vec{x}/\vec{y} \Rb s , n \rangle}
      \Rtab \mbox{(E-Call)}
$$
% Error : access the null memory cell
$$
      \frac{R(x) = null}
            {\langle H\coma R\coma  *x \leftarrow y , n \rangle
              \longrightarrow_{D}
             \bf NullEx }
      \Rtab \mbox{(E-AssignNullError)}
$$
% ERROR : access the null memory cell
$$
      \frac{R(y) = null}
             {\langle H\coma R\coma  x = *y, n \rangle
               \longrightarrow_{D}
              \bf NullEx }
             \Rtab \mbox{(E-DrefNullError)}
$$
$$
     \frac{R(x) =  null }
           {\langle H\coma R\coma  \FREE , n \rangle
             \xlongrightarrow{\Free}_{D} \bf NullEx  }
      \Rtab \mbox{(E-FreeNullError)}
$$
% ERROR :
$$
     \frac{R(x) \notin dom(H) \cup \Lfc null \Rfc}
           {\langle H\coma R\coma   *x \leftarrow y,  n \rangle
             \longrightarrow_{D}
           \bf  Error }
    \Rtab \mbox{(E-AssignError)}
$$
% ERROR
$$
      \frac{R(y) \notin dom(H) \cup \Lfc null \Rfc}
           {\langle H\coma R\coma  \LET x  = *y \; \IN s, n \rangle
              \longrightarrow_{D}
                \bf  Error }
      \Rtab \mbox{(E-DrefError)}
$$
%
$$
      \frac{R(x) \notin dom(H) \cup \Lfc null \Rfc}
            {\langle H\coma R\coma  \FREE , n \rangle
              \xlongrightarrow{\Free}_{D}
              \bf Error }
     \Rtab \mbox{(E-FreeError)}
$$
 % ERROR: no enough space
$$
      \langle H\coma R\coma \LET x = \Malloc() \ \IN s ,  0  \rangle
      \xlongrightarrow{\Malloc}_{D}
      \mathbf{Error}
      \Rtab \mbox{(E-MallocError)}
$$
$$
     \mathbf{Figure \; 3.} \;\;  \mbox{ Operational Semantics}
$$
%
%\caption{Operational Semantics.}
%\label{example:os}
%\end{figure}

\section{Type System}
This section elaborate the behavioral type system to prevent leaking memory in non-terminating programs. We define behavioral types, CCS-like processes that abstract the behavior of programs, as follows.
\subsection{Syntax of Types}
     \begin{eqnarray*}
       P (\mathit{behavioral\ types})::=&& {\bf 0} \tB P_{1};P_{2} \tB P_{1}+P_{2} \tB \Malloc\\
       &&\tB \Free \tB \alpha \tB \mu\alpha.P \\
       \tau (\mathit{value\ types})::=&&    \mathbf{Ref}  \\ %
       \sigma (\mathit{function\ types})::=&& (\tau_{1},\dots, \tau_{n}) P
     \end{eqnarray*}

The type $\bf 0$ abstracts the behavior of $\SKIP$ and means "does nothing". $P_{1};P_{2}$ is for sequential execution. $P_{1} + P_{2}$ is abstracted as conditional. $\Malloc$ is the behavior of a statement that allocates a memory cell exactly once. $\Free$ is for deallocating memory cell exactly once. $\mu \alpha. P$ is a recursive type. For example, the behavior of  the body of function $h$ in Figure 2 is abstracted as $\mu \alpha. \Malloc;\Malloc;\Free;\Free;\alpha$. $\alpha$ is a type variable and bounded to the recursive constructor $\mu \alpha$.

The only value in our paper is reference, and its type is $\mathbf{Ref}$.

The function type is described as $(\tau_{1}, \dots, \tau_{n})P$, which means a function receives some pointers as arguments and its body is abstracted as a behavioral type $P$.

% Semantics of Behavioral Types %
\subsection{Semantics of Behavioral Types}
The semantics of behavioral type are given by labeled transition system, and listed as follows:
    $$
         0;P \rightarrow P
    $$
    $$
          \Malloc \xlongrightarrow{\Malloc} 0
    $$
    $$
           \Free \xlongrightarrow{\Free} 0
    $$
    $$
          \mu \alpha.P \rightarrow  [\mu \alpha . P/\alpha]  P
    $$
   $$
          P_{1} + P_{2} \longrightarrow P_{1}
   $$
   $$
          P_{1} + P_{2} \longrightarrow P_{2}
   $$
   $$
           \frac{P_{1} \xlongrightarrow{\alpha} P_{1}' }
                 {P_{1};P_{2} \xlongrightarrow{\alpha} P_{1}';P_{2}}
   $$
The notation $\rightarrow$ denotes that a behavioral type can be reduced by the internal action. Notation $\xlongrightarrow{\alpha}$ means that a behavioral type can be reduced by executing $\alpha$ actions, and the $\alpha$ here is $\{\Malloc, \Free\}$.

% Type Judgments
\subsection{Typing Rules}
The type judgment of our type system is given by the form $\Theta ; \Gamma \vdash s : P$, where $\Theta$ is a mapping from function variables to function types, $\Gamma$ is a type environment that denotes a mapping from variables to value types.
It reads ``the behavior of $s$ is abstracted as $P$ under $\Theta$ and $\Gamma$ environments''. We design the type system so that this type judgment implies the property: when $s$ executes $\Malloc$(resp.$\Free$), then $P$ is equivalent to $\Malloc;P'$(resp.$\Free;P'$) for a type $P'$ such that $\Theta; \Gamma \vdash s': P'$, where $s'$ is the continuation of $s$. This property guarantees the behavioral type soundly abstracts the upper bound of the consumed memory cells.

Typing rules are presented in Figure 4. In the rule for assignment, the behavior of  $*x \leftarrow y$ is $\bf 0$. The rule for $\Free$ represents that the behavior of $\Free \Cirx$ is $\Free$. The rule T-Malloc represents that $\LET x = \MALLOC \; \IN s$ has the behavior $\Malloc;P$, where $P$ is the behavior of statement $s$. The rule for function call represents that function $f$ has the behavior $P$ which is the behavior of the body of this function .

In the rule for subtyping, $P_{1} \le P_{2}$ represents that $P_{1}$ is the subtype of $P_{2}$ and  means that: \\
(1) if $P_{1} \xlongrightarrow{\alpha}  P_{1}'$ then $\exists P_{2}' $ s.t. $P_{2} \overset{\text{$\alpha$}}{\Longrightarrow} P_{2}'$ and $ P_{1}' \le P_{2}' $\\
(2) if $P_{1} \rightarrow P_{1}'$ then $\exists P_{2}'$ s.t. $P_{2} \rightarrow^{*} P_{2}'$ and  $P_{1}' \le P_{2}'$\\
where $\overset{\text{$\alpha$}}{\Longrightarrow}$ means that: $\rightarrow^{*} \xlongrightarrow{\alpha} \rightarrow^{*}$.

In the rule for program, the main statement $s$ is executed under $\Theta$ and $\Gamma$ environments without free variables. At the end of $s$, memory leak freedom is guaranteed by $OK_{n}(P)$ ,where $P$ is the behavior of $s$. $OK_{n}(P)$ is defined as $\mathbf{Definition\; 1}$ in which $\sharp_{malloc}(\alpha)$ and $\sharp_{free}(\alpha)$ are functions to count the number of $\Malloc$ and $\Free$ actions in $\alpha$ respectively. This definition, intuitively, means at every running step the number of allocated memory cells will never go out of memory scope.
\begin{myDef}
 $OK_{n}(P) \iff \forall P',\; P \xlongrightarrow{\alpha}^{*}P'$ then $\sharp_{malloc}(\alpha)-\sharp_{free}(\alpha)\le n$.
\end{myDef}

% Skip type
$$
         \Theta ; \Gamma \vdash \SKIP : \mathbf{0}
      \Rtab \mbox{(T-Skip)}
$$
% Sequence type
$$
      \frac{\Theta ; \Gamma \vdash s_{1} : P_{1} \Rtab \Theta ; \Gamma \vdash s_{2} : P_{2}}
          {\Theta ; \Gamma \vdash s_{1} ; s_{2} : P_{1};P_{2} }
     \Rtab \mbox{(T-Seq)}
$$
% Assignment type
$$
     \frac{\Theta ; \Gamma \vdash y :  \mathbf{Ref} \Rtab \Theta ; \Gamma \vdash x : \mathbf{Ref} }
          {\Theta ; \Gamma \vdash *x \leftarrow y : \mathbf{0} }
     \Rtab \mbox{(T-Assign)}
$$
% Free(deallocate) type
$$
     \frac{\Theta ; \Gamma \vdash x : \mathbf{Ref} }
           {\Theta ; \Gamma \vdash \Free(x) : \Free}
     \Rtab \mbox{(T-Free)}
$$
% Malloc type
$$
     \frac{\Theta ; \Gamma,x : \mathbf{Ref} \vdash s : P}
           {\Theta ; \Gamma \vdash \LET x = \MALLOC \; \IN s  : \Malloc;P}
           \Rtab \mbox{(T-Malloc)}
$$
% Let eq type
$$
     \frac{\Theta ; \Gamma \vdash y : \mathbf{Ref}  \Rtab \Theta ; \Gamma , x : \mathbf{Ref} \vdash s : P}
           {\Theta ; \Gamma \vdash \LET x = y \; \IN s : P}
     \Rtab \mbox{(T-LetEq)}
$$
% Dereference type
$$
     \frac{\Theta ; \Gamma \vdash y : \mathbf{Ref}  \Rtab \Theta ; \Gamma , x : \mathbf{Ref} \vdash s : P}
           {\Theta ; \Gamma \vdash \LET x = *y \; \IN s : P}
     \Rtab \mbox{(T-LetDref)}
$$
% Let NULL type
$$
     \frac{\Theta ; \Gamma, x : \mathbf{Ref} \vdash s : P}
           {\Theta ; \Gamma \vdash \LET x = \mathbf{null} \; \IN s : P}
     \Rtab \mbox{(T-LetNull)}
$$
% Subtyping
$$
     \frac{\Theta ; \Gamma \vdash s : P_{1} \Rtab P_{1} \le P_{2}}
            {\Theta ; \Gamma \vdash s : P_{2}}
     \Rtab \mbox{(T-Sub)}
$$
 % ifnull s then s type
$$
     \frac{\Theta ; \Gamma \vdash x : \mathbf{Ref}   \ \ \ \  \Theta ; \Gamma \vdash s_{1} : P \ \ \ \ \Theta ; \Gamma \vdash s_{1} : P}
           {\Theta ; \Gamma \vdash \IFNULL(x) \; \THEN s_{1}\; \ELSE s_{2} : P}
     \Rtab \mbox{(T-IfNull)}
$$
% Function call type
$$ \frac{ \Theta(f) = P}
{\Theta; \Gamma, \vec{x} : \vec{\tau} \vdash f(\vec{x}) : P}
\Rtab \mbox{(T-Call)} $$
% Program
$$\frac{\vdash D : \Theta \;\;\;\; \Theta; \emptyset\vdash s : P \Rtab OK_{n}(P)}
{\vdash (D, s)}
\Rtab \mbox{(T-Program)} $$
$$
    \mathbf{Figure \; 4.} \;\;\mbox{Typing Rules}
$$

\subsection{Type Soundness}
This subsection describes some theorems and lemmas for type safety.
\begin{theorem}\label{thm1}
If $\vdash (D, s)$ then $(D, s)$ does not lead to $memory\;leak$.\\
Memory leak freedom: $\exists n \in \mathbb{N}$ s.t.
$\langle \emptyset, \emptyset, s, n \rangle \nrightarrow^{*}Error$
\end{theorem}
\noindent
This theorem says that a well typed program guarantees memory leak freedom.
% Lemma Preservation %
\begin{lemma}[Preservation $\mathbf{I}$]%\label{preser}
If $OK_{n}(P)$, $\Theta; \Gamma \vdash s : P$ and $\langle H,R,s, n \rangle
\xlongrightarrow{\alpha}\langle H',R',s', n'
\rangle$, then $\exists P'$ s.t. \\
(1) $ \Theta; \Gamma \vdash s' : P' $ \\
(2) $ P \overset{\text{$\alpha$}}{\Longrightarrow} P'$\\
(3) $ OK_{n'}(P') $
\end{lemma}
\begin{lemma}[Preservation $\mathbf{II}$]%\label{preser}
If $OK_{n}(P)$, $\Theta ; \Gamma \vdash s : P$ and $\langle H,R,s,n \rangle
\rightarrow \langle H',R',s', n'
\rangle$, then $\exists P'$ s.t. \\
(1) $\Theta; \Gamma \vdash s' : P'$\\
(2) $ P \rightarrow^{*} P'  $\\
(3) $OK_{n'}(P')$
\end{lemma}
\begin{lemma}%\label{error}
 The partial correctness is guaranteed $\vdash \langle H,R,s \rangle$ , so that if $\vdash \langle H,R,s,n \rangle$, then $\vdash \langle H',R',s',n' \rangle \nrightarrow Error$
\end{lemma}
\section{Type Inference Algorithm}
This section describes how to construct syntax directed typing rules according to the typing rules of above section ,and it provides an algorithm which inputs statements and returns a pair containing constraints and behavior types.
\subsection{Syntax Directed Typing Rules}
Typing rules showed in Figure 4 are not immediately suitable for type inference. The reason is that the subtyping rule can be applied to any kind of term. This means that, any kind of term $s$ can be applied by either subtyping rule or the other rule whose conclusion matches the shape of the $s$ \cite{plain:book1}.

In order to yield a type inference algorithm, we should do something with the subtyping rule. The method is to merge the subtyping rule with the other rules by introducing a set $C$ of constraints, where $C$ consists of subtype constraints on behavioral types of the form $P_{1}\le P_{2}$ and $OK_{n}(P)$.

Syntax directed typing rules are listed in Figure 5.

$$
     \frac{ C = \emptyset}
           {\Theta; \Gamma; C \vdash \SKIP : \mathbf{0}}
      \Rtab \mbox{(ST-Skip)}
$$
$$
      \frac{\Theta;\Gamma ; C_{1} \vdash s_{1} : P_{1} \Rtab \Theta; \Gamma ; C_{2} \vdash s_{2} : P_{2} \Rtab C = C_{1}\cup C_{2} \cup \{ P_{1};P_{2} \le P\}}
      {\Theta;\Gamma; C \vdash s_{1};s_{2} : P}
      \Rtab \mbox{(ST-Seq)}
$$
$$
      \frac{\Theta;\Gamma;C_{1} \vdash y \Rtab \Theta;\Gamma; C_{2} \vdash x : \mathbf{Ref} \Rtab C = C_{1}\cup C_{2}}
      {\Theta;\Gamma; C \vdash *x \leftarrow y : \mathbf{0}}
      \Rtab \mbox{(ST-Assign)}
$$
$$
      \frac{C = \emptyset}
      {\Gamma ; C \vdash \Free() : \Free}
     \Rtab \mbox{(ST-Free)}
$$
$$
     \frac{\Theta;\Gamma, x ; C_{1} \vdash s : P_{1} \Rtab C = C_{1} \cup\{P_{1}\le P\}}
     {\Theta;\Gamma; C \vdash \LET x = \Malloc() \; \IN s : \Malloc ; P}
     \Rtab \mbox{(ST-Malloc)}
$$
$$
     \frac{\Theta;\Gamma; C_{1} \vdash y \Rtab \Theta;\Gamma, x ; C_{2} \vdash s : P_{1} \Rtab C = C_{1}\cup C_{2} \cup \{P_{1} \le P \}}
     {\Theta;\Gamma ; C \vdash \LET x = y \;  \IN s : P}
     \Rtab \mbox{(ST-LetEq)}
$$
$$
     \frac{\Theta;\Gamma ; C_{1} \vdash y: \mathbf{Ref} \Rtab \Theta;\Gamma, x ; C_{2} \vdash s : P_{1} \Rtab C = C_{1}\cup C_{2}\cup\{P_{1} \le P\}}
     {\Theta;\Gamma ; C \vdash \LET x = *y \; \IN s : P}
     \Rtab \mbox{(ST-LetDref)}
$$
$$
     \frac{\Theta;\Gamma; C_{1} \vdash x \Rtab \Theta;\Gamma; C_{2} \vdash s_{1} : P_{1} \Rtab \Theta;\Gamma; C_{3} \vdash s_{2} : P_{2}  \Rtab  C = C_{1} \cup C_{2} \cup C_{3} \{P_{1}\le P, P_{2}\le P \}}
     {\Theta;\Gamma; C \vdash \IFNULL\Cirx \THEN s_{1} \ELSE s_{2} : P }
    \; \;  \mbox{(ST-IfNull)}
$$
$$
     \frac{\Theta(f) = P_{1} \Rtab C = P_{1} \le P}
     {\Gamma,\vec{x}:\vec{\tau} \vdash f(\vec{x}) : P }
     \Rtab \mbox{(ST-Call)}
$$
$$
     \frac{\Theta \vdash D : \Theta \Rtab \Theta ; \emptyset ; C_{1} \vdash s : P \Rtab C = C_{1}\cup\{OK_{n}(P)\}}
     {C \vdash (D , s) }
     \Rtab \mbox{(ST-Program)}
$$
$$
    \mathbf{Figure \; 5.} \;\; \mbox{Syntax Directed Typing Rules}
$$
\subsection{Algorithm}
By syntax directed typing rules, the type inference algorithm has been designed as in Figure 6.

Function $PT_{v}(x) = (C,\emptyset)$ denotes that it receives a pointer variable $x$ and outputs a pair consisting of constraints set $C$ and an empty set. $PT_{\Theta}(s) = (C, P)$ is a mapping from statements to a pair --  constraints set $C$ and behavioral types $P$, where $\Theta$ is mapping from function names to function types. $PT(\langle D,s \rangle) = (C, P)$ denotes that it receives a program and produces a pair $(C, P)$. $\alpha_{i}$ and $\beta$ are fresh type variables.

\begin{flalign*}
   PT_{\Theta}(f) &  =  &\\
  \LET & \alpha = \Theta(f) & \\
  \IN   & (C = \{\alpha \le \beta \}, \beta) &
\end{flalign*} \noindent
\begin{flalign*}
   PT_{\Theta}(\SKIP) &  =  (\emptyset, 0)&
\end{flalign*}
\begin{flalign*}
   PT_{\Theta}(s_{1}&;s_{2})  =  &\\
   \LET & (C_{1}, P_{1}) = PT_{\Theta}(s_{1}) & \\
          & (C_{2}, P_{2}) = PT_{\Theta}(s_{2}) & \\
   \IN   &(C_{1} \cup C_{2}\cup \{P_{1}; P_{2} \le \beta \}, \beta) &
\end{flalign*}
\begin{flalign*}
   PT_{\Theta}(*x \leftarrow y) &  =  &\\
   \LET & (C_{1}, \emptyset) = PT_{v}(*x) & \\
          & (C_{2}, \emptyset) = PT_{v}(y) & \\
   \IN    &(C_{1} \cup C_{2},  0) &
\end{flalign*}
\begin{flalign*}
   PT_{\Theta}(\Free(x)) &  = (\emptyset, \Free)  &
\end{flalign*}
\begin{flalign*}
   PT_{\Theta}(\LET x& = \Malloc() \  \IN s)  =  &\\
   \LET & (C_{1}, P_{1}) = PT_{v}(s) & \\
   \IN   & (C_{1} \cup \{P_{1} \le \beta \} ,  \Malloc; \beta) &
\end{flalign*}
\begin{flalign*}
   PT_{\Theta}(\LET x& = y \  \IN s )  =  &\\
   \LET & (C_{1}, \emptyset) = PT_{v}(y) & \\
          & (C_{2}, P_{1}) = PT_{\Theta}(s) & \\
   \IN   &(C_{1} \cup C_{2}\cup \{P_{1} \le \beta \},  \beta) &
\end{flalign*}
\begin{flalign*}
   PT_{\Theta}(\LET x& = *y \  \IN s )  =  &\\
   \LET & (C_{1}, \emptyset) = PT_{v}(y) & \\
          & (C_{2}, P_{1}) = PT_{\Theta}(s) & \\
   \IN   &(C_{1} \cup C_{2}\cup \{P_{1} \le \beta \},  \beta) &
\end{flalign*}
\begin{flalign*}
   PT_{\Theta}(\IFNULL(x)& \  \THEN  s_{1} \  \ELSE \ s_{2} )  =  &\\
   \LET & (C_{1}, P_{1}) = PT_{\Theta}(s_{1}) & \\
          & (C_{2}, P_{2}) = PT_{\Theta}(s_{2}) & \\
          & (C_{3}, \emptyset) = PT_{v}(x) & \\
   \IN   &(C_{1} \cup C_{2}\cup C_{3}\cup \{P_{1} \le \beta, P_{2} \le \beta \},  \beta) &
\end{flalign*}
\begin{flalign*}
   PT(\langle D, s \rangle) &  =  &\\
   \LET & \Theta = \{ f_{1}:\alpha_{1}, \dots, f_{n}:\alpha_{n}  \} &\\
          & where \ \{ f_{1},\dots, f_{n} \} = dom(D) \ and \ \alpha_{1}, \dots, \alpha_{n} \  are \ fresh  & \\
  \IN   & \LET  (C_{i}, P_{i}) = PT_{\Theta}(D(f_{i})) \  for \  each \ i & \\
  \IN   & \LET  C_{i}^{'} = \{ \alpha_{i} \le P_{i} \} \ for \  each \ i & \\
  \IN   & \LET  (C, P) = PT_{\Theta}(s)  & \\
  \IN   &(C_{i} \cup C_{i}^{'} ) \cup C \cup  \{OK(P)\},  P) &
\end{flalign*}
$$
\mathbf{Figure\; 6.}\;\; \mbox{Type Inference Algorithm}
$$
\section{Prliminary Experiment}
\section{Related Work}
\section{Conclusion}
We have described a type-based approach to safe memory deallocation for non-terminating programs. The approach is based on the idea of decomposing safe memory memory deallocation into partial correctness, which is verified by previous type system, and behavioral correctness. We designed a behavioral type system in our paper for verification of behavioral correctness. Currently, we are looking for a model checker to estimate an upper bound of consumption given a behavioral type and planning to implement a verifier and conduct experiment to see whether our approach is feasible.
\bibliographystyle{IEEEtran}
\bibliography{tan}

\newpage
\appendix
\section*{Appendix}
\subsection*{1. Proof for Lemma Preservation}

By induction on the derivation of evaluation rules.\\

\noindent Case: $\langle H, R, \FREE, n \rangle \xlongrightarrow{\Free} \langle H', R', \SKIP, n + 1 \rangle $. \\

From the assumption, we have known that: \textcircled{1} $OK_{n}(P)$, and \textcircled{2} $\Theta; \Gamma \vdash \Free\Cirx:P$.

By the inversion lemma on \textcircled{2}, we have: \textcircled{3} $\Free \le P$.

From the definition of subtyping, \textcircled{3} and rule $\Free \xlongrightarrow{\Free} 0$, we get:
\begin{center}
$\exists P''$ s.t. \textcircled{4} $P \overset{\text{$\Free$}}{\Longrightarrow} P''$,  and \textcircled{5} $0 \le P''$
\end{center}

We need to prove that there exists $P'$ and $\Gamma'$ such that:
\begin{center}
\textcircled{6} $\Theta; \Gamma' \vdash \SKIP: P'$,  and \textcircled{7} $P \overset{\text{$\Free$}}{\Longrightarrow} P'$
\end{center}

Take $P''$ as $P'$. Then \textcircled{7} holds. By the typing rule T-Skip and \textcircled{5}, we get:
$$
   \frac{\Theta; \Gamma' \vdash \SKIP : 0 \ \ \ \  0 \le P''}
   {\Theta; \Gamma' \vdash \SKIP : P''}
   \Rtab \mbox{(T-Sub)}
$$

Therefore, \textcircled{6} holds. \\

\noindent Case: $\langle H, R, \LET x = \MALLOC \IN s_{1}, n \rangle \xlongrightarrow{\Malloc} \langle H', R', [x'/x]s_{1}, n - 1  \rangle $.\\

From the assumption, we already have \textcircled{1}$\Theta; \Gamma \vdash \LET x = \MALLOC \IN s_{1} : P$,

and \textcircled{2} $OK_{n}(P)$.

By the inversion lemma and \textcircled{1}, we have \textcircled{3} $\Malloc;P_{1} \le P$, and \textcircled{4} $\Theta; \Gamma \vdash s_{1} : P_{1}$

We need to find $P'$ and $\Gamma'$ such that \textcircled{5} $\Theta; \Gamma' \vdash s_{1} : P'$, and \textcircled{6}$P \overset{\text{$\Malloc$}}{\Longrightarrow} P'$

Because of the following derivation:
$$
  \frac{ \Malloc \xlongrightarrow{\Malloc} 0}
  {\Malloc;P_{1} \xlongrightarrow{\Malloc} 0;P_{1}}
$$

and $0;P_{1} \Rightarrow P_{1}$. Therefore $\Malloc;P_{1} \xlongrightarrow{\Malloc} P_{1}$.

By the definition of subtyping and $\Malloc;P_{1} \xlongrightarrow{\Malloc} P_{1}$, we have that:
\begin{center}
$\exists P''$ s.t. \textcircled{7} $P \overset{\text{$\Malloc$}}{\Longrightarrow} P''$, and \textcircled{8} $P_{1} \le P''$
 \end{center}

Taking $P''$ as $P'$, then \textcircled{6} holds.

And by using subtyping rule T-Sub with premises \textcircled{4} and \textcircled{8}
$$
    \frac{\Gamma \vdash s_{1} : P_{1} \ \ \ \ P_{1} \le P''}
     {\Gamma \vdash s_{1} : P''}
     \Rtab \mbox{(T-Sub)}
$$

Therefore we prove that $\Gamma \vdash s_{1} : P'$, \textcircled{5} holds.\\

\noindent Case: $\langle H, R, \SKIP;s_{1}, n \rangle \rightarrow \langle H', R', s_{1}, n \rangle $. \\

From the assumption, we have
\begin{center}
\textcircled{1} $\Theta; \Gamma \vdash \SKIP;s_{1} : P$, and \textcircled{2} $OK_{n}(P)$
\end{center}

By the inversion lemma on \textcircled{1}, we have
\begin{center}
\textcircled{3} $\Theta; \Gamma \vdash s_{1} : P_{1}$, and \textcircled{4} $0;P_{1} \le P $
\end{center}

We need to prove that there exists $P'$ and $\Gamma'$ such that
\begin{center}
\textcircled{5} $\Theta; \Gamma' \vdash s_{1} : P'$, and \textcircled{6} $P \rightarrow^{*} P'$
\end{center}

By the definition of subtyping and $0;P_{1} \rightarrow P_{1}$, then we get that $\exists P''$
\begin{center}
 \textcircled{7} $P \rightarrow^{*} P''$, and \textcircled{8} $P_{1} \le P''$
\end{center}

Taking $P''$ as  $P'$, we get $P \rightarrow^{*} P'$

And by using rule T-Sub with premises $\Gamma \vdash s_{1} : P_{1}$ and $P_{1} \le P''$, then we have 
$$
    \frac{\Theta; \Gamma \vdash s_{1} : P_{1} \  \  P_{1} \le P''}
    {\Gamma \vdash s_{1} : P''}
    \Rtab \mbox{(T-Sub)}
$$

Therefore, we prove that $\Gamma \vdash s_{1} : P'$ \\

\noindent Case: $\langle H, R, *x \leftarrow y , n \rangle \rightarrow  \langle H', R', \SKIP, n  \rangle $. \\

From the assumption, we already have
\begin{center}
\textcircled{1} $\Theta; \Gamma \vdash *x \leftarrow y : P$, and \textcircled{2} $OK_{n}(P)$
\end{center}

From the inversion lemma on \textcircled{1}, we have \textcircled{3} $0 \le P$.

We need to find $P'$ and $\Gamma'$ such that
\begin{center}
 \textcircled{4} $\Theta; \Gamma' \vdash \SKIP: P'$, and \textcircled{5} $P \rightarrow^{*} P'$
\end{center}

Taking $P$ as $P'$, then \textcircled{5} holds.

And because of the following derivation:
$$
  \frac{\Theta; \Gamma' \vdash \SKIP: 0 \ \ \ 0 \le P}
   {\Theta;\Gamma' \vdash \SKIP : P}
  \Rtab \mbox{(T-Sub)}
$$
therefore \textcircled{4} holds. \\

\noindent Case: $\langle H, R, \LET x = y\  \IN s_{1} , n \rangle \rightarrow  \langle H', R', [x'/x]s_{1}, n  \rangle $. \\

From assumption, we have 
\begin{center}
\textcircled{1} $\Theta; \Gamma \vdash \LET x = y\  \IN \  s_{1} : P$, and \textcircled{2} $OK_{n}(P)$.
\end{center}

From the inversion lemma and \textcircled{1}, we have 
\begin{center}
\textcircled{3} $\Theta; \Gamma \vdash s_{1} : P_{1}$, and $P_{1} \le P$.
\end{center}

We need to find $P'$ and $\Gamma'$ such that:
\begin{align}
  &\Theta; \Gamma' \vdash s_{1} : P' \ \ and& \label{eq5.4.1}\\
  &P \xlongrightarrow{\tau}^{*} P'& \label{eq5.4.2}
\end{align}

Taking P as P'. Therefore \eqref{eq5.4.2} holds, because of the definition of $\rightarrow^{*}$.

And because of the following derivation, \eqref{5.4.1} holds.
$$
  \frac{\Theta; \Gamma' \vdash s_{1} : P_{1} \ \ \ P_{1} \le P}
  {\Theta; \Gamma' \vdash s_{1} : P}
  \Rtab \mbox{(T-Sub)}
$$

\noindent Case: $\langle H, R, \LET x = \NULL \  \IN \  s_{1}, n \rangle \rightarrow \langle H', R', [x'/x]s_{1}, n \rangle $\\

From the assumption, we know that
\begin{center}
\textcircled{1}$\Theta; \Gamma \vdash \LET x = \NULL \  \IN \ s_{1} : P$, and \textcircled{2} $OK_{n}(P)$.
\end{center}

By inversion lemma on \eqref{eqa.1}, we get:
\begin{center}
$\Theta; \Gamma \vdash s_{1} : P_{1}$, and $ P_{1} \le P$.
\end{center}

We need to prove that there exists $P'$ and $\Gamma'$ such that
\begin{center}
$\Theta; \Gamma' \vdash s_{1} : P'$, and $P \rightarrow^{*} P'$.
\end{center}

Taking P as P'. Because of the following derivation, the \eqref{eqa.5} holds.
$$
  \frac{\Theta; \Gamma' \vdash s_{1} : P_{1} \ \ \ \ P_{1} \le P}
  {\Theta; \Gamma' \vdash s_{1} : P}
  \Rtab \mbox{(T-Sub)}
$$

And because of the definition of $\xlongrightarrow{\tau}^{*}$, the \eqref{eqa.6} holds. \\

\noindent Case: $\langle H, R, \LET x = *y \  \IN \  s_{1}, n \rangle \rightarrow \langle H', R', [x'/x]s_{1}, n \rangle $\\

From the assumption, we know that
\begin{center}
$\Theta; \Gamma \vdash \LET x = *y \  \IN \  s_{1} : P$, and $OK_{n}(P)$.
\end{center}

By the inversion lemma on \eqref{eqb.1}, we get:
\begin{center}
$\Theta; \Gamma \vdash s_{1} : P_{1}$, and $P_{1} \le P$.
\end{center}

We need to prove there exists $P'$ and $\Gamma'$ such that:
\begin{center}
$\Theta; \Gamma' \vdash s_{1} : P'$, and $P \rightarrow^{*} P'$.
\end{center}

Taking P as P'. Because of  following derivation, the \eqref{eqb.5} holds.
$$
   \frac{\Theta; \Gamma' \vdash s_{1} : P_{1} \ \ \ P_{1} \le P}
   {\Theta; \Gamma' \vdash s_{1} : P}
   \Rtab \mbox{(T-Sub)}
$$

And because of the definition of $\xlongrightarrow{\tau}^{*}$, \eqref{eqb.6} holds. \\

\noindent Case $\langle H, R, \IFNULL \Cirx \  \THEN s_{1} \  \ELSE \  s_{2}, n \rangle \rightarrow \langle H', R',s_{1}, n \rangle $\\

From the assumption, we have that:
\begin{center}
$\Theta; \Gamma \vdash \IFNULL \Cirx \  \THEN \  s_{1} \ \ELSE \ s_{2} : P$, and $OK_{n}(P)$.
\end{center}

By the inversion leamma on \eqref{eqc.1}, we get:
\begin{center}
$\Theta; \Gamma \vdash s_{1} : P_{1}$, and $ p_{1} \le P'$.
\end{center}

We need to prove that there exists $P'$ and $\Gamma'$ such that:
\begin{center}
 $\Theta; \Gamma' \vdash s_{1} : P_{1}$, and $P \rightarrow^{*} P'$.
\end{center}

Taking P as P'. Because of the following derivation, \eqref{eqc.5} holds.
$$
  \frac{\Theta; \Gamma' \vdash s_{1} : P_{1} \ \ \ \ P_{1} \le P}
  {\Theta; \Gamma' \vdash s_{1} : P}
  \Rtab \mbox{(T-Sub)}
$$

And by the definition of $\xlongrightarrow{\tau}^{*}$, \eqref{eqc.6} holds. \\

\noindent Case: $\langle H, R, f(x) , n \rangle \rightarrow  \langle H', R', [x'/x]s_{1}, n  \rangle $ where the body of function $f(x)$ is $s_{1}$. we can see $f(x)$ and $s_{1}$ as $s$ and $s'$ respectively. \\

From the assumption, we already have
\begin{center}
$\Gamma \vdash f(x) : P$, and $OK_{n}(P)$.
\end{center}

By the inversion lemma and \eqref{eq6.1}, we have
\begin{center}
$P_{1} \le P$, and $\Gamma \vdash s_{1} : P_{1}$.
\end{center}

From the definition of subtyping and $P_{1} \xlongrightarrow{0} P_{1}$, we get $\exists P''$ s.t.
\begin{center}
$P \xlongrightarrow{0} P''$, and $P_{1} \le P''$.
\end{center}

Taking the $P''$ to be $P'$, then we get $P \xlongrightarrow{0} P'$.\\
And by using the subtyping rule with premises \eqref{eq6.4} and  \eqref{eq6.6}, we have
$$
\frac{\Gamma \vdash s_{1} : P_{1} \ \ \ \ P_{1} \le P'}{\Gamma \vdash s_{1} : P'}
$$

Therefore we prove that $\Gamma \vdash s' : P'$ where $s'$ is the command $s_{1}$.

\noindent Finally to prove $OK_{n'}P'$ that is, $\sharp_{m}(P')-\sharp_{f}(P') \le n'$. And  proceed by case analysis.\\

\noindent Case $P =  \SKIP;P'$\\

According to rule E-Skip, we should prove  $\sharp_{m}(P')-\sharp_{f}(P') \le n'$ where $n'$ is $n$.

Because we have
\begin{eqnarray*}
  OK_{n}(P)  & & =  OK_{n}(\SKIP;P')\\
  & & \Rightarrow \sharp_{m}(\SKIP;P') - \sharp_{f}(\SKIP;P') \le n \\
  & & \Rightarrow \sharp_{m}(P') - \sharp_{f}(P') \le n \
\end{eqnarray*}
Then it is proved. \\

\noindent Case $P = \Malloc;P'$ \\

Here according to rule E-Malloc, we know the $n'$ is $n-1$.

Therefore we should prove $\sharp_{m}(P') - \sharp_{f}(P') \le n-1$
\begin{eqnarray*}
  OK_{n}(P)&& =  OK_{n}(\Malloc;P')\\
  &&\Rightarrow \sharp_{m}(\Malloc;P') - \sharp_{f}(\Malloc;P') \le n \\
  &&\Rightarrow  \sharp_{m}(P') + 1 - \sharp_{f}(P') \le n\\
  &&\Rightarrow  \sharp_{m}(P')  - \sharp_{f}(P') \le n-1\\
\end{eqnarray*}

Then it is proved.\\

\noindent Case $P = \Free;P'$ \\

According to rule E-Free, we should prove $\sharp_{m}(P') - \sharp_{f}(P') \le n+1$.

\begin{eqnarray*}
  OK_{n}(P)  & & =  OK_{n}(\Free;P')\\
  & &\Rightarrow  \sharp_{m}(\Free;P') - \sharp_{f}(\Free;P') \le n \\
  & & \Rightarrow \sharp_{m}(P')  - \sharp_{f}(P') - 1  \le n\\
  & & \Rightarrow \sharp_{m}(P')  - \sharp_{f}(P') \le n+1\\
\end{eqnarray*}

Then it is proved. \\

\noindent Case $P = P_{1};P_{2}$\\

To prove it by contradiction.

Suppose that $OK_{n'}(P_{1}';P_{2})$ does not hold. Then we have 
$P_{1};P_{2} \xlongrightarrow{\alpha} P_{1}';P_{2} \xlongrightarrow{\exists \sigma} Q$, $s.t.$ $\sharp_{m}(\sigma) - \sharp_{f}(\sigma) > n'$\\

From the premise $OK_{n}(P) = OK_{n}(P_{1};P_{2})$, we get 
\setcounter{equation}{0}
\begin{align}
  &  \sharp_{m}(\alpha \cdot \sigma) - \sharp_{f}(\alpha \cdot \sigma) \le n \label{eqok1.1}
\end{align}

From \eqref{eqok1.1}, we get
\begin{align}
\sharp_{m}(\alpha) + \sharp_{m}(\sigma) - \sharp_{f}(\alpha) - \sharp_{f}(\sigma) \  \label{eqok1.2}
\end{align}
and with
$$
   n'=\left\{
   \begin{aligned}
     n + 1, && \alpha = \Free \\
     n - 1,  && \alpha = \Malloc  \\
     n ,      && otherwise
   \end{aligned}
   \right.
$$
Therefore, we get \\
$n' + \sharp_{m}(\alpha) - \sharp_{f}(\alpha) < \sharp_{m}(\alpha) + \sharp_{m}(\sigma) - \sharp_{f}(\alpha) - \sharp_{f}(\sigma) \le n $ \\
When $\alpha = \Free$, we get that $n + 1 - 1 < n$\\
When $\alpha = \Malloc$, we get that $ n - 1 + 1 < n $ \\
When $ \alpha = other$,  we  get that $ n < n $ \\
 All of the three cases are equal to $n$. Therefore we get the contradiction.

\end{document}
