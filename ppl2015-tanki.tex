%\documentclass[japanese]{jssst_ppl} %%
 \documentclass[english]{jssst_ppl} %% English
% \documentclass[japanese,draft]{jssst_ppl} %% You can use the draft option
\usepackage{bcprules,amsmath,amsthm,amssymb,amsfonts,extarrows,geometry,amsopn,enumerate,xcolor,url}
\usepackage[sort]{cite}
\usepackage{graphicx}
\usepackage{boxedminipage}
\usepackage{url}
\usepackage{multirow}
% Write \fulltrue after \iffull for the full version; otherwise, write
% \fullfalse.
\newif\iffull\fullfalse
% Write \finaltrue after \iffinal for the final version; otherwise,
% write \finalfalse.
\newif\iffinal\finalfalse

% set some new commands %
\newcommand\tB{\;|\;}
\newcommand\LET{\mathbf{let}\;}
\newcommand\FREE{\mathbf{free(x)}\;}
\newcommand\IN{\mathbf{in}\;}
\newcommand\SKIP{\mathbf{skip}}
\newcommand\Rtab{\; \; \; \;}
\newcommand\NULL{\mathbf{null}}
\newcommand\IFNULL{\mathbf{ifnull}\;}
\newcommand\THEN{\mathbf{then}\;}
\newcommand\ELSE{\mathbf{else}\;}
\newcommand\Lcc{\left(}
\newcommand\Rcc{\right)}
\newcommand\Lfc{\left\{}
\newcommand\Rfc{\right\}}
\newcommand\Lb{\left[}
\newcommand\Rb{\right]}
\newcommand\coma{,\;}
\newcommand\MALLOC{\mathbf{malloc()}\;}
\newcommand\Malloc{\mathbf{malloc}}
\newcommand\Free{\mathbf{free}}
\newcommand\Cirx{(x)}
\newcommand\dtb{\;\;\ \;\;\ \;\;\ \;\;\  }
\newcommand\set[1]{\{#1\}}
\newcommand\VAR{\mathbf{Var}}
\newcommand\OK{\mathit{OK}}
\newcommand\COL{\!:\!}
\newcommand\TSEQ{;\!}
\newcommand\TSKIP{\mathbf{0}}
\renewcommand\rn[1]{\textsc{{#1}}}
\newcommand\OVERFLOW{\mathbf{OutOfMemory}}
\newcommand\DOM{\mathbf{Dom}}
\newcommand\FUNTYPE{\varphi}
% \newcommand\set[1]{\{{#1}\}}
\newcommand\bs{\backslash}
\newcommand\MEMEX{\mathbf{MemEx}}

\newenvironment{pfof}[1]{%
  {\it Proof of {#1}}:%
}{\qed}


\newtheorem{theorem}{Theorem}[section]
\newtheorem{lemma}[theorem]{Lemma}
\newtheorem{proposition}[theorem]{Proposition}
\newtheorem{corollary}[theorem]{Corollary}
\newtheorem{myDef}{Definition}
\newtheorem{remark}{Remark}[section]

\theoremstyle{definition}
\newtheorem{exmp}{Example}[section]

\newenvironment{nospaceflalign*}
 {\setlength{\abovedisplayskip}{0pt}\setlength{\belowdisplayskip}{0pt}%
  \csname flalign*\endcsname}
 {\csname endflalign*\endcsname\ignorespacesafterend}

\iffinal
\newcommand\todo[1]{}
\else
\newcommand\todo[1]{{\textcolor{red}{\bf KS: {#1}}}}
\fi

\title{A Behavioral Type System for \\ Memory-Leak Freedom}
\author{Qi Tan, Kohei Suenaga, and Atsushi Igarashi}
\inst{
    Department of Communications and Computer Engineering\\
    Graduate School of Informatics\\
    Kyoto University\\
\texttt{\{tanki,ksuenaga,igarashi\}@fos.kuis.kyoto-u.ac.jp}
%\medskip\par%
}
\begin{document}
\maketitle
\begin{abstract}
We propose a type system to abstract the behavior of a program under
manual memory management. Our type system uses sequential processes as
types where each action corresponds to an allocation and a
deallocation of a fixed-size memory block. The abstraction obtained by
our type system makes it possible to estimate an upper bound of memory
consumption of a program. Hence, by using our type system with another
safe-memory-deallocation analysis proposed by Suenaga and Kobayashi,
we can verify memory-leak freedom even for nonterminating programs.
We define the type system, prove type soundness, and show a type
reconstruction procedure that estimates an upper bound of memory
consumption using an off-the-shelf model checker.
\end{abstract}

\section{Introduction}
\label{sec:introduction}

Dynamic memory management is a crucial function of programming
languages.  Correct allocation and disposal of memory cells are
fundamental for software to be reliable.

%% This paper proposes a type-based approach to static verification of
%% memory-leak freedom that works for nonterminating programs.  Although
%% memory leaks are relatively more serious in nonterminating programs
%% (e.g., operating systems and Web servers) than terminating ones, the
%% analyses proposed so far put less emphasis to nontermination; they
%% rather verify \emph{partial} memory-leak freedom: if a program
%% terminates, then all the allocated memory cells are deallocated.  We
%% say a program is \emph{totally} memory-leak free if it does not
%% consume unbounded amount of memory during execution.
%% =======

Correct dynamic memory management is challenging if a programming
language is equipped with manual memory management primitives (e.g.,
\texttt{malloc} and \texttt{free} in the C language).  With such
primitives, one can write a program that accesses to deallocated
memory cells (i.e., accesses to dangling pointers) and that does not
dispose memory cells even when they become unnecessary (i.e., memory leaks).
In order to detect bugs related to such primitives at the early stage
of software development, many static verification methods have been
proposed~\cite{DBLP:conf/aplas/SuenagaK09,DBLP:conf/pldi/HeineL03,DBLP:conf/sigsoft/XieA05,DBLP:journals/scp/SwamyHMGJ06,DBLP:conf/sas/OrlovichR06,DBLP:conf/issta/SuiYX12}.

%% This paper proposes a type-based approach to static verification of
%% memory-leak freedom that works for nonterminating programs.  Although
%% memory leaks are more serious in nonterminating programs (e.g.,
%% operating systems and Web servers) than terminating ones, the analyses
%% proposed so far put less emphasis to nontermination; they rather
%% verify \emph{partial} memory-leak freedom: If a program terminates,
%% then all the allocated memory cells are deallocated.  We say a program
%% is \emph{totally} memory-leak free if it does not consume unbounded
%% amount of memory during execution.

The analyses proposed so far put less emphasis to nonterminating
programs although memory leaks in such programs are more serious
(e.g., operating systems and Web servers) than in terminating ones.
They rather verify \emph{partial} memory-leak freedom: All the
allocated memory cells are eventually deallocated \emph{if a program
  terminates}.  We say a program is \emph{totally} memory-leak free if
it does not consume unbounded amount of memory during execution.

In real-world programs, the total memory-leak freedom is a very
important property for programs, such as the OS and Web servers
mentioned above which run for a long time and allocate memory cells
over time, digital image processing which may allocate memory cell
frequently, embedded systems like medical instruments and airplanes
appliances which have limited memory but process unbounded stream of
inputs. Without this property, these programs may eventually crash and
cause serious problems due to lack of memory cells. If the number of
of memory consumption can be determined by static analysis, the OS can
allocate a mount of memory for a program before the execution so
that allocating operation never fails.

\begin{exmp}\label{ex:ex1}
%% Functions $h$ and $h'$ shown in Figure~\ref{ex:np} describe
%% memory-leak freedom and memory leaks in nonterminating
%% programs. Function $h$ requires two memory cells at most, whereas
%% function $h'$ requires unbounded number of memory cells to be
%% executed.
\begin{figure}[h]
1  \Rtab $h()$= \dtb \dtb\dtb\Rtab$h'()$= \\
2  \dtb $\LET \; x = \MALLOC  \; \IN$ \dtb \Rtab$\LET \; x = \MALLOC  \; \IN$\\
3  \dtb $\LET \; y = \MALLOC  \; \IN$ \dtb \Rtab$\LET \; y = \MALLOC  \; \IN$\\
4  \dtb $\Free(x)$; $\Free(y) $;\;$h()$ \dtb \Rtab\ \ $h'()$; $\Free\Cirx$; \ $\Free(y)$
\caption{Memory leaks in nonterminating programs.}
\label{ex:np}
\end{figure}
Figure~\ref{ex:np} describes partial and total memory-leak freedom.
Both \(h\) and \(h'\) are partially memory-leak free because they do
not terminate.  The function \(h\) is totally memory-leak free since
it consumes at most two cells\footnote{We assume that every memory
  cell allocated by \(\Malloc\) is fixed size. Because variable-length
  memory cell is ubiquitous, we plan to deal with it in the future.}.
However, the function \(h'\), when it is invoked, consumes unbounded
number of memory cells; hence \(h'\) is not totally memory-leak free.
\end{exmp}

As a first step to the verification of total memory-leak freedom, this
paper proposes a \emph{behavioral type
  system}~\cite{DBLP:journals/lmcs/KobayashiSW06,DBLP:journals/tcs/IgarashiK04,DBLP:conf/esop/HondaVK98}
for a programming language with manual memory-management primitives.
Our type system approximates the behavior of a program by a sequential
process whose actions represent memory allocation and deallocation.
For example, our type system can assign a type
\(\mu\alpha.\Malloc\TSEQ\Malloc\TSEQ\Free\TSEQ\Free\TSEQ\alpha\) to
the function \(h\) above.  This type expresses that \(h\) can allocate
a memory cell twice, deallocate a memory cell twice, and then iterate
this behavior.  The type assigned to \(h'\) is
\(\mu\alpha.\Malloc\TSEQ\Malloc\TSEQ\alpha\TSEQ\Free\TSEQ\Free\),
which expresses that \(h'\) can allocate a memory cell twice, call
itself recursively, and then deallocate a memory cell twice.  Hence,
by inspecting the inferred types (by using off-the-shelf model
checkers, for example), one can estimate the upper bound required to
execute \(h\) and \(h'\).

%% This way may be better than directly modelchecking the original
%% program, because we can ignore any other statements except allocation
%% and deallocation. We are now investigating some practical programs to
%% do experiments for this, and experiments are not included in our
%% paper.

One may not observe, in the example above, much difference between
applying a model checker to the original programs and to the assigned
behavioral types.  However, we expect the latter is faster than the
former in many programs because a behavioral type focuses on the
actions related to allocations and deallocations, abstracting away the
other actions. %%  Although the current paper deals with only the
%% theoretical aspect, we plan to conduct experiment with actual programs
%% in future.

Notice that our type system alone does not prevent incorrect usage of
\(\Malloc\) and \(\Free\).  Indeed, as observed from the type assigned
to \(h\) and \(h'\) above, our types include information only about
the number and the order of allocations, deallocations, and recursive
function calls; hence, the type system does not guarantee, for
example, that there is no access to a deallocated cell.  For such
properties, we can use other no-illegal-access verifiers~\cite{DBLP:conf/aplas/SuenagaK09}.

The rest of this paper is structured as
follows. Section~\ref{sec:language} introduces a simple imperative
language and the operational semantics of the
language. Section~\ref{sec:typesystem} introduces the behavioral type
system and states its type soundness. Section~\ref{sec:reconstruction}
describes a type reconstruction
procedure. Section~\ref{sec:experiment} shows the preliminary
experiments. Section~\ref{sec:relatedwork} discusses the related
work. Section~\ref{sec:conclusion} concludes the paper.  The proof of
type soundness and the detailed definition of the type reconstruction
procedure are in the full version~\cite{fullversion}.

%% \subsection{Motivation and Problems}

%% Manual memory management primitives (e.g., \texttt{malloc} and
%% \texttt{free} in C) often cause that forgetting to deallocate memory
%% cells after use, which we call \emph{memory leaks}. It can diminish
%% the performance of the computer by reducing the amount of available
%% memory cells. Memory leaks may not be serious or even detectable by
%% normal means. Normal memory used by an application is released when
%% application terminates. This means that a memory leak in a program
%% that only runs for a short time may not be noticed and is rarely
%% serious. However, in the real-world programs, nonterminating programs
%% such as Web servers and operating systems are very important. If
%% memory leaks in such nonterminating programs, eventually, too much of
%% the available memory cells may become allocated and all or part of the
%% system stops working correctly~\cite{wiki:xxx}.

%% Most of the static analysis of memory-leak freedom proposed so
%% far~\cite{DBLP:conf/aplas/SuenagaK09,
%%   DBLP:conf/sas/OrlovichR06,DBLP:conf/pldi/HeineL03,DBLP:conf/sigsoft/XieA05,DBLP:journals/scp/SwamyHMGJ06}
%% deal with only \emph{partial memory-leak freedom}: if a program
%% terminates, allocated memory cells are all deallocated at the end. For
%% example, the type system by Suenaga and
%% Kobayashi~\cite{DBLP:conf/aplas/SuenagaK09}, which is called
%% \textbf{SK} type system in our paper, guarantees that (1) a well-typed
%% program does not conduct illegal accesses and that (2) after execution
%% of a well-typed program, all the memory cells are deallocated.

%% We tackle the problem of verifying \emph{total memory-leak freedom} in
%% this paper.\footnote{we often write memory-leak freedom for
%%   \emph{total} memory-leak freedom.} By a program being totally
%% memory-leak free, we mean that the program requires only a bounded
%% amount of memory even if it does not terminate.


%% From examples in Figure~\ref{ex:np}, we notice that a possible way to
%% guarantee memory-leak freedom for nonterminating programs is to check
%% whether the number of allocations and deallocations is balanced before
%% recursive call. If balanced, the program consumes bounded number of
%% memory cells, say function $h$ allocates two memory cells and
%% deallocates them before recursive call, it consumes two memory cells
%% at most once; if not balanced, as function $h'$ shows, it allocates
%% two memory cells but does not deallocate them before recursive call,
%% which consumes unbounded number of memory cells as time goes by.

%% To estimate the upper bound of memory consumption, we count the number
%% of allocations and deallocations by a behavioral type system. The
%% behavioral type system is mainly used to abstract the behavior of a
%% program and heavily used in the context of concurrent program
%% verification~\cite{DBLP:journals/lmcs/KobayashiSW06,DBLP:journals/tcs/IgarashiK04,DBLP:conf/esop/HondaVK98}. The
%% behavior of a program in our paper is abstracted as CCS-like
%% processes~\cite{DBLP:journals/iandc/MilnerPW92a}. For example, the
%% behavior of function $h$ is as $\mu
%% \alpha. \Malloc;\Malloc;\Free;\Free;\alpha$ which denotes it executes
%% $\Malloc$ twice, $\Free$ twice and calls itself. Similarly, the
%% behavior of $h'$ is abstracted as $\mu
%% \alpha. \Malloc;\Malloc;\alpha;\Free;\Free$.

%% One thing we should consider is like the function $f$ shown in Figure~\ref{ex:bbd}. The behavior of function $f$ is that $\Malloc$ twice, $\Free$ twice and calling itself. The abstracted behavior is $\mu \alpha. \Malloc;\Malloc;\Free;\Free;\alpha$. The number of allocation and deallocation is balanced before recursive call; the function is safe in our behavior type system, but it causes double frees: the variable $x$ is deallocated twice.
%% \begin{figure}[h]
%% 1  \Rtab\dtb\dtb $f(x)$= \\
%% 2  \dtb \dtb\dtb$\LET \; x = \MALLOC  \; \IN$ \\
%% 3  \dtb \dtb\dtb$\LET \; y = \MALLOC  \; \IN$ \\
%% 4  \dtb \dtb\dtb$\Free(y)$; $\Free(y) $;\;$f(x)$
%% \caption{balanced but double free}
%% \label{ex:bbd}
%% \end{figure}

%% Thanks to the \textbf{SK} type system, proposed by Suenaga and Kobayashi, it guarantees no double frees or illegal access to a deallocated memory cell. By combining the \textbf{SK} type system, our behavior type system can ignore the relationship between variables and pointers, so to estimate the upper bound of consumption memory cells according to the abstraction of the behavior of programs is sound.

%% \subsection{Overview of the algorithm}

%% \begin{figure}
%%  \centering
%% \includegraphics[width=10cm]{overview.jpg}
%% \caption{Overview of the algorithm}
%% \label{fig:ov}
%% \end{figure}


%% Hence, by using our type system with \textbf{SK} type system, we can verify memory-leak freedom even for nonterminating programs.
%% %% >>>>>>> 9a92815042ab957eb9b2d406af7a45d4abb729b3

\paragraph{Notation} We write \(\vec{X}\) for a finite sequence of
\(X\).  We write \([\vec{x'}/\vec{x}]s\) for the term obtained by
replacing every free occurrence of \(\vec{x}\) in \(s\) with
\(\vec{x'}\).  We write \(\DOM(f)\) for the domain of the map \(f\).
For a map \(f\), we write \(f \set{x \mapsto v}\) and \(f \bs x\) for
the maps defined as follows:
\[
\begin{array}{rcl}
f \set{x \mapsto v} (w) &=&
\left\{
\begin{array}{ll}
v & \mbox{if \(x = w\)}\\
f(w) & \mbox{otherwise.}
\end{array}
\right.\\
(f \bs x)(w) &=&
\left\{
\begin{array}{ll}
\mbox{undefined} & \mbox{if \(x = w\)}\\
f(w) & \mbox{otherwise.}
\end{array}
\right.
\end{array}
\]


\section{Language \(\mathcal{L}\)}\label{sec:language}

This section gives the definition of language \(\mathcal{L}\), an
imperative language with memory allocation/deallocation primitives.
The language is essentially the same as one used by Sueanga et
al.~\cite{DBLP:conf/aplas/SuenagaK09}, where they propose a
type-based analysis for partial memory-leak freedom analysis.

The syntax of language \(\mathcal{L}\) is as follows.
\begin{eqnarray*}
  x,y,z,\dots \mbox{ (variables)} &\in& \VAR\\
  s \mbox{ (statements)} & ::= &  \SKIP \mid s_{1};s_{2} \mid *x \leftarrow y \mid \Free(x) \\
  & \mid & \LET x = \MALLOC \IN s \mid \LET x = \NULL\ \IN s  \\
  & \mid & \LET x = y \; \IN s \mid   \LET x = *y \; \IN s \\
  & \mid & \IFNULL(x) \; \THEN s_{1}\; \ELSE s_{2} \mid f(\vec{x})\\
  d \mbox{ (proc. defs.)} & ::= & \set{f \mapsto (x_1,\dots,x_n)s}\\
  D \mbox{(definitions) } &::=& \langle d_1 \cup \dots \cup d_n \rangle\\
  P \mbox{ (programs)} &::=& \langle D, s \rangle\\
\end{eqnarray*}

The language is equipped with procedure calls, dynamic memory
allocation and deallocation, and memory accesses with pointers.
\(\VAR\) is a countably infinite set of \emph{variables}.  The
statement \(\SKIP\) does nothing.  The statement \(s_1;s_2\) executes
\(s_1\) and \(s_2\) sequentially.  The statement \(*x \leftarrow y\)
writes \(y\) to the memory cell that \(x\) points to.  The statement
\(\LET x = e\ \IN s\) evaluates the expression \(e\), binds \(x\) to
the result, and executes \(s\).  The expression \(\Malloc()\)
allocates a new memory cell and evaluates to the pointer to the cell.
The expression \(\NULL\) evaluates to the null pointer.  The
expression \(y\) evaluates to its value.  The expression \(*y\)
evaluates to the value in the memory cell that \(y\) points to.  The
statement \(\IFNULL(x)\ \THEN\ s_1\ \ELSE\ s_2\) executes \(s_1\) if
\(x\) is \(\NULL\) and executes \(s_2\) otherwise.  The statement
\(f(\vec{x})\) calls procedure \(f\) with arguments \(\vec{x}\).

A procedure definition ranged over by \(d\) is a map from a procedure
name to an abstraction of the form \((\vec{x})s\).  We assume that
there is no arity mismatch between function definitions and function
calls.  We use a metavariable \(D\) for a set of function definitions
\(d_1 \cup \dots \cup d_n\).  A program is a pair of function
definitions \(D\) and a main statement \(s\).


%% A program is a pair $(D,s)$, where $D$ is the set of definition.\\ The
%% command $\SKIP$ does nothing. The command $s_{1};s_{2}$ is executed as
%% a sequence, first executing $s_{1}$ and then $s_{2}$. The command $*x
%% \leftarrow y$ updates the content of the memory cell which is pointed
%% by pointer $x$ with value $y$. The command $\Free \Cirx$ deallocates
%% the memory cell which is pointed by a pointer $x$. Then command $\LET
%% x = e \; \IN s$ first evaluates the expression $e$ and binds the
%% return value of $e$ to $x$ and then executes statement $s$. The
%% command $\LET x = \Malloc \; \IN s$ first allocates a memory cell to a
%% pointer $x$ and then executes the statement $s$. The command $\LET x =
%% \NULL \IN s$ first allocates a null pointer to $x$ and then executes
%% $s$. The command $\LET x = y \; \IN s$ assign the pointer $y$ to $x$,
%% so the pointer $x$ and $y$ are said aliases for the same memory cell,
%% and then executes statement $s$. The command $\LET x = *y \; \IN s$
%% transfers a part of memory cells pointed by $y$ and then executes
%% statement $s$. The command $\IFNULL(x) \; \THEN s_{1} \; \ELSE s_{2}$
%% denotes that executing statement $s_{1}$ if pointer $x$ is a null
%% pointer, otherwise executing statement $s_{2}$. The command
%% $f(\vec{x})$ is a function call in which $\vec{x}$ denotes mutually
%% distinct variables like \{$x_{1}, \dots, x_{n}$\}. The notation $d$
%% denotes the definition of function $f(\vec{x})$ which has a body of
%% statement $s$. And examples are described by this syntax you can see
%% in Figure 1 and Figure 2.

\subsection{Operational semantics}
\label{sec:languageSemantics}

This section introduces the operational semantics of \(\mathcal{L}\).
\(\mathcal{H}\) is a countably infinite set of \emph{locations} ranged
over by \(l\).

We represent a state of computation by \emph{configuration} \(\langle
H, R, s, n \rangle\).  A configuration consists of the following four
components:
\begin{itemize}
\item \(H\), a \emph{heap}, is a finite mapping from \(\mathcal{H}\)
  to \(\mathcal{H} \cup \set{\NULL}\);
\item \(R\), an \emph{environment}, is a finite mapping from \(\VAR\)
  to \(\mathcal{H} \cup \set{\NULL}\);
\item \(s\) is the statement that is being executed; and
\item \(n\) is a natural number that represents the number of
  available memory cells.
\end{itemize}
\(n\) in a configuration is later used to formalize memory leaks
caused by nonterminating program.

The operational semantics is given by relation \(\langle H, R, s, n
\rangle \xlongrightarrow{\rho}_D \langle H', R', s', n' \rangle\)
where \(\rho\), an \emph{action}, is \(\Malloc\), \(\Free\), or
\(\tau\).  The action \(\Malloc\) expresses an allocation of a memory
cell; \(\Free\) expresses a deallocation; \(\tau\) expresses the other
actions.  We often omit \(\tau\) in \(\xlongrightarrow{\tau}_D\).  We
use a metavariable \(\sigma\) for a finite sequence of actions
\(\rho_1\dots\rho_n\).  We write
\(\xlongrightarrow{\rho_1\dots\rho_n}_D\) for
\(\xlongrightarrow{\rho_1}_D\xlongrightarrow{\rho_2}_D\dots\xlongrightarrow{\rho_n}_D\).
We write \(\xLongrightarrow{\rho}_D\) for
\(\xlongrightarrow{}_D^*\xlongrightarrow{\rho}_D\xlongrightarrow{}_D^*\).
We write \(\xLongrightarrow{\rho_1\dots\rho_n}_D\) for
\(\xLongrightarrow{\rho_1}_D\dots\xLongrightarrow{\rho_n}_D\).

%% Because we want to estimate the number of available memory cells at
%% every operation step, we extend the triple $\langle H\coma R\coma s
%% \rangle$ that is represented as run-time state in previous type system
%% to a quadruple $\langle H\coma R\coma s\coma n \rangle$ in our
%% paper. The introduced notation $n$ denotes the number of available
%% memory cells, a nature number. When executing the operation $\Malloc$,
%% the number of available memory cells will decrease 1, which is denoted
%% as ($n - 1$); when executing the operation $\Free$, the number of
%% available memory cells will increase 1, which is denoted as ($n +
%% 1$). The notation $H$, which models heap memory, is a mapping from
%% finite subset of $\mathcal{H}$ to $\mathcal{H}$ $\cup$ \{$null$\},
%% where $\mathcal{H}$ represents the set of \emph{heap addresses}. $R$,
%% which models registers, is a mapping from finite set of variables to
%% $\mathcal{H}$ $\cup$ \{$null$\}.

Figure~\ref{fig:transitionRules} defines the relation
%% <<<<<<< HEAD
%% \(\xlongrightarrow{\rho}_D\). In the rule for \(\Free\), the \(\Free\) command correctly disposes  a memory cell which is pointed to by a pointer \(x\), and after doing \(\Free\) action, the number of available memory cells increases one.  In the rule for \(\Malloc\), the \(\Malloc()\) command need at least one cell for it to allocate a new memory cell \(l\) which is pointed to by the pointer \(x\), and after performing the \(\Malloc\) action, the number of available memory cells will decrease one.  The \(\bf MemEx\) for accessing null pointers or illegal accessing to a deallocated memory cell, and \(\OVERFLOW\) for performing allocation after running out of memory cells.
%
%% we require
%% that a memory cell has been allocated before deallocating it, and
%% after deallocating the memory cell from the heap \(H\), the number
%% \(n\) of available memory cells should increase one.  In the rule for
%% \(\Malloc\), the number of available memory cells should be positive;
%% the \(\Malloc\) command allocate e cell for allocates a new memory
%% cell \(l\) of which contents \(v\) can be any value in \(\mathcal{H}
%% \cup \set{\NULL}\) and reduces the number of available memory cells by
%% one. The \(\bf NullEx\) for accessing null pointers or illegal accessing to a
%% deallocated memory cell, and \(\OVERFLOW\) for performing allocation
%% after running out of memory cells.
%=======
\(\xlongrightarrow{\rho}_D\).  We add explanations to several important
rules.
\begin{itemize}
\item \rn{Sem-Free}: Deallocation of a memory cell pointed to by \(x\)
  is expressed by deleting the entry for \(R(x)\) from the heap.  This
  action increments the number of available cells (i.e., \(n\)) by one
  (i.e., \(n+1\)).
\item \rn{Sem-Malloc} and \rn{Sem-OutOfMem}: Allocation of a memory
  cell is expressed by adding a fresh entry to the heap.  This action is
  allowed only if the number of available cells is positive; if the
  number is zero, then the configuration leads to an error state
  \(\OVERFLOW\).
\item \rn{Sem-*Exn}: These rules express an illegal access to memory.
  If such action is performed, then the configuration leads to
  exceptional state \(\MEMEX\).  This state \(\MEMEX\) is not seen as
  an erroneous state in the current paper, hence is not excluded by
  the type system in Section~\ref{sec:typesystem}.
\end{itemize}


%>>>>>>> 5ea5427003cd6520f97c31833e1440ce3dff0481

\begin{figure}[p]
\begin{minipage}{\textwidth}

%% =======
%% %% Transition rules are listed in Figure 3. In these rules, $f\{x\to v\}$ is defined as a function $f'$ such that $f'(y) = v$ if $x = y$, otherwise $f'(y) = f(y)$ and $y \in dom(f)$. There are three rules about $\mathbf{NullEx}$ which denotes accessing a null pointer, three rules about $\mathbf{Error}$ for accessing a deallocated memory cell, and one rule about $\mathbf{Error}$ which denotes allocating a memory cell when there is no memory space.
%% %\begin{figure}[h]
%% % Skip Command
%% Runtime state is represented by $\langle H, R, s, n \rangle$, where $H$ is a mapping from finite subset of $\mathcal{H}$ to $\mathcal{H}$ $\cup$ \{$null$\}, in which $\mathcal{H}$ represents the set of \emph{heap addresses}, intuitively, the $H$ models a heap memory; $R$, which models registers, is a mapping from finite set of variables to $\mathcal{H}$ $\cup$ \{$null$\}.

%% Operational semantics are defined by the relations: $\rightarrow_{D}, \xlongrightarrow{\Malloc}_{D}, and \xlongrightarrow{\Free}_{D} $, in which $\xlongrightarrow{\Malloc}_{D}, and \xlongrightarrow{\Free}_{D} $ denote performing a allocation and deallocation operation respectively, and $\rightarrow_{D}$ expresses that performing an internal action like $\SKIP$ or assignment.
%% >>>>>>> 9a92815042ab957eb9b2d406af7a45d4abb729b3

\begin{minipage}{0.5\textwidth}
\infax[Sem-Skip]
{\langle H, R, \SKIP;s, n \rangle
\longrightarrow_{D}
\langle H, R, s, n \rangle}
\end{minipage}
\begin{minipage}{0.5\textwidth}
\infrule[Sem-Seq]
{\langle H, R, s_1, n \rangle \xlongrightarrow{\rho}_{D} \langle H', R', s_1', n' \rangle}
{\langle H, R, s_1;s_2, n \rangle \xlongrightarrow{\rho}_{D} \langle H', R', s_1';s_2, n' \rangle}
\end{minipage}

\infrule[Sem-LetNull]
{x' \notin \DOM(R)}
{\langle H\coma R\coma  \LET x = \NULL \ \IN s , n \rangle
  \longrightarrow_{D}
  \langle H\coma R\Lfc x' \mapsto \NULL \Rfc \coma   \Lb x'/x \Rb s , n  \rangle }

\infrule[Sem-LetEq]
{x' \notin \DOM(R)}
{\langle H\coma R\coma \LET x = y \; \IN s , n \rangle
  \longrightarrow_{D}
  \langle H\coma R\Lfc x' \mapsto R(y) \Rfc \coma   \Lb x'/x \Rb s , n  \rangle }

\infrule[Sem-IfNullT]
{R(x) = \NULL}
{\langle H \coma R \coma \IFNULL\Cirx\ \THEN   s_{1}\ \ELSE\  s_{2} \coma  n \rangle
  \longrightarrow_{D}
  \langle H\coma R\coma s_{1} \coma n \rangle}

\infrule[Sem-IfNullF]
{R(x) \neq \NULL}
{\langle H \coma R \coma \IFNULL\Cirx\ \THEN  s_{1}\ \ELSE  s_{2} \coma  n \rangle
  \longrightarrow_{D}
  \langle H\coma R\coma s_{2} \coma  n \rangle}

\infrule[Sem-Call]
{D(f) = (\vec{y})s}
{ \langle H\coma R\coma  f(\vec{x}) , n \rangle
  \longrightarrow_{D}
  \langle H\coma R\coma  \Lb \vec{x}/\vec{y} \Rb s , n \rangle}

\infax[Sem-Assign]
{ \langle H \set{R(x) \mapsto v}, R, *x \leftarrow y , n \rangle \xlongrightarrow{}_{D}
  \langle H \Lfc R(x) \mapsto R(y) \Rfc , R, \SKIP , n \rangle }

\infrule[Sem-LetDeref]
{x' \notin \DOM(R) \andalso R(y) \in \DOM(H)}
{\langle H\coma R\coma  \LET x = *y \; \IN s , n \rangle
  \longrightarrow_{D}
  \langle H\coma R\Lfc x' \mapsto H(R(y)) \Rfc \coma   \Lb x'/x \Rb s , n  \rangle }

\infax[Sem-Free] 
{\langle H\set{R(x) \mapsto v}\coma R\coma \Free(x) , n \rangle \xlongrightarrow{\Free}_{D}
  \langle H\backslash R(x) \coma R \coma \SKIP , n+1 \rangle}

\infrule[Sem-Malloc]
{l \notin \DOM(H) \andalso n > 0}
{\langle H\coma R\coma  \LET x = \Malloc() \; \IN s , n \rangle
  \xlongrightarrow{\Malloc}_{D}
  \langle H \Lfc l \rightarrow v\Rfc \coma R\Lfc x' \rightarrow l \Rfc \coma   \Lb x'/x \Rb s , n-1  \rangle }


\begin{minipage}{0.5\textwidth}
\infrule[Sem-AssignExn]
{R(x) = \NULL \mbox{ or } R(x) \notin \DOM(H)}
{\langle H\coma R\coma  *x \leftarrow y , n \rangle
  \longrightarrow_{D} \MEMEX }
\end{minipage}
\begin{minipage}{0.5\textwidth}
\infrule[Sem-DerefExn]
{R(y) = \NULL \mbox{ or } R(y) \notin \DOM(H)}
{\langle H\coma R\coma  \LET x = *y \; \IN s, n \rangle
    \longrightarrow_{D} \MEMEX}
\end{minipage}

\infrule[Sem-FreeExn]
{R(x) = \NULL \mbox{ or } R(x) \notin \DOM(H)}
{\langle H\coma R\coma \Free(x) , n \rangle \xlongrightarrow{\Free}_{D} \MEMEX}

\infax[Sem-OutOfMem]
{ \langle H\coma R\coma \LET x = \Malloc() \ \IN s ,  0  \rangle    \xlongrightarrow{\Malloc}_{D}
     \OVERFLOW}

\end{minipage}

\caption{Operational semantics of \(\mathcal{L}\).}
\label{fig:transitionRules}
\end{figure}

%% Transition rules are listed in Figure 3. In these rules, $f\{x\to v\}$
%% is defined as a function $f'$ such that $f'(y) = v$ if $x = y$,
%% otherwise $f'(y) = f(y)$ and $y \in dom(f)$. There are three rules
%% about $\mathbf{NullEx}$ which denotes accessing a null pointer, three
%% rules about $\mathbf{Error}$ for accessing a deallocated memory cell,
%% and one rule about $\mathbf{Error}$ which denotes allocating a memory
%% cell when there is no memory space.
%% %\begin{figure}[h]
%% % Skip Command
%% Runtime state is represented by $\langle H, R, s, n \rangle$, where
%% Operational semantics is defined by the relations\\
%% $\rightarrow_{D}, \xlongrightarrow{\Malloc}_{D}, and \xlongrightarrow{\Free}_{D} $ defined by the rules in ....\\
%% Here, $\rightarrow_{D}$ expresses; ....\\

%% $$
%%     \frac{n \in \mathbb{N}}
%%             {\langle H\coma R\coma  \SKIP;s , n \rangle
%%               \longrightarrow_{D}
%%                 \langle H\coma R\coma   s , n \rangle }
%%      \Rtab \mbox{(E-Skip)}
%% $$
%% % Assignment
%% $$
%%      \frac{R(x) \in dom(H), n \in \mathbb{N}}
%%            {\langle H\coma R\coma  *x \leftarrow y , n \rangle
%%              \longrightarrow_{D}
%%              \langle H \Lfc R(x) \rightarrow R(y) \Rfc \coma R \coma   \SKIP , n  \rangle }
%%      \Rtab \mbox{(E-Assign)}
%% $$
%% % Free Command
%% $$
%%      \frac{R(x) \in dom(H) , n \in \mathbb{N}}
%%           {\langle H\coma R\coma  \FREE , n \rangle
%%             \xlongrightarrow{\Free}_{D}
%%             \langle H\backslash \Lfc R(x) \Rfc \coma R \coma   \SKIP , n+1  \rangle }
%%      \Rtab \mbox{(E-Free)}
%% $$
%% % Let Null Command
%% $$
%%      \frac{x' \notin dom(R)}
%%            {\langle H\coma R\coma  \LET x = \NULL \IN s , n \rangle
%%              \longrightarrow_{D}
%%              \langle H\coma R\Lfc x' \rightarrow \NULL \Rfc \coma   \Lb x'/x \Rb s , n  \rangle }
%%      \Rtab \mbox{(E-LetNull)}
%% $$
%% % Let Eq Command
%% $$
%%      \frac{x' \notin dom(R)}
%%             {\langle H\coma R\coma \LET x = y \; \IN s , n \rangle
%%               \longrightarrow_{D}
%%               \langle H\coma R\Lfc x' \rightarrow R(y) \Rfc \coma   \Lb x'/x \Rb s , n  \rangle }
%% \Rtab \mbox{(E-LetEq)}
%% $$
%% % Reference Command
%% $$
%%      \frac{x' \notin dom(R)}
%%             {\langle H\coma R\coma  \LET x = *y \; \IN s , n \rangle
%%               \longrightarrow_{D}
%%               \langle H\coma R\Lfc x' \rightarrow H(R(y)) \Rfc \coma   \Lb x'/x \Rb s , n  \rangle }
%%      \Rtab \mbox{(E-LetDref)}
%% $$
%% % Malloc (allocate) Command
%% $$
%%      \frac{h \notin dom(H)}
%%             {\langle H\coma R\coma  \LET x = \Malloc() \; \IN s , n \rangle
%%               \xlongrightarrow{\Malloc}_{D}
%%               \langle H \Lfc h \rightarrow v\Rfc \coma R\Lfc x' \rightarrow h \Rfc \coma   \Lb x'/x \Rb s , n-1  \rangle }
%% \Rtab \mbox{(E-Malloc)}
%% $$
%% % IFNULL T
%% $$
%%     \frac{R(x) = \NULL}
%%            {\langle H \coma R \coma \IFNULL\Cirx   \THEN   s_{1} \ELSE\  s_{2} \coma  n \rangle
%%            \longrightarrow_{D}
%%            \langle H\coma R\coma s_{1} \coma n \rangle}
%%     \Rtab \mbox{(E-IfNullT)}
%% $$
%% % IFNULL F
%% $$
%%     \frac{R(x) \neq \NULL}
%%            {\langle H \coma R \coma \IFNULL\Cirx \THEN  s_{1} \ELSE  s_{2} \coma  n \rangle
%%            \longrightarrow_{D}
%%            \langle H\coma R\coma s_{2} \coma  n, \rangle}
%%     \Rtab \mbox{(E-IfNullF)}
%% $$
%% % Function Call
%% $$
%%      \frac{f(\vec{y}) = s \in D}
%%             { \langle H\coma R\coma  f(\vec{x}) , n \rangle
%%                \longrightarrow_{D}
%%                \langle H\coma R\coma  \Lb \vec{x}/\vec{y} \Rb s , n \rangle}
%%       \Rtab \mbox{(E-Call)}
%% $$
%% % Error : access the null memory cell
%% $$
%%       \frac{R(x) = null}
%%             {\langle H\coma R\coma  *x \leftarrow y , n \rangle
%%               \longrightarrow_{D}
%%              \bf NullEx }
%%       \Rtab \mbox{(E-AssignNullError)}
%% $$
%% % ERROR : access the null memory cell
%% $$
%%       \frac{R(y) = null}
%%              {\langle H\coma R\coma  x = *y, n \rangle
%%                \longrightarrow_{D}
%%               \bf NullEx }
%%              \Rtab \mbox{(E-DrefNullError)}
%% $$
%% $$
%%      \frac{R(x) =  null }
%%            {\langle H\coma R\coma  \FREE , n \rangle
%%              \xlongrightarrow{\Free}_{D} \bf NullEx  }
%%       \Rtab \mbox{(E-FreeNullError)}
%% $$
%% % ERROR :
%% $$
%%      \frac{R(x) \notin dom(H) \cup \Lfc null \Rfc}
%%            {\langle H\coma R\coma   *x \leftarrow y,  n \rangle
%%              \longrightarrow_{D}
%%            \bf  Error }
%%     \Rtab \mbox{(E-AssignError)}
%% $$
%% % ERROR
%% $$
%%       \frac{R(y) \notin dom(H) \cup \Lfc null \Rfc}
%%            {\langle H\coma R\coma  \LET x  = *y \; \IN s, n \rangle
%%               \longrightarrow_{D}
%%                 \bf  Error }
%%       \Rtab \mbox{(E-DrefError)}
%% $$
%% %
%% $$
%%       \frac{R(x) \notin dom(H) \cup \Lfc null \Rfc}
%%             {\langle H\coma R\coma  \FREE , n \rangle
%%               \xlongrightarrow{\Free}_{D}
%%               \bf Error }
%%      \Rtab \mbox{(E-FreeError)}
%% $$
%%  % ERROR: no enough space
%% $$
%%       \langle H\coma R\coma \LET x = \Malloc() \ \IN s ,  0  \rangle
%%       \xlongrightarrow{\Malloc}_{D}
%%       \mathbf{Error}
%%       \Rtab \mbox{(E-MallocError)}
%% $$
%% $$
%%      \mathbf{Figure \; 3.} \;\;  \mbox{ Operational Semantics}
%% $$
%
%\caption{Operational Semantics.}
%\label{example:os}
     %\end{figure}

As mentioned in Section~\ref{sec:introduction}, we define a memory
leak to be consuming unbounded number of memory cells.  Formally, this
is defined as follows.

\begin{myDef}[Memory leaks]
\label{df:ml}
%% memory leaks:\\
%% if $\vdash \langle D, s \rangle : n$, then $\langle \emptyset, \emptyset, s, n \rangle \xlongrightarrow{\rho_{1}, \dots, \rho_{n}}_{D} \langle H, R, s', 0 \rangle \xlongrightarrow{\Malloc}_{D} Overflow $\\
%% memory-leak \ freedom: $\exists n \in \mathbb{N}$ s.t. $\langle \emptyset, \emptyset, s, n \rangle \nrightarrow^{*}Overflow$
A configuration \(\langle H, R, s, n \rangle\) \emph{goes overflow} if
there is \(\sigma\) such that \(\langle H, R, s, n \rangle
\xLongrightarrow{\sigma} \OVERFLOW\).  A program \(\langle D, s
\rangle\) \emph{requires more than \(n\) cells} if \(\langle
\emptyset, \emptyset, s, n \rangle\) goes overflow.  A program
\(\langle D, s \rangle\) is \emph{totally memory-leak free} if there
is a natural number \(n\) such that it does not require more than
\(n\) cells.
\end{myDef}


\section{Type system}
\label{sec:typesystem}

\subsection{Types}

Types are described as follows.

\[
\begin{array}{rlcl}
  P & (\mbox{behavioral types})&::=& {\bf 0} \tB P_{1};P_{2} \tB P_{1}+P_{2} \tB \Malloc \tB \Free \tB \alpha \tB \mu\alpha.P \\
  \FUNTYPE & (\mbox{function\ types}) &::=& P    \\
  \Gamma & (\mbox{variable type environments}) &::=& x_1, x_2, \dots, x_n\\
  \Theta & (\mbox{function type environments}) &::=& f_1\COL\FUNTYPE_1,\dots,f_n\COL\FUNTYPE_n.
\end{array}
\]

We use a metavariable \(P\) for \emph{behavioral types} which are
abstractions of the behavior of programs.  The type ${\bf 0}$
represents do-nothing behavior.  The type \(P_1;P_2\) represents an
sequential execution behavior as doing behavior \(P_1\) first and then
\(P_2\) behavior.  The type \(P_1 + P_2\) represents conditional
behavior which can proceed either \(P_1\) or \(P_2\) behavior.  The
type \(\Malloc\) is the behavior of allocating a memory cell exactly
once.  The type \(\Free\) represents deallocation behavior.  \(\alpha\) is a
type variable which is bounded to recursive constructor \(\mu
\alpha\). The type \(\mu \alpha.P\) represents recursive behavior,
which substitutes \(\mu\alpha.P\) itself for \(\alpha\) in behavior
\(P\); for example, given \(\Malloc;\alpha;\Free\) as \(P\), a
recursive type is of the form \(\mu \alpha. \Malloc;\alpha;\Free\) and
can go further as \(\Malloc;(\mu\alpha.\Malloc;\alpha;\Free);\Free\).

%%  $P_{1};P_{2}$ is for sequential execution. $P_{1} + P_{2}$
%% is abstracted as conditional. $\Malloc$ is the behavior of a statement
%% that allocates a memory cell exactly once. $\Free$ is for deallocating
%% memory cell exactly once. $\mu \alpha. P$ is a recursive type. For
%% example, the behavior of the body of function $h$ in Figure~\ref{ex:np} is
%% abstracted as $\mu
%% \alpha. \Malloc;\Malloc;\Free;\Free;\alpha$. $\alpha$ is a type
%% variable and bounded to the recursive constructor $\mu \alpha$.

The function type is described as $(\tau_{1}, \dots, \tau_{n})P$, which means a function receives some pointers as arguments and its body is abstracted as a behavioral type $P$.

% Semantics of Behavioral Types %
\subsection{Semantics of behavioral types}
The semantics of behavioral type are given by labeled transition system, and listed as follows:
\infax
{\mathbf{0};P \xlongrightarrow{\tau} P}
\infax
{\Malloc \xlongrightarrow{\Malloc} 0}
\infax
{\Free \xlongrightarrow{\Free} 0}
\infax
{\mu \alpha.P \xlongrightarrow{\tau}  [\mu \alpha . P/\alpha]  P}
\infax
{P_{1} + P_{2} \xlongrightarrow{\tau} P_{1}}
\infax
{P_{1} + P_{2} \xlongrightarrow{\tau} P_{2}}
\infrule
{P_{1} \xlongrightarrow{\alpha} P_{1}' }
{P_{1};P_{2} \xlongrightarrow{\alpha} P_{1}';P_{2}}

The notation $\rightarrow$ denotes that a behavioral type can be reduced by the internal action. Notation $\xlongrightarrow{\alpha}$ means that a behavioral type can be reduced by executing $\alpha$ actions, and the $\alpha$ here is $\{\Malloc, \Free\}$.

% Type Judgments
\subsection{Typing rules}
The type judgment of our type system is given by the form $\Theta ;
\Gamma \vdash s : P$. It intuitively reads that the behavior of $s$ is
$P$ under $\Theta$ and $\Gamma$, where \(\Theta\) is a mapping from function variables to function types and \(\Gamma\) is an environment which includes variables.%%  We design the type system so that
%% this type judgment implies the property: when $s$ executes
%% $\Malloc$(resp.$\Free$), then $P$ is equivalent to
%% $\Malloc;P'$(resp.$\Free;P'$) for a type $P'$ such that $\Theta;
%% \Gamma \vdash s': P'$, where $s'$ is the continuation of $s$. This
%% property guarantees the behavioral type soundly abstracts the upper
%% bound of the consumed memory cells.

%% Typing rules are presented in Figure~\ref{fig:typingrules}. In the rule for assignment, the behavior of  $*x \leftarrow y$ is $\bf 0$. The rule for $\Free$ represents that the behavior of $\Free \Cirx$ is $\Free$. The rule T-Malloc represents that $\LET x = \MALLOC \; \IN s$ has the behavior $\Malloc;P$, where $P$ is the behavior of statement $s$. The rule for function call represents that function $f$ has the behavior $P$ which is the behavior of the body of this function .
\(\vdash D \COL \Theta\) denotes that the set \(D\) of definitions has a type \(\Theta\).
\begin{myDef}[\(\sharp_{\rho}(\sigma)\)]
 \(\sharp_{malloc}(\sigma)\) and \(\sharp_{free}(\sigma)\) are functions to count the number of \(\Malloc\) and \( \Free \) actions in a sequence \( \sigma\) respectively.
 \label{df:sharf}
 \end{myDef}

%% begin{myDef}[OK(P)]
%%  \(OK_n(P)\) holds if a program executes sequence actions \( \sigma\) for any \(P'\) such that \(P\xLongrightarrow{\sigma}P'\), then the difference of \( \sharp_{malloc}(\sigma)\) and \( \sharp_{free}(\sigma)\) does not exceed the number \(n\).
%% \label{df:okn}
%% \
%%end{myDef}[OK_n(P)]
 \begin{myDef}
   \(OK_{n}(P) \iff \forall P',\; P \xlongrightarrow{\sigma}P'\) then \(\sharp_{malloc}(\sigma)-\sharp_{free}(\sigma)\le n\).
 \label{df:okn}
 \end{myDef}

 \begin{myDef}[Subtype]
$P_{1} \le P_{2}$ represents that $P_{1}$ is the subtype of $P_{2}$ and  means that: \\
(1) if $P_{1} \xlongrightarrow{\alpha}  P_{1}'$ then $\exists P_{2}' $ s.t. $P_{2} \overset{\text{$\alpha$}}{\Longrightarrow} P_{2}'$ and $ P_{1}' \le P_{2}' $\\
(2) if $P_{1} \rightarrow P_{1}'$ then $\exists P_{2}'$ s.t. $P_{2} \rightarrow^{*} P_{2}'$ and  $P_{1}' \le P_{2}'$\\
where $\overset{\text{$\alpha$}}{\Longrightarrow}$ means that: $\rightarrow^{*} \xlongrightarrow{\alpha} \rightarrow^{*}$.
\label{df:subtype}
\end{myDef}
%% At the end of $s$, memory leak freedom is guaranteed by $OK_{n}(P)$ ,where $P$ is the behavior of $s$. $OK_{n}(P)$ is defined as Definition~\ref{df:okn} in which $\sharp_{malloc}(\alpha)$ and $\sharp_{free}(\alpha)$ are functions to count the number of $\Malloc$ and $\Free$ actions in $\alpha$ respectively. This definition, intuitively, means at every running step the number of allocated memory cells will never go out of memory scope.
%% \begin{myDef}
%%   $OK_{n}(P) \iff \forall P',\; P \xlongrightarrow{\alpha}^{*}P'$ then $\sharp_{malloc}(\alpha)-\sharp_{free}(\alpha)\le n$.
%% \label{df:okn}
%% \end{myDef}

\begin{figure}[tp]
\begin{minipage}{\textwidth}

% Skip type
\infax[T-Skip]
{\Theta ; \Gamma \vdash \SKIP : \mathbf{0}}
% Sequence type
\infrule[T-Seq]
{\Theta ; \Gamma \vdash s_{1} : P_{1} \Rtab \Theta ; \Gamma \vdash s_{2} : P_{2}}
{\Theta ; \Gamma \vdash s_{1} ; s_{2} : P_{1};P_{2} }
% Assignment type
\infrule[T-Assign]
{\Theta ; \Gamma \vdash y  \Rtab \Theta ; \Gamma \vdash x }
{\Theta ; \Gamma \vdash *x \leftarrow y : \mathbf{0} }
% Free(deallocate) type
\infrule[T-Free]
{\Theta ; \Gamma \vdash x  }
{\Theta ; \Gamma \vdash \Free(x) : \Free}
% Malloc type
\infrule[T-Malloc]
{\Theta ; \Gamma,x \vdash s : P}
{\Theta ; \Gamma \vdash \LET x = \MALLOC \; \IN s  : \Malloc;P}
% Let eq type
\infrule[T-LetEq]
{\Theta ; \Gamma \vdash y   \Rtab \Theta ; \Gamma , x  \vdash s : P}
{\Theta ; \Gamma \vdash \LET x = y \; \IN s : P}
% Dereference type
\infrule[T-LetDref]
{\Theta ; \Gamma \vdash y  \Rtab \Theta ; \Gamma , x  \vdash s : P}
{\Theta ; \Gamma \vdash \LET x = *y \; \IN s : P}
% Let NULL type
\infrule[T-LetNull]
{\Theta ; \Gamma, x  \vdash s : P}
{\Theta ; \Gamma \vdash \LET x = \mathbf{null} \; \IN s : P}
% Subtyping
\infrule[T-Sub]
{\Theta ; \Gamma \vdash s : P_{1} \Rtab P_{1} \le P_{2}}
{\Theta ; \Gamma \vdash s : P_{2}}
 % ifnull s then s type
\infrule[T-IfNull]
{\Theta ; \Gamma \vdash x    \ \ \ \  \Theta ; \Gamma \vdash s_{1} : P \ \ \ \ \Theta ; \Gamma \vdash s_{1} : P}
{\Theta ; \Gamma \vdash \IFNULL(x) \; \THEN s_{1}\; \ELSE s_{2} : P}
% Function call type
\infrule[T-Call]
{ \Theta(f) = P}
{\Theta; \Gamma, \vec{x} : \vec{\tau} \vdash f(\vec{x}) : P}
% Program
\infrule[T-Program]
{\vdash D : \Theta \;\;\;\; \Theta; \emptyset\vdash s : P \Rtab OK_{n}(P)}
{\vdash (D, s) : n}

\end{minipage}
\caption{Typing Rules.}
\label{fig:typingrules}
\end{figure}


\subsection{Type soundness}

% This subsection describes some theorems and lemmas for type safety.

The following theorem is the main result of the current paper.  The
proof is in Appendix~\ref{sec:proof}.

\begin{theorem}\label{thm1}
If $\vdash \langle D, s \rangle : n$ for some \(n\), then \(\langle D,
s \rangle\) is totally memory-leak free.
\end{theorem}

% This theorem says that a well typed program guarantees memory leak
% freedom.

The proof is based on the following lemmas: preservation and lack of
immediate overflow.

\begin{lemma}[Preservation]
\label{lem:preservation}
If $OK_{n}(P)$, $\Theta; \Gamma \vdash s : P$ and $\langle H,R,s,n
\rangle \xlongrightarrow{\rho} \langle H',R',s', n' \rangle$, then
there exists $P'$ such that (1) $ \Theta; \Gamma \vdash s' : P'$, (2)
\(P \xLongrightarrow{\rho} P'\), and (3) \(OK_{n'}(P')\).
\end{lemma}

%% \begin{lemma}[Preservation $\mathbf{II}$]%\label{preser}
%% If $OK_{n}(P)$, $\Theta ; \Gamma \vdash s : P$ and $\langle H,R,s,n \rangle
%% \rightarrow \langle H',R',s', n'
%% \rangle$, then $\exists P'$ s.t. \\
%% (1) $\Theta; \Gamma \vdash s' : P'$\\
%% (2) $ P \rightarrow^{*} P'  $\\
%% (3) $OK_{n'}(P')$
%% \end{lemma}

\begin{lemma}[Lack of immediate overflow]
\label
If $\Theta; \Gamma \vdash s \COL P$, \(\vdash D \COL \Theta\), and
\(\OK_n(P)\), then $\langle H, R, s, n \rangle
\not\xlongrightarrow{\rho}$ for any \(\rho\).
\end{lemma}

\section{Type reconstruction}
\label{sec:reconstruction}

%% This section describes how to construct syntax directed typing rules
%% according to the typing rules of above section, and it provides an
%% algorithm which inputs statements and returns a pair containing
%% constraints and behavior types.

This section describes a type reconstruction procedure for the type
system in Section~\ref{sec:typesystem}\footnote{The procedure
  described here is not complete; see Section~\ref{sec:conclusion}.
  We could reject a program that is well-typed in the type system.}.
Since the procedure is essentially the same as one in Kobayashi et
al.~\cite{DBLP:journals/lmcs/KobayashiSW06}, we do not give a concrete
definition here.

The reconstruction procedure is a constraint-based one.  It generates,
given a program, constraints for the program to be well-typed by
constructing a derivation tree based on the rules in
Figure~\ref{fig:typingrules}.  A constraint is either a subtyping
constraint \(\alpha \ge P\), or \(\OK_\nu(\alpha)\), where \(\nu\) is
a symbol for an unknown natural number.  Since \rn{T-Prog} is the only
place where the condition \(\OK_n(P)\) is involved, the constraint set
includes exactly one constraint of the form \(\OK_\nu(\alpha)\).  The
concrete definition of the constraint generation is in the full
version~\cite{fullversion}.

By using the result obtained by Kobayashi et al.~\cite[Lemma
  3.8]{DBLP:journals/lmcs/KobayashiSW06}, a subtyping constraint
\(\alpha \ge P\) can be resolved by setting \(\alpha = \mu
\alpha. P\), which is the least solution of the constraint.  Hence,
the generated constraint set is reduced to a single constraint
\(\OK_{\nu}(P')\) for some behavioral type \(P'\).

By definition, \(\OK_{\nu}(P)\) holds if there is a natural number
\(n\) such that, for all \(\sigma\) and \(P'\), \(P
\xlongrightarrow{\sigma} P'\) implies \(\sharp_{\Malloc}(\sigma) -
\sharp_{\Free}(\sigma) \le n\).  In order to check this condition
soundly, we fix the upper bound of \(\nu\) to be checked.  Then,
\(\OK_{\nu}(P)\) can be checked by model-checking a system with
finitely many states; hence, model checkers like
CPAChecker~\cite{beyer2011cpachecker} and
SPIN~\cite{holzmann2004spin,ben2008principles} are
applicable.  More detailed description on how we apply a
model-checking algorithm to check \(\OK_\nu(P)\) is in the forthcoming
version.

%% \subsection{Constraint generation}

%% %% By syntax directed typing rules, the type reconstruction algorithm has
%% %% been designed as in Figure~\ref{fig:tyin}.

%% %% Function $PT_{v}(x) = (C,\emptyset)$ denotes that it receives a
%% %% pointer variable $x$ and outputs a pair consisting of
%% %% constraints set $C$ and an empty set. $PT_{\Theta}(s) = (C, P)$
%% %% is a mapping from statements to a pair -- constraints set $C$
%% %% and behavioral types $P$, where $\Theta$ is mapping from
%% %% function names to function types. $PT(\langle D,s \rangle) = (C,
%% %% P)$ denotes that it receives a program and produces a pair $(C,
%% %% P)$. $\alpha_{i}$ and $\beta$ are fresh type variables.



%% \subsection{Constraints reduction} $PT\langle D, s \rangle$
%% receives a program as argument and produces a pair which consists
%% of the subtype constraints on behavior types of the form $\alpha
%% \ge A$, and constraints of the form $OK_{n}(P)$. Thus, we obtain
%% the following constraints:\\ $$ \{ \alpha_{1} \ge A_{1}, \dots,
%% \alpha_{n} \ge A_{n}, OK_{n}(P)\} $$ Here, we can assume that
%% $\alpha_{1}, \dots, \alpha_{n}$ are pairwise-distinct, since
%% $\alpha \ge A_{1}$ and $\alpha \ge A_{2}$ can be replaced with
%% $\alpha \ge A_{1}+A_{2}$ by lemma 3.8 in
%% paper~\cite{DBLP:journals/lmcs/KobayashiSW06}. we can also assume
%% that $\{ \alpha_{1}, \dots, \alpha_{n} \}$ contains all the type
%% variables in the constraints, since otherwise we can always add the
%% tautology $\alpha \ge \alpha$. Each subtype constraints $\alpha \ge
%% A$ can be replaced by $\alpha \ge \mu \alpha. A$, by lemma
%% 3.8(4)~\cite{DBLP:journals/lmcs/KobayashiSW06} ( substituting
%% $\alpha$ for $B$ in this lemma). Therefore the above constraints
%% can be further reduced to $OK_{n}([\vec{A} \slash
%% \vec{\alpha}]P)$. Here, $A'_{1}, \dots, A'_{n}$ are the least
%% solutions for the subtype constraints.

%% \section{Preliminary Experiment}

\section{Preliminary Experiments}
\label{sec:experiment}

%% by comparing applying a model checker to original C language programs
%% with to abstracted behavior.

\subsection{Setup of the experiments}

We conducted preliminary experiments to check feasibility and to
investigate the problems in the current framework.  For the following
C programs, we extracted the behavioral types manually in the form of
another C programs.
\begin{itemize}
\item \texttt{poker.c}: A program that models a poker game.  It
  randomly deals cards to two players, compares the decks of the
  players, and decides the winner.
\item \texttt{database.c}: A program that models a database management
  system.  This program is taken from an online course of the C
  language~\cite{lch}.  The program allocates memory cells when it
  opens a database and deallocates the cells when it closes the
  database.
  %% It opens a database and performs some operations on the database
  %% (retrieve, delete and update data, or create a new database) and
  %% closes the database.  We rewrite this program as
  %% \texttt{database\_rw.c} without changing its meaning.  from website
  %% (http://c.learncodethehardway.org/book/ex17.html) by Zed Shaw,
  %% modeling a database.
\item \texttt{gen\_init\_cpio.c}: A module for a file system in the
  Linux kernel v.3.18.1.  It allocates a memory cell when it creates a
  file.  Deallocation is conducted in error-handling code.
\item \texttt{decompress\_unlzo.c}: A module in the Linux kernel
  v.3.18.1 that decompresses LZO files.  It allocates a memory cell to
  store the contents obtained from an input LZO file.  Deallocation
  occurs in error-handling code.
\end{itemize}

%% \begin{itemize}
%%   \item We rewrite this program as \texttt{poker\_rw.c} without
%%     changing its meaning.
%% \end{itemize}

For verifying that each program consumes only fixed amount of memory
cells, we applied CPAChecker~\cite{beyer2011cpachecker} (1) to the
original file and (2) to the file that represents the extracted
behavior.  All of the experiments are conducted on a machine with an
Intel(R) Core(TM) i7-3770 CPU @ 3.40GHz, 8MB cache and 3.76GB memory,
running on Debian (kernel version 2.6.32-5-amd64) and CPAchecker
(version 1.3.4).

%% We apply a model checker on these programs to check the property --
%% the number of memory usage does not exceed the number of available
%% memory cells.  In order to do it, in a program, we fixed a global
%% number $n$ which denotes the number of available memory for
%% allocation, it will decrease 1 when doing allocation and increase 1
%% when doing deallocation; we insert an assertion statement $assert(n
%% >= 0)$ after allocation statements. Hence, when doing model
%% checking on these programs, the checker will check if the property
%% is guaranteed.

\subsection{Result}

\begin{table}
  \scriptsize
\begin{tabular}{|c|c|c|c|c|c|c|}
\hline
& \multicolumn{6}{|c|}{\texttt{original programs}}  \\
\hline
& \texttt{loc} & \texttt{nfun} & \texttt{cpu time} & \texttt{memory (MB)} & \texttt{fixed num}& \texttt{verified result} \\
\hline
\texttt{poker.c} & 86 & 4 & 2.700 & 2797 & 4  & \texttt{TRUE}  \\
\hline
\texttt{database.c} & 153 & 10 & 12.010 & 2907 & 2  & \texttt{TRUE}  \\
\hline
\texttt{gen\_init\_cpio.c} & 346 & 19 & 9.580 & 2809 & 2  & \texttt{TRUE}  \\
\hline
\texttt{decompress\_unlzo.c} & 162 & 2  & 3.000  & 2806  & 2  & \texttt{TRUE}  \\
\hline
\end{tabular}
\caption{Result of the verification of the original C programs.}
\label{tb:mcc}
\end{table}

\begin{table}
  \scriptsize
\begin{tabular}{|c|c|c|c|c|c|c|}
\hline
&\multicolumn{6}{|c|}{abstracted behavior} \\
\hline
 &\texttt{loc} & \texttt{nfun} & \texttt{cpu time} & \texttt{memory (MB)} & \texttt{fixed num} & \texttt{verified result} \\
\hline
\texttt{poker.c} & 16 & 4 & 1.980 & 2803 & 4  & \texttt{FALSE}  \\
\hline
\texttt{database.c} &  16 & 4 & 2.060 & 2800 & 2 & \texttt{FALSE} \\
\hline
\texttt{gen\_init\_cpio.c} & 16 & 4 & 2.020 & 2802 & 2  & \texttt{FALSE}  \\
\hline
\texttt{decompress\_unlzo.c} & 16 & 4 & 1.970  & 2738  & 2  & \texttt{FALSE}  \\
\hline
\end{tabular}
\caption{Result of the verification of the extracted behavior.}
\label{tb:mca}
\end{table}

\begin{table}
  \scriptsize
\begin{tabular}{|c|c|c|c|c|c|c|}
  \hline
\texttt{poker\_rw.c} & 89 & 4 & 2.740 & 2800 & 4  & \texttt{TRUE}  \\
\hline
\texttt{database\_rw.c} & 151 & 10 & 7.080 & 2907 & 2  & \texttt{TRUE}  \\
\hline
\texttt{gen\_init\_cpio\_rw.c} & 343 &19  & 4.850  & 2744  & 2  & \texttt{TRUE}  \\
\hline
\texttt{decompress\_unlzo\_rw.c} & 92 & 2  & 2.650  & 2800  & 2  & \texttt{TRUE}  \\
\hline
\end{tabular}
\caption{Model checking on the original C programs.  (Rewritten.)}
\label{tb:mcarw}
\end{table}

\begin{table}
  \scriptsize
\begin{tabular}{|c|c|c|c|c|c|c|}
  \hline
  \texttt{poker\_rw.c} & 18 & 4 & 2.020 & 2798 & 4  & \texttt{TRUE}  \\
  \hline
  \texttt{database\_rw.c} &  18 & 4 & 1.990 & 2737 & 2 & \texttt{TRUE} \\
  \hline
  \texttt{gen\_init\_cpio\_rw.c} & 18 & 4 & 2.000  & 2742  & 2  & \texttt{TRUE}  \\
  \hline
  \texttt{decompress\_unlzo\_rw.c} & 18 & 4  & 2.000  & 2796  & 2  & \texttt{TRUE}  \\
  \hline
\end{tabular}
\caption{Model checking on an abstracted behavior}
\label{tb:mca}
\end{table}

Table~\ref{tb:mcc} and Table~\ref{tb:mca} show the result.  We present
the number of lines of each file (\texttt{loc}), the number of
functions (\texttt{nfun}), time spent by CPAChecker in seconds
(\texttt{cpu time}), the amount of memory in megabytes
(\texttt{memory}), the upper-bound of the number of consumed memory
cells (\texttt{fixed num}), and the result of the verification
(\texttt{verified result}).

\subsection{Discussion}

CPAChecker, when it is applied to the original programs, was able to
verify that the programs are totally memory-leak free.  However, the
verification failed for experiments with extracted behaviors.  This is
because our type system is not path-sensitive.  For example, a typical
pattern where verification with the extracted behaviors fail is as
follows.
\begin{verbatim}
while (...) {
  if (/* some condition c */) {
    x = malloc(sizeof(int));
  }
  /* Do something */
  if (/* condition equivalent to c */) {
    free(x);
  }
}
\end{verbatim}
For the program above, extracted behavior is \(\mu\alpha. (\mathbf{0}
+ \Malloc); (\mathbf{0} + \Free); \alpha\), which is not enough to
check no-memory-leak, although it is memory-leak free if the condition
\texttt{c} does not change between the allocation and the
deallocation.  Our type system can deal with the program above by
rewriting it to the following one.
\begin{verbatim}
while (...) {
  if (/* some condition c */) {
    x = malloc(sizeof(int));
    /* Do something */
    free(x);
  } else {
    /* Do something */
  }
}
\end{verbatim}
For the rewritten program, the extracted behavior is
\(\mu\alpha. ((\Malloc; \Free) + \mathbf{0}); \alpha\), for which the
predicate \(\OK_1\) holds.  We confirmed that CPAChecker can verify
\(\OK_1\) for the extracted behavior of the rewritten programs without
penalty on cpu time.

From the comparison of the Table~\ref{tb:mcc} and Table~\ref{tb:mca},
we can observe that the latter is more efficient than the former.

%% The column \texttt{original programs} and \texttt{abstracted behavior}
%% mean applying the model checker to a original C language program and
%% to abstracted behavior respectively.  These two columns consist of
%% several columns: $\sharp$\texttt{loc} means the number of program
%% locations; $\sharp$\texttt{fun} means the number of functions in a
%% program; \texttt{cpu time (total)} means the total execution time of
%% CPU in seconds; \texttt{memory(MB)} means the number of virtual memory
%% cells consumed by model checker in \texttt{MByte}; \texttt{fixed num}
%% means the number of available memory cells for allocation; the
%% \texttt{TRUE} in the \texttt{verified result} column means the number
%% of the memory cells which a program consumes does not exceed the
%% \texttt{fixed num}, otherwise it is \texttt{FALSE} which denotes
%% overflow.

%% The results present in the these two tables show that the
%% resources required for model checking become smaller if applying model
%% checkers to the abstracted behavior, since the abstracted behavior
%% only consists of allocation and deallocation.

%% One thing we should notice in these two tables is that for the same
%% programs, the \texttt{verified result} should be the same when model
%% checking on original programs and abstracted behavior; but it is not,
%% for example, \texttt{database.c} has the \texttt{TRUE} in the
%% Table~\ref{tb:mcc} but the \texttt{FALSE} in the
%% Table~\ref{tb:mca}. The reason is that our approach, abstracted
%% behavior, may not correctly deal with some conditional statements
%% about allocation and deallocation. For example, the behavior of
%% \texttt{database.c} is $\mu\alpha.\Malloc;\Malloc;(((\Free + 0);\Free)
%% + 0);\alpha)$, due to the choice type, it may perform like
%% $\mu\alpha.\Malloc;\Malloc;0;0;\alpha$, which means consuming unbound
%% number of memory cells. The rewritten \texttt{database\_rw.c}, which
%% does not change the semantics of original program, has the behavior
%% $\mu\alpha.(\Malloc;(\Malloc + \Free) + 0);\Free;\alpha)$ which
%% returns \texttt{TRUE} when doing model checking on it.


%% \begin{table}
%% \tiny
%% \begin{tabular}{|c|c|c|c|c|c|c|c|c|c|c|}
%% \hline
%% & \multicolumn{5}{|c|}{original programs} & \multicolumn{5}{|c|}{abstracted behavior} \\
%% \hline
%%  & $\sharp$loc & $\sharp$fun & cpu time (total) & memory (MB) & fixed num & $\sharp$loc & $\sharp$fun & cpu time (total) & memory (MB) & fixed num \\
%% \hline
%% linklist.c & 154 & 13 & 9.770 & 2943 & 6(true) & 20 & 4 & 3.190 & 2918 & 6(false) \\
%% \hline
%% linklst2.c & 140 & 13 & 25.620 & 2955 & 21(true) & 20 & 4 & 10.72 & 2945 & 21(false) \\
%% \hline
%% linkstack.c  & 87 & 10 & 10.830 & 2941 & 11(true) & 20 & 4 & 4.990 & 2916 & 11(false) \\
%% \hline
%% linkqueue.c & 119 & 11 & 13.110 & 2939 & 4(true) & 25 & 4 & 2.660 & 2919 & 4(false) \\
%% \hline
%% binarysorttree.c & 80 & 5 & 30.210 & 2950 & 10(true) & 19 & 4 & 5.130 & 2935 & 10(false) \\
%% \hline
%% database.c & 179 & 12 & 4.930 & 2922 & 3(true) & 21 & 4 & 2.760 & 2920 & 2(false) \\
%% \hline
%% ihex2fw.c & 202 & 7 & 23.490 & 2882 & 5(false) & 15 & 4 & 2.160 & 2797 & 5(false) \\
%% \hline
%% gen\_init\_cpio.c & 346 & 19 & 9.580 & 2809 & 1(true) & 15 & 4 & 2.160 & 2799 & 1(false) \\
%% \hline
%% \end{tabular}
%% %%\caption{My first table}
%% \end{table}

%% \begin{figure}
%%  \centering
%%  \includegraphics[width=14cm]{statistic.png}
%% \caption{Comparison}
%% \label{fig:statistic}
%% \end{figure}


\section{Related work}\label{sec:relatedwork}
Many methods for static memory-leak freedom verification have been
proposed~\cite{DBLP:conf/aplas/SuenagaK09,DBLP:conf/pldi/HeineL03,DBLP:conf/sigsoft/XieA05,DBLP:journals/scp/SwamyHMGJ06,DBLP:conf/sas/OrlovichR06,DBLP:conf/issta/SuiYX12}. These
methods guarantee partial memory-leak freedom and lack of illegal
accesses, whereas our type system guarantees total memory-leak
freedom. By using both their methods and our type system, we can
guarantee safe memory deallocation for nonterminating programs.

Behavioral types are extensively studied in the context of concurrent
program
verification~\cite{DBLP:conf/esop/HondaVK98,DBLP:journals/tcs/IgarashiK04,DBLP:conf/esop/VieiraCS08,DBLP:journals/lmcs/KobayashiSW06}.
These type systems guarantee that the communication pattern of
concurrent programs are as intended.  Our type system is largely
inspired by one by Kobayashi et
al.~\cite{DBLP:journals/lmcs/KobayashiSW06}, which guarantees that a
concurrent program accesses resources according to specification.

Model checking~\cite{clarke1999model,ben2008principles,beyer2011cpachecker} is a widely used technique for automatically
verifying correctness properties of finite-state system. We assume
that some model checkers solves the constraints \(\OK_\nu(P)\) in our
paper. We expect that applying model checker to inferred behavioral
types is better than to the original programs, because a behavioral
type focuses on the actions related to allocations and deallocations,
abstracted away from other features. We plan to do some experiments
for this prospects.



\section{Conclusion}\label{sec:conclusion}

We have described a type system to verify memory-leak freedom for
(possibly) nonterminating programs with manual memory-management
primitives where every memory cell is fixed size.  Our type system
abstracts the memory allocation/deallocation behavior of a program
with a sequential process with actions corresponding to memory
allocation and deallocation.  We have described a type reconstruction
algorithm for the type system.

Our current type system excludes many features of the real-world
programs for simplification.  We are currently investigating the C
programs in the real world to investigate what extension we need to
make to the type system.  One feature we have already noticed is
variable-sized memory blocks.  The current behavioral types ignores
the size of the allocated block, counting only the number of
\(\Malloc\) and \(\Free\).  Hence, a program that contains memory leak
in usual sense may be well-typed in our type system.  We need to
refine the abstraction obtained by types in order to address this
issue.

Currently, our type reconstruction first fixes an upper bound for
\(\nu\) in solving a constraint \(\OK_\nu(P)\).  This makes our
reconstruction incomplete; even if constraint \(\OK_\nu(P)\) holds for
\(\nu \ge n\), our procedure will reject the program if the upper
bound is less than \(n\).  We have not yet figured out whether a
constraint of the form \(\exists \nu. \OK_\nu(P)\), which is needed to
be solvable for reconstruction to be complete, is decidable or not.

%% type-based approach to safe memory deallocation
%% for non-terminating programs. The approach is based on the idea of
%% decomposing safe memory memory deallocation into partial correctness,
%% which is verified by previous type system, and behavioral
%% correctness. We designed a behavioral type system in our paper for
%% verification of behavioral correctness. Currently, we are looking for
%% a model checker to estimate an upper bound of consumption given a
%% behavioral type and planning to implement a verifier and conduct
%% experiment to see whether our approach is feasible.


\paragraph{Acknowledgment}
We thank the comments by the reviewers of PPL 2015.  This research is
partially supported by KAKENHI 25730040 and 25280024.

\bibliographystyle{abbrv}
\bibliography{tan}

\iffull
\newpage
\appendix
\section*{Appendix}

\section{Proof of Lemmas}
\label{sec:proof}

\begin{lemma}
\label{lem:okPreserved}
If \(\OK_n(P)\) and \(P \xlongrightarrow{\rho} P'\), then
\begin{itemize}
\item \(\OK_{n-1}(P')\) if \(\rho = \Malloc\),
\item \(\OK_{n+1}(P')\) if \(\rho = \Free\), and
\item \(\OK_n(P')\) if \(\rho = \tau\).
\end{itemize}
\end{lemma}
\begin{proof}

Case analysis on \(P \xlongrightarrow{\rho} P'\).

\todo{To be revised.}

\noindent Case $P = \TSKIP;P'$\\

According to rule E-Skip, we should prove
$\sharp_{m}(P')-\sharp_{f}(P') \le n'$ where $n'$ is $n$.

Because we have
\begin{eqnarray*}
  OK_{n}(P)  & & =  OK_{n}(\SKIP;P')\\
  & & \Rightarrow \sharp_{m}(\SKIP;P') - \sharp_{f}(\SKIP;P') \le n \\
  & & \Rightarrow \sharp_{m}(P') - \sharp_{f}(P') \le n \
\end{eqnarray*}
Then it is proved. \\

\noindent Case $P = \Malloc;P'$ \\

Here according to rule E-Malloc, we know the $n'$ is $n-1$.

Therefore we should prove $\sharp_{m}(P') - \sharp_{f}(P') \le n-1$
\begin{eqnarray*}
  OK_{n}(P)&& =  OK_{n}(\Malloc;P')\\
  &&\Rightarrow \sharp_{m}(\Malloc;P') - \sharp_{f}(\Malloc;P') \le n \\
  &&\Rightarrow  \sharp_{m}(P') + 1 - \sharp_{f}(P') \le n\\
  &&\Rightarrow  \sharp_{m}(P')  - \sharp_{f}(P') \le n-1\\
\end{eqnarray*}

Then it is proved.\\

\noindent Case $P = \Free;P'$ \\

According to rule E-Free, we should prove $\sharp_{m}(P') - \sharp_{f}(P') \le n+1$.

\begin{eqnarray*}
  OK_{n}(P)  & & =  OK_{n}(\Free;P')\\
  & &\Rightarrow  \sharp_{m}(\Free;P') - \sharp_{f}(\Free;P') \le n \\
  & & \Rightarrow \sharp_{m}(P')  - \sharp_{f}(P') - 1  \le n\\
  & & \Rightarrow \sharp_{m}(P')  - \sharp_{f}(P') \le n+1\\
\end{eqnarray*}

Then it is proved. \\

\noindent Case $P = P_{1};P_{2}$\\

To prove it by contradiction.

Suppose that $OK_{n'}(P_{1}';P_{2})$ does not hold. Then we have 
$P_{1};P_{2} \xlongrightarrow{\alpha} P_{1}';P_{2} \xlongrightarrow{\exists \sigma} Q$, $s.t.$ $\sharp_{m}(\sigma) - \sharp_{f}(\sigma) > n'$\\

From the premise $OK_{n}(P) = OK_{n}(P_{1};P_{2})$, we get 
\setcounter{equation}{0}
\begin{align}
  &  \sharp_{m}(\alpha \cdot \sigma) - \sharp_{f}(\alpha \cdot \sigma) \le n \label{eqok1.1}
\end{align}

From \eqref{eqok1.1}, we get
\begin{align}
\sharp_{m}(\alpha) + \sharp_{m}(\sigma) - \sharp_{f}(\alpha) - \sharp_{f}(\sigma) \  \label{eqok1.2}
\end{align}
and with
$$
   n'=\left\{
   \begin{aligned}
     n + 1, && \alpha = \Free \\
     n - 1,  && \alpha = \Malloc  \\
     n ,      && otherwise
   \end{aligned}
   \right.
$$
Therefore, we get \\
$n' + \sharp_{m}(\alpha) - \sharp_{f}(\alpha) < \sharp_{m}(\alpha) + \sharp_{m}(\sigma) - \sharp_{f}(\alpha) - \sharp_{f}(\sigma) \le n $ \\
When $\alpha = \Free$, we get that $n + 1 - 1 < n$\\
When $\alpha = \Malloc$, we get that $ n - 1 + 1 < n $ \\
When $ \alpha = other$,  we  get that $ n < n $ \\
 All of the three cases are equal to $n$. Therefore we get the contradiction.
\end{proof}

\begin{pfof}{Lemma~\ref{}}
By induction on the derivation of evaluation rules.\\

\begin{itemize}
\item Case: $\langle H, R, \FREE, n \rangle \xlongrightarrow{\Free} \langle H', R', \SKIP, n + 1 \rangle $.

We have \(\OK_n(P)\) and \(\Theta; \Gamma \vdash \Free(x) \COL P\).
From inversion of the typing rules, we have \(\Theta; \Gamma \vdash
\Free(x) \COL \Free\) and \(\Free \le P\) for some \(P'\).  Hence,
from the definition of subtyping, we have \(\TSKIP \le P''\) and \(P
\xLongrightarrow{\Free} P''\) for some \(P''\).

We need to find \(P_1\) such that \(P \xLongrightarrow{\Free} P_1\),
\(\Theta; \Gamma \vdash \SKIP \COL P_1\), and \(\OK_{n+1}(P_1)\).
Take \(P''\) as \(P_1\).  Then, \(P \xLongrightarrow{\Free} P''\) as
we stated above.  We also have \(\Theta; \Gamma \vdash \SKIP \COL
P''\) from \rn{T-Skip}, \(\TSKIP \le P''\), and \rn{T-Sub}.
\(\OK_{n+1}(P'')\) follows from Lemma~\ref{lem:okPreserved}.

%% From the assumption, we have known that: \textcircled{1} $OK_{n}(P)$, and \textcircled{2} $\Theta; \Gamma \vdash \Free\Cirx:P$.

%% By the inversion lemma on \textcircled{2}, we have: \textcircled{3} $\Free \le P$.

%% From the definition of subtyping, \textcircled{3} and rule $\Free \xlongrightarrow{\Free} 0$, we get:
%% \begin{center}
%% $\exists P''$ s.t. \textcircled{4} $P \overset{\text{$\Free$}}{\Longrightarrow} P''$,  and \textcircled{5} $0 \le P''$
%% \end{center}

%% We need to prove that there exists $P'$ and $\Gamma'$ such that:
%% \begin{center}
%% \textcircled{6} $\Theta; \Gamma' \vdash \SKIP: P'$,  and \textcircled{7} $P \overset{\text{$\Free$}}{\Longrightarrow} P'$
%% \end{center}

%% Take $P''$ as $P'$. Then \textcircled{7} holds. By the typing rule T-Skip and \textcircled{5}, we get:
%% $$
%%    \frac{\Theta; \Gamma' \vdash \SKIP : 0 \ \ \ \  0 \le P''}
%%    {\Theta; \Gamma' \vdash \SKIP : P''}
%%    \Rtab \mbox{(T-Sub)}
%% $$

%% Therefore, \textcircled{6} holds. \\

\item Case: $\langle H, R, \LET x = \MALLOC \IN s_{1}, n \rangle
  \xlongrightarrow{\Malloc} \langle H', R', [x'/x]s_{1}, n - 1 \rangle
  $.

  From the assumption, we have \(\Theta; \Gamma \vdash \LET x =
  \MALLOC \IN s_{1} \COL P\) and \(\OK_{n}(P)\). By the inversion of
  typing rules, we have \(\Malloc;P_1 \le P\) and \(\Theta; \Gamma
  \vdash s_{1} : P_{1}\) for some \(P_1\). We have the following
  derivation: \infrule{ \Malloc \xlongrightarrow{\Malloc} 0}
  {\Malloc;P_1 \xlongrightarrow{\Malloc} 0;P_{1}} and \(0;P_1
  \rightarrow P_{1}\), then we have \(\Malloc;P_{1}
  \xLongrightarrow{\Malloc} P_{1}\). Hence, By the definition of
  subtyping, we have \(P \xLongrightarrow{\Malloc}P''\) and \(P_{1}
  \le P''\) for some \(P''\)

  We need to find \(P'\) such that \(\Theta; \Gamma' \vdash s_{1} :
  P'\), and \(P \overset{\text{$\Malloc$}}{\Longrightarrow} P'\). Take
  \(P''\) as \(P'\). Then \(P \xLongrightarrow{\Malloc}\) as we state
  above. We also have \(\Theta;\Gamma \vdash s_1 \COL P''\) from
  \(T-Sub\), \(\Theta;\Gamma \vdash s_1 \COL P_1\) and \(P_1 \le
  P''\). \(\OK_{n-1}(P'')\) follows from Lemma~\ref{lem:okPreserved}.

\item Case: $\langle H, R, \SKIP;s_{1}, n \rangle \rightarrow \langle
  H', R', s_{1}, n \rangle $.
we have
\(\Theta; \Gamma \vdash \SKIP;s_{1} : P$, and \textcircled{2} $OK_{n}(P)$


By the inversion lemma on \textcircled{1}, we have
\begin{center}
\textcircled{3} $\Theta; \Gamma \vdash s_{1} : P_{1}$, and \textcircled{4} $0;P_{1} \le P $
\end{center}

We need to prove that there exists $P'$ and $\Gamma'$ such that
\begin{center}
\textcircled{5} $\Theta; \Gamma' \vdash s_{1} : P'$, and \textcircled{6} $P \rightarrow^{*} P'$
\end{center}

By the definition of subtyping and $0;P_{1} \rightarrow P_{1}$, then we get that $\exists P''$
\begin{center}
 \textcircled{7} $P \rightarrow^{*} P''$, and \textcircled{8} $P_{1} \le P''$
\end{center}

Taking $P''$ as  $P'$, we get $P \rightarrow^{*} P'$

And by using rule T-Sub with premises $\Gamma \vdash s_{1} : P_{1}$ and $P_{1} \le P''$, then we have 
$$
    \frac{\Theta; \Gamma \vdash s_{1} : P_{1} \  \  P_{1} \le P''}
    {\Gamma \vdash s_{1} : P''}
    \Rtab \mbox{(T-Sub)}
$$

Therefore, we prove that $\Gamma \vdash s_{1} : P'$ \\

\item Case: $\langle H, R, *x \leftarrow y , n \rangle \rightarrow  \langle H', R', \SKIP, n  \rangle $. \\

From the assumption, we already have
\begin{center}
\textcircled{1} $\Theta; \Gamma \vdash *x \leftarrow y : P$, and \textcircled{2} $OK_{n}(P)$
\end{center}

From the inversion lemma on \textcircled{1}, we have \textcircled{3} $0 \le P$.

We need to find $P'$ and $\Gamma'$ such that
\begin{center}
 \textcircled{4} $\Theta; \Gamma' \vdash \SKIP: P'$, and \textcircled{5} $P \rightarrow^{*} P'$
\end{center}

Taking $P$ as $P'$, then \textcircled{5} holds.

And because of the following derivation:
$$
  \frac{\Theta; \Gamma' \vdash \SKIP: 0 \ \ \ 0 \le P}
   {\Theta;\Gamma' \vdash \SKIP : P}
  \Rtab \mbox{(T-Sub)}
$$
therefore \textcircled{4} holds. \\

\item Case: $\langle H, R, \LET x = y\  \IN s_{1} , n \rangle \rightarrow  \langle H', R', [x'/x]s_{1}, n  \rangle $. \\

From assumption, we have 
\begin{center}
\textcircled{1} $\Theta; \Gamma \vdash \LET x = y\  \IN \  s_{1} : P$, and \textcircled{2} $OK_{n}(P)$.
\end{center}

From the inversion lemma and \textcircled{1}, we have 
\begin{center}
\textcircled{3} $\Theta; \Gamma \vdash s_{1} : P_{1}$, and $P_{1} \le P$.
\end{center}

We need to find $P'$ and $\Gamma'$ such that:
\begin{align}
  &\Theta; \Gamma' \vdash s_{1} : P' \ \ and& \label{eq5.4.1}\\
  &P \xlongrightarrow{\tau}^{*} P'& \label{eq5.4.2}
\end{align}

Taking P as P'. Therefore \eqref{eq5.4.2} holds, because of the definition of $\rightarrow^{*}$.

And because of the following derivation, \eqref{5.4.1} holds.
$$
  \frac{\Theta; \Gamma' \vdash s_{1} : P_{1} \ \ \ P_{1} \le P}
  {\Theta; \Gamma' \vdash s_{1} : P}
  \Rtab \mbox{(T-Sub)}
$$

\item Case: $\langle H, R, \LET x = \NULL \  \IN \  s_{1}, n \rangle \rightarrow \langle H', R', [x'/x]s_{1}, n \rangle $\\

From the assumption, we know that
\begin{center}
\textcircled{1}$\Theta; \Gamma \vdash \LET x = \NULL \  \IN \ s_{1} : P$, and \textcircled{2} $OK_{n}(P)$.
\end{center}

By inversion lemma on \eqref{eqa.1}, we get:
\begin{center}
$\Theta; \Gamma \vdash s_{1} : P_{1}$, and $ P_{1} \le P$.
\end{center}

We need to prove that there exists $P'$ and $\Gamma'$ such that
\begin{center}
$\Theta; \Gamma' \vdash s_{1} : P'$, and $P \rightarrow^{*} P'$.
\end{center}

Taking P as P'. Because of the following derivation, the \eqref{eqa.5} holds.
$$
  \frac{\Theta; \Gamma' \vdash s_{1} : P_{1} \ \ \ \ P_{1} \le P}
  {\Theta; \Gamma' \vdash s_{1} : P}
  \Rtab \mbox{(T-Sub)}
$$

And because of the definition of $\xlongrightarrow{\tau}^{*}$, the \eqref{eqa.6} holds. \\

\item Case: $\langle H, R, \LET x = *y \  \IN \  s_{1}, n \rangle \rightarrow \langle H', R', [x'/x]s_{1}, n \rangle $\\

From the assumption, we know that
\begin{center}
$\Theta; \Gamma \vdash \LET x = *y \  \IN \  s_{1} : P$, and $OK_{n}(P)$.
\end{center}

By the inversion lemma on \eqref{eqb.1}, we get:
\begin{center}
$\Theta; \Gamma \vdash s_{1} : P_{1}$, and $P_{1} \le P$.
\end{center}

We need to prove there exists $P'$ and $\Gamma'$ such that:
\begin{center}
$\Theta; \Gamma' \vdash s_{1} : P'$, and $P \rightarrow^{*} P'$.
\end{center}

Taking P as P'. Because of  following derivation, the \eqref{eqb.5} holds.
$$
   \frac{\Theta; \Gamma' \vdash s_{1} : P_{1} \ \ \ P_{1} \le P}
   {\Theta; \Gamma' \vdash s_{1} : P}
   \Rtab \mbox{(T-Sub)}
$$

And because of the definition of $\xlongrightarrow{\tau}^{*}$, \eqref{eqb.6} holds. \\

\item Case $\langle H, R, \IFNULL \Cirx \  \THEN s_{1} \  \ELSE \  s_{2}, n \rangle \rightarrow \langle H', R',s_{1}, n \rangle $\\

From the assumption, we have that:
\begin{center}
$\Theta; \Gamma \vdash \IFNULL \Cirx \  \THEN \  s_{1} \ \ELSE \ s_{2} : P$, and $OK_{n}(P)$.
\end{center}

By the inversion leamma on \eqref{eqc.1}, we get:
\begin{center}
$\Theta; \Gamma \vdash s_{1} : P_{1}$, and $ p_{1} \le P'$.
\end{center}

We need to prove that there exists $P'$ and $\Gamma'$ such that:
\begin{center}
 $\Theta; \Gamma' \vdash s_{1} : P_{1}$, and $P \rightarrow^{*} P'$.
\end{center}

Taking P as P'. Because of the following derivation, \eqref{eqc.5} holds.
$$
  \frac{\Theta; \Gamma' \vdash s_{1} : P_{1} \ \ \ \ P_{1} \le P}
  {\Theta; \Gamma' \vdash s_{1} : P}
  \Rtab \mbox{(T-Sub)}
$$

And by the definition of $\xlongrightarrow{\tau}^{*}$, \eqref{eqc.6} holds. \\

\item Case: $\langle H, R, f(x) , n \rangle \rightarrow  \langle H', R', [x'/x]s_{1}, n  \rangle $ where the body of function $f(x)$ is $s_{1}$. we can see $f(x)$ and $s_{1}$ as $s$ and $s'$ respectively. \\

From the assumption, we already have
\begin{center}
$\Gamma \vdash f(x) : P$, and $OK_{n}(P)$.
\end{center}

By the inversion lemma and \eqref{eq6.1}, we have
\begin{center}
$P_{1} \le P$, and $\Gamma \vdash s_{1} : P_{1}$.
\end{center}

From the definition of subtyping and $P_{1} \xlongrightarrow{0} P_{1}$, we get $\exists P''$ s.t.
\begin{center}
$P \xlongrightarrow{0} P''$, and $P_{1} \le P''$.
\end{center}

Taking the $P''$ to be $P'$, then we get $P \xlongrightarrow{0} P'$.\\
And by using the subtyping rule with premises \eqref{eq6.4} and  \eqref{eq6.6}, we have
$$
\frac{\Gamma \vdash s_{1} : P_{1} \ \ \ \ P_{1} \le P'}{\Gamma \vdash s_{1} : P'}
$$
\end{itemize}

Therefore we prove that $\Gamma \vdash s' : P'$ where $s'$ is the command $s_{1}$.

\end{pfof}

\section{Syntax Directed Typing Rules}

%% Typing rules in Figure are not immediately suitable for type
%% inference. The reason is that the subtyping rule can be applied to any
%% kind of term. This means that, any kind of term $s$ can be applied by
%% either subtyping rule or the other rule whose conclusion matches the
%% shape of the $s$ \cite{plain:book1}.

%% In order to yield a type inference algorithm, we should do something with the subtyping rule. The method is to merge the subtyping rule with the other rules by introducing a set $C$ of constraints, where $C$ consists of subtype constraints on behavioral types of the form $P_{1}\le P_{2}$ and $OK_{n}(P)$.

%% Syntax directed typing rules are listed in Figure 

Figure~\ref{fig:sdtyping} shows the syntax-directed version of the
typing rules in Figure~\ref{fig:typingrules}.  Based on these rules,
we design a constraint generation algorithm for the type
reconstruction, which is presented in Figure~\ref{fig:tyin}.

\begin{figure}[tp]
\begin{minipage}{\textwidth}

$$
     \frac{ C = \emptyset}
           {\Theta; \Gamma; C \vdash \SKIP : \mathbf{0}}
      \Rtab \mbox{(ST-Skip)}
$$
$$
      \frac{\Theta;\Gamma ; C_{1} \vdash s_{1} : P_{1} \Rtab \Theta; \Gamma ; C_{2} \vdash s_{2} : P_{2} \Rtab C = C_{1}\cup C_{2} \cup \{ P_{1};P_{2} \le P\}}
      {\Theta;\Gamma; C \vdash s_{1};s_{2} : P}
      \Rtab \mbox{(ST-Seq)}
$$
$$
      \frac{\Theta;\Gamma;C_{1} \vdash y \Rtab \Theta;\Gamma; C_{2} \vdash x : \mathbf{Ref} \Rtab C = C_{1}\cup C_{2}}
      {\Theta;\Gamma; C \vdash *x \leftarrow y : \mathbf{0}}
      \Rtab \mbox{(ST-Assign)}
$$
$$
      \frac{C = \emptyset}
      {\Gamma ; C \vdash \Free() : \Free}
     \Rtab \mbox{(ST-Free)}
$$
$$
     \frac{\Theta;\Gamma, x ; C_{1} \vdash s : P_{1} \Rtab C = C_{1} \cup\{P_{1}\le P\}}
     {\Theta;\Gamma; C \vdash \LET x = \Malloc() \; \IN s : \Malloc ; P}
     \Rtab \mbox{(ST-Malloc)}
$$
$$
     \frac{\Theta;\Gamma; C_{1} \vdash y \Rtab \Theta;\Gamma, x ; C_{2} \vdash s : P_{1} \Rtab C = C_{1}\cup C_{2} \cup \{P_{1} \le P \}}
     {\Theta;\Gamma ; C \vdash \LET x = y \;  \IN s : P}
     \Rtab \mbox{(ST-LetEq)}
$$
$$
     \frac{\Theta;\Gamma ; C_{1} \vdash y: \mathbf{Ref} \Rtab \Theta;\Gamma, x ; C_{2} \vdash s : P_{1} \Rtab C = C_{1}\cup C_{2}\cup\{P_{1} \le P\}}
     {\Theta;\Gamma ; C \vdash \LET x = *y \; \IN s : P}
     \Rtab \mbox{(ST-LetDref)}
$$
$$
     \frac{\Theta;\Gamma; C_{1} \vdash x \Rtab \Theta;\Gamma; C_{2} \vdash s_{1} : P_{1} \Rtab \Theta;\Gamma; C_{3} \vdash s_{2} : P_{2}  \Rtab  C = C_{1} \cup C_{2} \cup C_{3} \{P_{1}\le P, P_{2}\le P \}}
     {\Theta;\Gamma; C \vdash \IFNULL\Cirx \THEN s_{1} \ELSE s_{2} : P }
    \; \;  \mbox{(ST-IfNull)}
$$
$$
     \frac{\Theta(f) = P_{1} \Rtab C = P_{1} \le P}
     {\Gamma,\vec{x}:\vec{\tau} \vdash f(\vec{x}) : P }
     \Rtab \mbox{(ST-Call)}
$$
$$
     \frac{\Theta \vdash D : \Theta \Rtab \Theta ; \emptyset ; C_{1} \vdash s : P \Rtab C = C_{1}\cup\{OK_{n}(P)\}}
     {C \vdash (D , s) }
     \Rtab \mbox{(ST-Program)}
$$

\end{minipage}
\caption{Syntax Directed Typing Rules.}
\label{fig:sdtyping}
\end{figure}

\section{Typing Inference Algorithm}\label{sec:typeinference}

\begin{figure}
\begin{nospaceflalign*}
   PT_{\Theta}(f) &  =  &\\
  & \ \  \LET  \alpha = \Theta(f) & \\
  & \ \ \IN   (C = \{\alpha \le \beta \}, \beta) &
\end{nospaceflalign*}
\begin{nospaceflalign*}
   PT_{\Theta}(\SKIP) &  =  (\emptyset, 0)&
\end{nospaceflalign*}
\begin{nospaceflalign*}
   PT_{\Theta}(s_{1}&;s_{2})  =  &\\
   & \ \ \ \LET (C_{1}, P_{1}) = PT_{\Theta}(s_{1}) & \\
   &\ \ \ \ \ \ \  \ (C_{2}, P_{2}) = PT_{\Theta}(s_{2}) & \\
   & \ \ \ \IN   (C_{1} \cup C_{2}\cup \{P_{1}; P_{2} \le \beta \}, \beta) &
\end{nospaceflalign*}
\begin{nospaceflalign*}
   PT_{\Theta}(*x& \leftarrow y)   =  &\\
  & \ \ \LET (C_{1}, \emptyset) = PT_{v}(*x) & \\
  & \ \ \ \ \ \ (C_{2}, \emptyset) = PT_{v}(y) & \\
  &\ \  \IN    (C_{1} \cup C_{2},  0) &
\end{nospaceflalign*}
\begin{nospaceflalign*}
   PT_{\Theta}(\Free(x)) &  = (\emptyset, \Free)  &
\end{nospaceflalign*}
\begin{nospaceflalign*}
   PT_{\Theta}(\LET &x = \Malloc() \  \IN s)  =  &\\
   &\LET (C_{1}, P_{1}) = PT_{v}(s) & \\
   &\IN  (C_{1} \cup \{P_{1} \le \beta \} ,  \Malloc; \beta) &
\end{nospaceflalign*}
\begin{nospaceflalign*}
   PT_{\Theta}(\LET &x = y \  \IN s )  =  &\\
   &  \LET (C_{1}, \emptyset) = PT_{v}(y) & \\
   & \ \ \ \ \ (C_{2}, P_{1}) = PT_{\Theta}(s) & \\
   &  \IN   (C_{1} \cup C_{2}\cup \{P_{1} \le \beta \},  \beta) &
\end{nospaceflalign*}
\begin{nospaceflalign*}
   PT_{\Theta}(\LET &x = *y \  \IN s )  =  &\\
   & \LET  (C_{1}, \emptyset) = PT_{v}(y) & \\
   &\ \ \ \ \ \ (C_{2}, P_{1}) = PT_{\Theta}(s) & \\
   & \IN   (C_{1} \cup C_{2}\cup \{P_{1} \le \beta \},  \beta) &
\end{nospaceflalign*}
\begin{nospaceflalign*}
   PT_{\Theta}(&\IFNULL(x) \  \THEN  s_{1} \  \ELSE \ s_{2} )  =  &\\
   & \ \ \ \ \ \LET  (C_{1}, P_{1}) = PT_{\Theta}(s_{1}) & \\
   &\ \ \ \ \  \ \ \ \ \ (C_{2}, P_{2}) = PT_{\Theta}(s_{2}) & \\
   &\ \ \ \ \  \ \ \ \ \ (C_{3}, \emptyset) = PT_{v}(x) & \\
   & \ \ \ \ \ \ \IN   (C_{1} \cup C_{2}\cup C_{3}\cup \{P_{1} \le \beta, P_{2} \le \beta \},  \beta) &
\end{nospaceflalign*}
\begin{nospaceflalign*}
   PT(\langle D, s \rangle&)   =  &\\
   &\LET  \Theta = \{ f_{1}:\alpha_{1}, \dots, f_{n}:\alpha_{n}  \} &\\
   & \ \ \ \ \  where \ \{ f_{1},\dots, f_{n} \} = dom(D) \ and \ \alpha_{1}, \dots, \alpha_{n} \  are \ fresh  & \\
   & \IN    \LET  (C_{i}, P_{i}) = PT_{\Theta}(D(f_{i})) \  for \  each \ i & \\
   & \IN    \LET  C_{i}^{'} = \{ \alpha_{i} \le P_{i} \} \ for \  each \ i & \\
   & \IN    \LET  (C, P) = PT_{\Theta}(s)  & \\
   & \IN   (C_{i} \cup C_{i}^{'} ) \cup C \cup  \{OK(P)\},  P) &
\end{nospaceflalign*}
\caption{Type Inference Algorithm}
\label{fig:tyin}
\end{figure}

\else
\fi

\end{document}
