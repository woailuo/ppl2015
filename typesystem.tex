
\section{Type system}
\label{sec:typesystem}

\subsection{Types}

Types are described as follows.

\[
\begin{array}{rlcl}
  P & (\mbox{behavioral types})&::=& {\bf 0} \tB P_{1};P_{2} \tB P_{1}+P_{2} \tB \Malloc \tB \Free \tB \alpha \tB \mu\alpha.P \\
  \FUNTYPE & (\mbox{function\ types}) &::=& P    \\
  \Gamma & (\mbox{variable type environments}) &::=& x_1, x_2, \dots, x_n\\
  \Theta & (\mbox{function type environments}) &::=& f_1\COL\FUNTYPE_1,\dots,f_n\COL\FUNTYPE_n.
\end{array}
\]

We use a metavariable \(P\) for \emph{behavioral types} which are
abstractions of the behavior of programs.  The type ${\bf 0}$
represents do-nothing behavior.  The type \(P_1;P_2\) represents the
sequential execution of \(P_1\) and \(P_2\).  The type \(P_1 + P_2\)
represents the choice between \(P_1\) and \(P_2\).  The type
\(\Malloc\) represents an allocation a memory cell exactly once.  The
type \(\Free\) represents a deallocation.  \(\alpha\) is a type
variable. The type \(\mu \alpha.P\) represents the behavior of
\(\alpha\) defined by the recursive equation \(\alpha = P\).

%% For example, given \(\Malloc;\alpha;\Free\) as \(P\), a recursive type
%% is of the form \(\mu \alpha. \Malloc;\alpha;\Free\) and can go further
%% as \(\Malloc;(\mu\alpha.\Malloc;\alpha;\Free);\Free\).

%%  $P_{1};P_{2}$ is for sequential execution. $P_{1} + P_{2}$
%% is abstracted as conditional. $\Malloc$ is the behavior of a statement
%% that allocates a memory cell exactly once. $\Free$ is for deallocating
%% memory cell exactly once. $\mu \alpha. P$ is a recursive type. For
%% example, the behavior of the body of function $h$ in Figure~\ref{ex:np} is
%% abstracted as $\mu
%% \alpha. \Malloc;\Malloc;\Free;\Free;\alpha$. $\alpha$ is a type
%% variable and bounded to the recursive constructor $\mu \alpha$.

%% The function type is described as $(\tau_{1}, \dots, \tau_{n})P$,
%% which means a function receives some pointers as arguments and its
%% body is abstracted as a behavioral type $P$.

% Semantics of Behavioral Types %
% \subsection{Semantics of behavioral types}

The semantics of behavioral type are given by the following labeled
transition system.

\infax
{\mathbf{0};P \xlongrightarrow{\tau} P}
\infax
{\Malloc \xlongrightarrow{\Malloc} 0}
\infax
{\Free \xlongrightarrow{\Free} 0}
\infax
{\mu \alpha.P \xlongrightarrow{\tau}  [\mu \alpha . P/\alpha]  P}
\infax
{P_{1} + P_{2} \xlongrightarrow{\tau} P_{1}}
\infax
{P_{1} + P_{2} \xlongrightarrow{\tau} P_{2}}
\infrule
{P_{1} \xlongrightarrow{\rho} P_{1}' }
{P_{1};P_{2} \xlongrightarrow{\rho} P_{1}';P_{2}}

%% The notation $\rightarrow$ denotes that a behavioral type can be
%% reduced by the internal action. Notation $\xlongrightarrow{\alpha}$
%% means that a behavioral type can be reduced by executing $\alpha$
%% actions, and the $\alpha$ here is $\{\Malloc, \Free\}$.

\(P \xlongrightarrow{\rho} P'\) means that \(P\) can make an action
\(\rho\) and, after \(P\) make the action \(\rho\), \(P\) turns into
\(P'\). \todo{Add explanation to several rules.}

% Type Judgments

\subsection{Typing rules}

The type judgment of our type system is of the form $\Theta ; \Gamma
\vdash s : P$. It intuitively expresses that the behavior of $s$ is
represented $P$ under variable and function environments $\Theta$ and
$\Gamma$.  

%%  We design the type system so that
%% this type judgment implies the property: when $s$ executes
%% $\Malloc$(resp.$\Free$), then $P$ is equivalent to
%% $\Malloc;P'$(resp.$\Free;P'$) for a type $P'$ such that $\Theta;
%% \Gamma \vdash s': P'$, where $s'$ is the continuation of $s$. This
%% property guarantees the behavioral type soundly abstracts the upper
%% bound of the consumed memory cells.

%% Typing rules are presented in Figure~\ref{fig:typingrules}. In the rule for assignment, the behavior of  $*x \leftarrow y$ is $\bf 0$. The rule for $\Free$ represents that the behavior of $\Free \Cirx$ is $\Free$. The rule T-Malloc represents that $\LET x = \MALLOC \; \IN s$ has the behavior $\Malloc;P$, where $P$ is the behavior of statement $s$. The rule for function call represents that function $f$ has the behavior $P$ which is the behavior of the body of this function .

%% \(\vdash D \COL \Theta\) denotes that the set \(D\) of definitions
%% has a type \(\Theta\).

Before presenting typing rules, we need several auxiliary definitions.
First, we define predicate \(\OK_n(P)\) that means that \(P\)
describes the behavior of a program that uses at most \(n\) memory
cells.

\begin{myDef}[\(\sharp_{\rho}(\sigma)\)]
 \label{df:sharf}
\(\sharp_{\rho}(\sigma)\) is the number of \(\rho\) in the sequence
\(\sigma\).
\end{myDef}

%% begin{myDef}[OK(P)]
%%  \(OK_n(P)\) holds if a program executes sequence actions \( \sigma\) for any \(P'\) such that \(P\xLongrightarrow{\sigma}P'\), then the difference of \( \sharp_{malloc}(\sigma)\) and \( \sharp_{free}(\sigma)\) does not exceed the number \(n\).
%% \label{df:okn}
%% \
%%end{myDef}[OK_n(P)]

\begin{myDef}
\label{df:okn}
\(OK_{n}(P)\) holds if, for any \(P'\), if \(P
\xlongrightarrow{\sigma}P'\) then
\(\sharp_{malloc}(\sigma)-\sharp_{free}(\sigma)\le n\).
\end{myDef}

We also define subtyping.  Following the idea of Kobayashi et
al.~\cite{}, we define subtyping by simulation between behavioral
types.

 \begin{myDef}[Subtype]
% $P_{1} \le P_{2}$ represents that $P_{1}$ is the subtype of $P_{2}$ and  means that:

\(P_1 \le P_2\) is the largest relation such that, for any \(P_1'\)
and \(\rho\), if \(P_1 \xlongrightarrow{\rho} P_1'\), then there
exists \(P_2'\) such that \(P_2 \xLongrightarrow{\rho} P_2'\) and
\(P_2 \le P_2'\).

\label{df:subtype}
\end{myDef}
%% At the end of $s$, memory leak freedom is guaranteed by $OK_{n}(P)$ ,where $P$ is the behavior of $s$. $OK_{n}(P)$ is defined as Definition~\ref{df:okn} in which $\sharp_{malloc}(\alpha)$ and $\sharp_{free}(\alpha)$ are functions to count the number of $\Malloc$ and $\Free$ actions in $\alpha$ respectively. This definition, intuitively, means at every running step the number of allocated memory cells will never go out of memory scope.
%% \begin{myDef}
%%   $OK_{n}(P) \iff \forall P',\; P \xlongrightarrow{\alpha}^{*}P'$ then $\sharp_{malloc}(\alpha)-\sharp_{free}(\alpha)\le n$.
%% \label{df:okn}
%% \end{myDef}

\begin{figure}[tp]
\begin{minipage}{\textwidth}

% Skip type
\infax[T-Skip]
{\Theta ; \Gamma \vdash \SKIP : \mathbf{0}}
% Sequence type
\infrule[T-Seq]
{\Theta ; \Gamma \vdash s_{1} : P_{1} \Rtab \Theta ; \Gamma \vdash s_{2} : P_{2}}
{\Theta ; \Gamma \vdash s_{1} ; s_{2} : P_{1};P_{2} }
% Assignment type
\infrule[T-Assign]
{\Theta ; \Gamma \vdash y  \Rtab \Theta ; \Gamma \vdash x }
{\Theta ; \Gamma \vdash *x \leftarrow y : \mathbf{0} }
% Free(deallocate) type
\infrule[T-Free]
{\Theta ; \Gamma \vdash x  }
{\Theta ; \Gamma \vdash \Free(x) : \Free}
% Malloc type
\infrule[T-Malloc]
{\Theta ; \Gamma,x \vdash s : P}
{\Theta ; \Gamma \vdash \LET x = \MALLOC \; \IN s  : \Malloc;P}
% Let eq type
\infrule[T-LetEq]
{\Theta ; \Gamma \vdash y   \Rtab \Theta ; \Gamma , x  \vdash s : P}
{\Theta ; \Gamma \vdash \LET x = y \; \IN s : P}
% Dereference type
\infrule[T-LetDref]
{\Theta ; \Gamma \vdash y  \Rtab \Theta ; \Gamma , x  \vdash s : P}
{\Theta ; \Gamma \vdash \LET x = *y \; \IN s : P}
% Let NULL type
\infrule[T-LetNull]
{\Theta ; \Gamma, x  \vdash s : P}
{\Theta ; \Gamma \vdash \LET x = \mathbf{null} \; \IN s : P}
% Subtyping
\infrule[T-Sub]
{\Theta ; \Gamma \vdash s : P_{1} \Rtab P_{1} \le P_{2}}
{\Theta ; \Gamma \vdash s : P_{2}}
 % ifnull s then s type
\infrule[T-IfNull]
{\Theta ; \Gamma \vdash x    \ \ \ \  \Theta ; \Gamma \vdash s_{1} : P \ \ \ \ \Theta ; \Gamma \vdash s_{2} : P}
{\Theta ; \Gamma \vdash \IFNULL(x) \; \THEN s_{1}\; \ELSE s_{2} : P}
% Function call type
\infrule[T-Call]
{ \Theta(f) = P}
{\Theta; \Gamma, \vec{x} \vdash f(\vec{x}) : P}
\infrule[T-Def]
{\DOM(D) = \DOM(\Theta)
\andalso
\Theta; x_1,\dots,x_n \vdash s \COL \Theta(f)
\mbox{ for each \(f \mapsto (x_1,\dots,x_n)s \in D\).}
}
{\vdash D \COL \Theta}
% Program
\infrule[T-Program]
{\vdash D : \Theta \;\;\;\; \Theta; \emptyset\vdash s : P \Rtab OK_{n}(P)}
{\vdash (D, s) : n}

\end{minipage}
\caption{Typing rules.}
\label{fig:typingrules}
\end{figure}

Figure~\ref{fig:typingrules} shows the typing rules of our type
system. \todo{Add explanation.}

\subsection{Type soundness}

% This subsection describes some theorems and lemmas for type safety.

The following theorem is the main result of the current paper.  The
proof is in Appendix~\ref{sec:proof}.

\begin{theorem}\label{thm1}
If $\vdash \langle D, s \rangle : n$ for some \(n\), then \(\langle D,
s \rangle\) is totally memory-leak free.
\end{theorem}

% This theorem says that a well typed program guarantees memory leak
% freedom.

The proof is based on the following lemmas: preservation and lack of
immediate overflow.

\begin{lemma}[Preservation]
\label{lem:preservation}
If $OK_{n}(P)$, $\Theta; \Gamma \vdash s : P$ and $\langle H,R,s,n
\rangle \xlongrightarrow{\rho} \langle H',R',s', n' \rangle$, then
there exists $P'$ such that (1) $ \Theta; \Gamma \vdash s' : P'$, (2)
\(P \xLongrightarrow{\rho} P'\), and (3) \(OK_{n'}(P')\).
\end{lemma}

%% \begin{lemma}[Preservation $\mathbf{II}$]%\label{preser}
%% If $OK_{n}(P)$, $\Theta ; \Gamma \vdash s : P$ and $\langle H,R,s,n \rangle
%% \rightarrow \langle H',R',s', n'
%% \rangle$, then $\exists P'$ s.t. \\
%% (1) $\Theta; \Gamma \vdash s' : P'$\\
%% (2) $ P \rightarrow^{*} P'  $\\
%% (3) $OK_{n'}(P')$
%% \end{lemma}

\begin{lemma}[Lack of immediate overflow]
\label
If $\Theta; \Gamma \vdash s \COL P$, \(\vdash D \COL \Theta\), and
\(\OK_n(P)\), then $\langle H, R, s, n \rangle
\not\xlongrightarrow{\rho}$ for any \(\rho\).
\end{lemma}
