
\section{Type system}
\label{sec:typesystem}

\subsection{Types}

The syntax of the types is as follows.
\[
\begin{array}{rlcl}
  P & (\mbox{behavioral types})&::=& {\bf 0} \tB P_{1};P_{2} \tB P_{1}+P_{2} \tB \Malloc \tB \Free \tB \alpha \tB \mu\alpha.P \\
  \Gamma & (\mbox{variable type environments}) &::=& \set{x_1, x_2, \dots, x_n}\\
  \Theta & (\mbox{function type environments}) &::=& \set{f_1\COL P_1,\dots,f_n\COL P_n}\\
\end{array}
\]
We use metavariable \(P\) for \emph{behavioral types}, which are
abstractions of the behavior of programs.  The type ${\bf 0}$
%<<<<<<< HEAD
%% represents the do-nothing behavior.  The type \(P_1;P_2\) represents
%% the sequential execution of \(P_1\) and \(P_2\).  The type \(P_1 +
%% P_2\) represents the choice between \(P_1\) and \(P_2\).  The type
%% \(\Malloc\) represents an allocation a memory cell exactly once.  The
%% =======
represents do-nothing behavior.  The type \(P_1;P_2\) represents the
sequential execution of \(P_1\) and \(P_2\).  The type \(P_1 + P_2\)
represents the choice between \(P_1\) and \(P_2\).  The type
\(\Malloc\) represents an allocation of a memory cell exactly once.  The
%>>>>>>> 4423a1ce9aed6c247aa532bd4d789da44f8e4939
type \(\Free\) represents a deallocation.  \(\alpha\) is a type
variable. The type \(\mu \alpha.P\) represents the behavior of
\(\alpha\) defined by the recursive equation \(\alpha = P\).

Type environments for variables, ranged over by \(\Gamma\), are set of
variables.  Since our interest is the behavior of a program, not the
types of values, a variable type environment does not carry
information on the types of variables.  We also designate type
environments for function names ranged over by \(\Theta\).  A function
type environment carries information on the behavior of functions
represented by behavioral types.  We often omit the curly braces
around variable and function environments.

%% For example, given \(\Malloc;\alpha;\Free\) as \(P\), a recursive type
%% is of the form \(\mu \alpha. \Malloc;\alpha;\Free\) and can go further
%% as \(\Malloc;(\mu\alpha.\Malloc;\alpha;\Free);\Free\).

%%  $P_{1};P_{2}$ is for sequential execution. $P_{1} + P_{2}$
%% is abstracted as conditional. $\Malloc$ is the behavior of a statement
%% that allocates a memory cell exactly once. $\Free$ is for deallocating
%% memory cell exactly once. $\mu \alpha. P$ is a recursive type. For
%% example, the behavior of the body of function $h$ in Figure~\ref{ex:np} is
%% abstracted as $\mu
%% \alpha. \Malloc;\Malloc;\Free;\Free;\alpha$. $\alpha$ is a type
%% variable and bounded to the recursive constructor $\mu \alpha$.

%% The function type is described as $(\tau_{1}, \dots, \tau_{n})P$,
%% which means a function receives some pointers as arguments and its
%% body is abstracted as a behavioral type $P$.

% Semantics of Behavioral Types %
% \subsection{Semantics of behavioral types}


\begin{figure}[t]

\begin{minipage}[t]{0.25\textwidth}
\infax[Tr-Skip]
{\mathbf{0};P \xlongrightarrow{\tau} P}
\end{minipage}
\begin{minipage}[t]{0.34\textwidth}
\infax[Tr-Malloc]
{\Malloc \xlongrightarrow{\Malloc} {\bf 0}}
\end{minipage}
\begin{minipage}[t]{0.3\textwidth}
\infax[Tr-Free]
{\Free \xlongrightarrow{\Free} {\bf 0}}
\end{minipage}

\vspace{2mm}

\begin{minipage}{0.33\textwidth}
\infax[Tr-Rec]
{\mu \alpha.P \xlongrightarrow{\tau}  [\mu \alpha . P/\alpha]  P}
\end{minipage}
\begin{minipage}{0.3\textwidth}
\infax[Tr-ChoiceL]
{P_{1} + P_{2} \xlongrightarrow{\tau} P_{1}}
\end{minipage}
\begin{minipage}{0.33\textwidth}
\infax[Tr-ChoiceR]
{P_{1} + P_{2} \xlongrightarrow{\tau} P_{2}}
\end{minipage}

\infrule[Tr-Seq]
{P_{1} \xlongrightarrow{\rho} P_{1}' }
{P_{1};P_{2} \xlongrightarrow{\rho} P_{1}';P_{2}}
\caption{Semantics of behavioral types.}
\label{fig:semBTypes}
\end{figure}

%% The notation $\rightarrow$ denotes that a behavioral type can be
%% reduced by the internal action. Notation $\xlongrightarrow{\alpha}$
%% means that a behavioral type can be reduced by executing $\alpha$
%% actions, and the $\alpha$ here is $\{\Malloc, \Free\}$.

The semantics of behavioral type are given by the labeled transition
system in Figure~\ref{fig:semBTypes}.  \(P \xlongrightarrow{\rho} P'\)
means that \(P\) can make an action \(\rho\) and, after \(P\) make the
action \(\rho\), \(P\) turns into \(P'\).

%% The behavioral types \(\Malloc\) and \(\Free\), intuitively, just
%% perform \(\Malloc\) and \(\Free\) actions respectively.  For the
%% type \(P_1 + P_2\), it only makes a choice between \(P_1\) and
%% \(P_2\), so it performs the internal action and then behave like
%% either \(P_1\) or \(P_2\).  The type \(\mu \alpha.P\) performs
%% nothing but substitution, so it makes an internal action and
%% behaves like a substitution \([\mu\alpha.P/\alpha]P\). For the
%% behavioral type \(P_1;P_2\), \(P_1\) make actions \(\rho\) until it
%% turns into \(0\); The type \(\bf{0};P_2\) The type \(\bf 0\)
%% represents do-nothing behavior, so the type \(\bf{0}\)\(;P\) should
%% perform an internal action \(\tau\) and then behave like the
%% following behavior \(P\). For the type \(\Malloc\) and \(\Free\),
%% they should perform \(\Malloc\) and \(\Free\) actions respectively
%% and, after performing actions they do nothing, so the following
%% behaves like \(\bf 0\). For the type \(P_1 + P_2\), it only makes a
%% choice between \(P_1\) and \(P_2\) and do nothing else, so it
%% performs the internal action and then behave like either \(P_1\) or
%% \(P_2\). The type \(\mu \alpha.P\) performs nothing but
%% substitution, so it makes an internal action and behaves like a
%% substitution \([\mu\alpha.P/\alpha]P\).

% Type Judgments

\subsection{Typing rules}

%%  We design the type system so that
%% this type judgment implies the property: when $s$ executes
%% $\Malloc$(resp.$\Free$), then $P$ is equivalent to
%% $\Malloc;P'$(resp.$\Free;P'$) for a type $P'$ such that $\Theta;
%% \Gamma \vdash s': P'$, where $s'$ is the continuation of $s$. This
%% property guarantees the behavioral type soundly abstracts the upper
%% bound of the consumed memory cells.

%% Typing rules are presented in Figure~\ref{fig:typingrules}. In the rule for assignment, the behavior of  $*x \leftarrow y$ is $\bf 0$. The rule for $\Free$ represents that the behavior of $\Free \Cirx$ is $\Free$. The rule T-Malloc represents that $\LET x = \MALLOC \; \IN s$ has the behavior $\Malloc;P$, where $P$ is the behavior of statement $s$. The rule for function call represents that function $f$ has the behavior $P$ which is the behavior of the body of this function .

%% \(\vdash D \COL \Theta\) denotes that the set \(D\) of definitions
%% has a type \(\Theta\).

We need several auxiliary definitions to present typing rules.  We
first define predicate \(\OK_n(P)\) that means that \(P\) describes
the behavior of a program that requires at most \(n\) memory cells.

\begin{myDef}[\(\sharp_{\rho}(\sigma)\)]
 \label{df:sharf}
\(\sharp_{\rho}(\sigma)\) is the number of \(\rho\) in the sequence
\(\sigma\).
\end{myDef}

%% begin{myDef}[OK(P)]
%%  \(OK_n(P)\) holds if a program executes sequence actions \( \sigma\) for any \(P'\) such that \(P\xLongrightarrow{\sigma}P'\), then the difference of \( \sharp_{malloc}(\sigma)\) and \( \sharp_{free}(\sigma)\) does not exceed the number \(n\).
%% \label{df:okn}
%% \
%%end{myDef}[OK_n(P)]

\begin{myDef}
\label{df:okn}
\(OK_{n}(P)\) holds if, for any \(P'\), if \(P
\xlongrightarrow{\sigma}P'\) then
\(\sharp_{malloc}(\sigma)-\sharp_{free}(\sigma)\le n\).
\end{myDef}

We also define subtyping.  Following the idea of Kobayashi et
al.~\cite{DBLP:journals/tcs/IgarashiK04}, we define subtyping by
using simulation between behavioral types.

 \begin{myDef}[Subtyping]
% $P_{1} \le P_{2}$ represents that $P_{1}$ is the subtype of $P_{2}$ and  means that:

\(P_1 \le P_2\) is the largest relation such that, for any \(P_1'\)
and \(\rho\), if \(P_1 \xlongrightarrow{\rho} P_1'\), then there
exists \(P_2'\) such that \(P_2 \xLongrightarrow{\rho} P_2'\) and
\(P_2 \le P_2'\).

\label{df:subtype}
\end{myDef}
%% At the end of $s$, memory leak freedom is guaranteed by $OK_{n}(P)$ ,where $P$ is the behavior of $s$. $OK_{n}(P)$ is defined as Definition~\ref{df:okn} in which $\sharp_{malloc}(\alpha)$ and $\sharp_{free}(\alpha)$ are functions to count the number of $\Malloc$ and $\Free$ actions in $\alpha$ respectively. This definition, intuitively, means at every running step the number of allocated memory cells will never go out of memory scope.
%% \begin{myDef}
%%   $OK_{n}(P) \iff \forall P',\; P \xlongrightarrow{\alpha}^{*}P'$ then $\sharp_{malloc}(\alpha)-\sharp_{free}(\alpha)\le n$.
%% \label{df:okn}
%% \end{myDef}

\begin{figure}[tp]
\begin{minipage}{\textwidth}

\begin{minipage}{0.5\textwidth}
\infax[T-Skip]
{\Theta ; \Gamma \vdash \SKIP : \mathbf{0}}
\end{minipage}
\begin{minipage}{0.5\textwidth}
\infrule[T-Seq]
{\Theta ; \Gamma \vdash s_{1} : P_{1} \Rtab \Theta ; \Gamma \vdash s_{2} : P_{2}}
{\Theta ; \Gamma \vdash s_{1} ; s_{2} : P_{1};P_{2} }
\end{minipage}

\vspace{2mm}

\begin{minipage}{0.5\textwidth}
\infax[T-Assign]
{\Theta ; \Gamma, x, y \vdash *x \leftarrow y : \mathbf{0} }
\end{minipage}
\begin{minipage}{0.5\textwidth}
\infax[T-Free]
{\Theta ; \Gamma, x \vdash \Free(x) : \Free}
\end{minipage}

\vspace{2mm}

\begin{minipage}{0.5\textwidth}
\infrule[T-Malloc]
{\Theta ; \Gamma,x \vdash s : P}
{\Theta ; \Gamma \vdash \LET x = \MALLOC \; \IN s  : \Malloc;P}
\end{minipage}
\begin{minipage}{0.5\textwidth}
\infrule[T-LetEq]
{\Theta ; \Gamma , x, y  \vdash s : P}
{\Theta ; \Gamma, y \vdash \LET x = y \; \IN s : P}
\end{minipage}

\vspace{2mm}

\begin{minipage}{0.5\textwidth}
\infrule[T-LetDeref]
{\Theta ; \Gamma , x, y  \vdash s : P}
{\Theta ; \Gamma, y \vdash \LET x = *y \; \IN s : P}
\end{minipage}
\begin{minipage}{0.5\textwidth}
\infrule[T-LetNull]
{\Theta ; \Gamma, x  \vdash s : P}
{\Theta ; \Gamma \vdash \LET x = \mathbf{null} \; \IN s : P}
\end{minipage}

\vspace{2mm}

\begin{minipage}{0.5\textwidth}
\infrule[T-IfNull]
{\Theta ; \Gamma, x \vdash s_{1} : P \andalso \Theta ; \Gamma, x \vdash s_{2} : P}
{\Theta ; \Gamma, x \vdash \IFNULL(x) \; \THEN s_{1}\; \ELSE s_{2} : P}
\end{minipage}
\begin{minipage}{0.5\textwidth}
\infax[T-Call]
{\Theta, f \COL P; \Gamma, \vec{x} \vdash f(\vec{x}) : P}
\end{minipage}

\vspace{2mm}

\infrule[T-Sub]
{\Theta ; \Gamma \vdash s : P_{1} \andalso P_{1} \le P_{2}}
{\Theta ; \Gamma \vdash s : P_{2}}

\infrule[T-Def]
{\DOM(D) = \DOM(\Theta)
\andalso
\Theta; x_1,\dots,x_n \vdash s \COL \Theta(f)
\mbox{ for each \(f \mapsto (x_1,\dots,x_n)s \in D\).}
}
{\vdash D \COL \Theta}
% Program
\infrule[T-Program]
{\vdash D : \Theta \andalso \Theta; \emptyset\vdash s : P \Rtab OK_{n}(P)}
{\vdash \langle D, s \rangle : n}

\end{minipage}
\caption{Typing rules.}
\label{fig:typingrules}
\end{figure}

The type judgment for statements is of the form $\Theta ; \Gamma
\vdash s : P$.  It expresses that, under the variable type environment
\(\Gamma\) and the function type environment \(\Theta\), the behavior
of $s$ is abstracted by behavior $P$.

%% \todo{Rewrite this paragraph} Figure~\ref{fig:typingrules} shows the
%% typing rules of our type system. In the rule for assignment, the
%% expression behaves like \(\bf 0\) . In the rule for \(\Free\), it
%% performs an allocation, so it behaves like \(\Free\).  In the rule for
%% \(\Malloc\), this let-expression performs an allocation and then
%% executes the statement \(s\), so its behavior like \(\Malloc;P\) where
%% the \(P\) is the behavior type of statement \(s\). In the rule for
%% other let-expressions like \rn{Tr-LetEq}, \rn{Tr-LetDref} and
%% \rn{Tr-LetNull}, they do not perform any allocation and deallocation,
%% so they behave like \(P\) which is the behavior of statement \(s\).

Figure~\ref{fig:typingrules} shows the typing rules.  It should be
easy to observe correspondence between a construct in a statement and
that in the behavioral type assigned to the statement.  For example,
rule \rn{T-Malloc} expresses that the behavior of \(\LET x =
\Malloc()\ \IN s\) is abstracted by \(\Malloc; P\) where \(P\) is the
behavior of \(s\); this statement first allocates a memory cell once,
and then executes \(s\).

\(\vdash D \COL \Theta\) is the judgment for function definitions.
This judgment represents that the behavior assigned to each function
symbol \(f\) in \(\Theta\) indeed abstracts the behavior of its
function body.  This property is guaranteed by \rn{T-Def}.

%<<<<<<< HEAD
%T%% ype judgment for programs, \(\vdash \langle D, s \rangle \COL n\),
%% expresses that the program requires at most \(n\) memory cells when it
%% is run.  In order to guarantee this, rule \rn{T-Prog} forces a side
%% condition \(\OK_n(P)\) where \(P\) is a behavioral type assigned to
%% \
%%(s\) under \(\Theta\) assigned to \(D\).
%=======
The type judgment for programs, \(\vdash \langle D, s \rangle \COL n\),
expresses that the program does not require more than \(n\) memory
cells when it is run.  In order to guarantee this, rule \rn{T-Prog}
forces a side condition \(\OK_n(P)\) where \(P\) is a behavioral type
assigned to \(s\) under \(\Theta\) assigned to \(D\).
%>>>>>>> 4423a1ce9aed6c247aa532bd4d789da44f8e4939

\subsection{Type soundness}

% This subsection describes some theorems and lemmas for type safety.

The following theorem states soundness of the type system.  The proof
is in the full version~\cite{fullversion}.

\begin{theorem}\label{thm1}
If $\vdash \langle D, s \rangle : n$ for some \(n\), then \(\langle D,
s \rangle\) is totally memory-leak free.
\end{theorem}

% This theorem says that a well typed program guarantees memory leak
% freedom.

The proof is based on the following lemmas: preservation and lack of
immediate overflow.

\begin{lemma}[Preservation]
\label{lem:preservation}
If $OK_{n}(P)$, $\Theta; \Gamma \vdash s : P$ and $\langle H,R,s,n
\rangle \xlongrightarrow{\rho} \langle H',R',s', n' \rangle$, then
there exists $P'$ such that (1) $ \Theta; \Gamma \vdash s' : P'$, (2)
\(P \xLongrightarrow{\rho} P'\), and (3) \(OK_{n'}(P')\).
\end{lemma}

%% \begin{lemma}[Preservation $\mathbf{II}$]%\label{preser}
%% If $OK_{n}(P)$, $\Theta ; \Gamma \vdash s : P$ and $\langle H,R,s,n \rangle
%% \rightarrow \langle H',R',s', n'
%% \rangle$, then $\exists P'$ s.t. \\
%% (1) $\Theta; \Gamma \vdash s' : P'$\\
%% (2) $ P \rightarrow^{*} P'  $\\
%% (3) $OK_{n'}(P')$
%% \end{lemma}

\begin{lemma}[Lack of immediate overflow]
\label{lem:immediateSafety}
If $\Theta; \Gamma \vdash s \COL P$, \(\vdash D \COL \Theta\), and
\(\OK_n(P)\), then $\langle H, R, s, n \rangle
\not\xlongrightarrow{\Malloc} \OVERFLOW$.
\end{lemma}
