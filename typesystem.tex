
\section{Type system}
\label{sec:typesystem}

\subsection{Syntax of types}
     \begin{eqnarray*}
       P (\mathit{behavioral\ types})::=&& {\bf 0} \tB P_{1};P_{2} \tB P_{1}+P_{2} \tB \Malloc\\
       &&\tB \Free \tB \alpha \tB \mu\alpha.P \\
      \sigma (\mathit{function\ types})::=&& (\tau_{1},\dots, \tau_{n}) P
     \end{eqnarray*}

The type $\bf 0$ abstracts the behavior of $\SKIP$ and means "does nothing". $P_{1};P_{2}$ is for sequential execution. $P_{1} + P_{2}$ is abstracted as conditional. $\Malloc$ is the behavior of a statement that allocates a memory cell exactly once. $\Free$ is for deallocating memory cell exactly once. $\mu \alpha. P$ is a recursive type. For example, the behavior of  the body of function $h$ in Figure 2 is abstracted as $\mu \alpha. \Malloc;\Malloc;\Free;\Free;\alpha$. $\alpha$ is a type variable and bounded to the recursive constructor $\mu \alpha$.

The only value in our paper is reference, and its type is $\mathbf{Ref}$.

The function type is described as $(\tau_{1}, \dots, \tau_{n})P$, which means a function receives some pointers as arguments and its body is abstracted as a behavioral type $P$.

% Semantics of Behavioral Types %
\subsection{Semantics of behavioral types}
The semantics of behavioral type are given by labeled transition system, and listed as follows:
    $$
        \mathbf{0};P \rightarrow P
    $$
    $$
          \Malloc \xlongrightarrow{\Malloc} 0
    $$
    $$
           \Free \xlongrightarrow{\Free} 0
    $$
    $$
          \mu \alpha.P \rightarrow  [\mu \alpha . P/\alpha]  P
    $$
   $$
          P_{1} + P_{2} \longrightarrow P_{1}
   $$
   $$
          P_{1} + P_{2} \longrightarrow P_{2}
   $$
   $$
           \frac{P_{1} \xlongrightarrow{\alpha} P_{1}' }
                 {P_{1};P_{2} \xlongrightarrow{\alpha} P_{1}';P_{2}}
   $$
The notation $\rightarrow$ denotes that a behavioral type can be reduced by the internal action. Notation $\xlongrightarrow{\alpha}$ means that a behavioral type can be reduced by executing $\alpha$ actions, and the $\alpha$ here is $\{\Malloc, \Free\}$.

% Type Judgments
\subsection{Typing rules}
The type judgment of our type system is given by the form $\Theta ;
\Gamma \vdash s : P$. It intuitively reads that the behavior of $s$ is
$P$ under $\Theta$ and $\Gamma$, where \(\Theta\) is a mapping from function variables to function types and \(\Gamma\) is an environment which includes variables.%%  We design the type system so that
%% this type judgment implies the property: when $s$ executes
%% $\Malloc$(resp.$\Free$), then $P$ is equivalent to
%% $\Malloc;P'$(resp.$\Free;P'$) for a type $P'$ such that $\Theta;
%% \Gamma \vdash s': P'$, where $s'$ is the continuation of $s$. This
%% property guarantees the behavioral type soundly abstracts the upper
%% bound of the consumed memory cells.

%% Typing rules are presented in Figure~\ref{fig:typingrules}. In the rule for assignment, the behavior of  $*x \leftarrow y$ is $\bf 0$. The rule for $\Free$ represents that the behavior of $\Free \Cirx$ is $\Free$. The rule T-Malloc represents that $\LET x = \MALLOC \; \IN s$ has the behavior $\Malloc;P$, where $P$ is the behavior of statement $s$. The rule for function call represents that function $f$ has the behavior $P$ which is the behavior of the body of this function .
\(\vdash D \COL \Theta\) denotes that the set \(D\) of definitions has a type \(\Theta\).
\begin{myDef}[\(\sharp_{\rho}(\sigma)\)]
 \(\sharp_{malloc}(\sigma)\) and \(\sharp_{free}(\sigma)\) are functions to count the number of \(\Malloc\) and \( \Free \) actions in a sequence \( \sigma\) respectively.
 \label{df:sharf}
 \end{myDef}

%% begin{myDef}[OK(P)]
%%  \(OK_n(P)\) holds if a program executes sequence actions \( \sigma\) for any \(P'\) such that \(P\xLongrightarrow{\sigma}P'\), then the difference of \( \sharp_{malloc}(\sigma)\) and \( \sharp_{free}(\sigma)\) does not exceed the number \(n\).
%% \label{df:okn}
%% \
%%end{myDef}[OK_n(P)]
 \begin{myDef}
   \(OK_{n}(P) \iff \forall P',\; P \xlongrightarrow{\sigma}P'\) then \(\sharp_{malloc}(\sigma)-\sharp_{free}(\sigma)\le n\).
 \label{df:okn}
 \end{myDef}

 \begin{myDef}[Subtype]
$P_{1} \le P_{2}$ represents that $P_{1}$ is the subtype of $P_{2}$ and  means that: \\
(1) if $P_{1} \xlongrightarrow{\alpha}  P_{1}'$ then $\exists P_{2}' $ s.t. $P_{2} \overset{\text{$\alpha$}}{\Longrightarrow} P_{2}'$ and $ P_{1}' \le P_{2}' $\\
(2) if $P_{1} \rightarrow P_{1}'$ then $\exists P_{2}'$ s.t. $P_{2} \rightarrow^{*} P_{2}'$ and  $P_{1}' \le P_{2}'$\\
where $\overset{\text{$\alpha$}}{\Longrightarrow}$ means that: $\rightarrow^{*} \xlongrightarrow{\alpha} \rightarrow^{*}$.
\label{df:subtype}
\end{myDef}
%% At the end of $s$, memory leak freedom is guaranteed by $OK_{n}(P)$ ,where $P$ is the behavior of $s$. $OK_{n}(P)$ is defined as Definition~\ref{df:okn} in which $\sharp_{malloc}(\alpha)$ and $\sharp_{free}(\alpha)$ are functions to count the number of $\Malloc$ and $\Free$ actions in $\alpha$ respectively. This definition, intuitively, means at every running step the number of allocated memory cells will never go out of memory scope.
%% \begin{myDef}
%%   $OK_{n}(P) \iff \forall P',\; P \xlongrightarrow{\alpha}^{*}P'$ then $\sharp_{malloc}(\alpha)-\sharp_{free}(\alpha)\le n$.
%% \label{df:okn}
%% \end{myDef}

\begin{figure}[tp]
\begin{minipage}{\textwidth}

% Skip type
\infax[T-Skip]
{\Theta ; \Gamma \vdash \SKIP : \mathbf{0}}
% Sequence type
\infrule[T-Seq]
{\Theta ; \Gamma \vdash s_{1} : P_{1} \Rtab \Theta ; \Gamma \vdash s_{2} : P_{2}}
{\Theta ; \Gamma \vdash s_{1} ; s_{2} : P_{1};P_{2} }
% Assignment type
\infrule[T-Assign]
{\Theta ; \Gamma \vdash y  \Rtab \Theta ; \Gamma \vdash x }
{\Theta ; \Gamma \vdash *x \leftarrow y : \mathbf{0} }
% Free(deallocate) type
\infrule[T-Free]
{\Theta ; \Gamma \vdash x  }
{\Theta ; \Gamma \vdash \Free(x) : \Free}
% Malloc type
\infrule[T-Malloc]
{\Theta ; \Gamma,x \vdash s : P}
{\Theta ; \Gamma \vdash \LET x = \MALLOC \; \IN s  : \Malloc;P}
% Let eq type
\infrule[T-LetEq]
{\Theta ; \Gamma \vdash y   \Rtab \Theta ; \Gamma , x  \vdash s : P}
{\Theta ; \Gamma \vdash \LET x = y \; \IN s : P}
% Dereference type
\infrule[T-LetDref]
{\Theta ; \Gamma \vdash y  \Rtab \Theta ; \Gamma , x  \vdash s : P}
{\Theta ; \Gamma \vdash \LET x = *y \; \IN s : P}
% Let NULL type
\infrule[T-LetNull]
{\Theta ; \Gamma, x  \vdash s : P}
{\Theta ; \Gamma \vdash \LET x = \mathbf{null} \; \IN s : P}
% Subtyping
\infrule[T-Sub]
{\Theta ; \Gamma \vdash s : P_{1} \Rtab P_{1} \le P_{2}}
{\Theta ; \Gamma \vdash s : P_{2}}
 % ifnull s then s type
\infrule[T-IfNull]
{\Theta ; \Gamma \vdash x    \ \ \ \  \Theta ; \Gamma \vdash s_{1} : P \ \ \ \ \Theta ; \Gamma \vdash s_{1} : P}
{\Theta ; \Gamma \vdash \IFNULL(x) \; \THEN s_{1}\; \ELSE s_{2} : P}
% Function call type
\infrule[T-Call]
{ \Theta(f) = P}
{\Theta; \Gamma, \vec{x} : \vec{\tau} \vdash f(\vec{x}) : P}
% Program
\infrule[T-Program]
{\vdash D : \Theta \;\;\;\; \Theta; \emptyset\vdash s : P \Rtab OK_{n}(P)}
{\vdash (D, s) : n}

\end{minipage}
\caption{Typing Rules.}
\label{fig:typingrules}
\end{figure}


\subsection{Type soundness}

% This subsection describes some theorems and lemmas for type safety.

The following theorem is the main result of the current paper.  The
proof is in Appendix~\ref{sec:proof}.

\begin{theorem}\label{thm1}
If $\vdash \langle D, s \rangle : n$ for some \(n\), then \(\langle D,
s \rangle\) is totally memory-leak free.
\end{theorem}

% This theorem says that a well typed program guarantees memory leak
% freedom.

The proof is based on the following lemmas: preservation and lack of
immediate overflow.

\begin{lemma}[Preservation]
\label{lem:preservation}
If $OK_{n}(P)$, $\Theta; \Gamma \vdash s : P$ and $\langle H,R,s,n
\rangle \xlongrightarrow{\rho} \langle H',R',s', n' \rangle$, then
there exists $P'$ such that (1) $ \Theta; \Gamma \vdash s' : P'$, (2)
\(P \xLongrightarrow{\rho} P'\), and (3) \(OK_{n'}(P')\).
\end{lemma}

%% \begin{lemma}[Preservation $\mathbf{II}$]%\label{preser}
%% If $OK_{n}(P)$, $\Theta ; \Gamma \vdash s : P$ and $\langle H,R,s,n \rangle
%% \rightarrow \langle H',R',s', n'
%% \rangle$, then $\exists P'$ s.t. \\
%% (1) $\Theta; \Gamma \vdash s' : P'$\\
%% (2) $ P \rightarrow^{*} P'  $\\
%% (3) $OK_{n'}(P')$
%% \end{lemma}

\begin{lemma}[Lack of immediate overflow]
\label
If $\Theta; \Gamma \vdash s \COL P$, \(\vdash D \COL \Theta\), and
\(\OK_n(P)\), then $\langle H, R, s, n \rangle
\not\xlongrightarrow{\rho}$ for any \(\rho\).
\end{lemma}
